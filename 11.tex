\documentclass[handout,xcolor=dvipsnames]{beamer}
\usepackage{multicol}
\usepackage{xy}
\everymath{\displaystyle}
\mode<presentation>
{\usetheme{Warsaw}\setbeamercovered{dynamic}}
\usecolortheme{crane}
\usepackage{beamerfoils}
\pgfdeclareimage[height=1in]{university-logo}{ISULogo}
\logo{\pgfuseimage{university-logo}}
\setbeamertemplate{navigation symbols}{}
\title[\S10]{Section 11\\Binomial Probability}
\author{Dr Marcus Bishop}
\subject{Math 104}
\beamerdefaultoverlayspecification{<+->}
\theoremstyle{definition}
\newtheorem{remark}{Remark}
\newtheorem{impact}{Impact}
\newtheorem{situation}{Situation}
\newtheorem{question}{Question}
\usepackage{arev}
\usepackage{tensor}
\newcommand\npr[2]{\tensor[_{#1}]P{_{#2}}}
\newcommand\ncr[2]{\tensor[_{#1}]C{_{#2}}}
\usepackage{cancel}
\newcommand{\hs}{\alert{\varheart}}
\newcommand{\ds}{\alert{\vardiamond}}
\newcommand{\s}{\spadesuit}
\newcommand{\cs}{\clubsuit}
\begin{document}
\begin{frame}\titlepage\end{frame}
\LogoOff

\begin{frame}{Binomial theorem}
\begin{itemize}
\item Recall FOIL method:
\begin{align*}
\left(x+y\right)^2
&=\left(x+y\right)\left(x+y\right)\\
\only<+->{&=x^2+xy+yx+y^2\\}
\only<+->{&=x^2+2xy+y^2}
\end{align*}
\begin{remark}
\begin{itemize}
\item $\left(x+y\right)^2\alert{\ne}x^2+y^2$
as commonly mistaken
\item Rather $\left(x+y\right)^2=x^2+2xy+y^2$
as shown above
\end{itemize}
\end{remark}
\end{itemize}
\end{frame}

\begin{frame}
\begin{itemize}
\item Similarly for third power:
\begin{align*}
\left(x+y\right)^3
&=\left(x+y\right)\left(x+y\right)\left(x+y\right)\\
\only<+->{&=\left(x+y\right)^2\left(x+y\right)\\}
\only<+->{&=\left(x^2+2xy+y^2\right)\left(x+y\right)\\}
\only<+->{&=x^3+x^2y+2xyx+2xy^2+y^2x+y^3\\}
\only<+->{&=x^3+3x^2y+3xy^2+y^3\\}
\only<+->{&=\alert{1}x^3+\alert{3}x^2y+\alert{3}xy^2+\alert{1}y^3}
\end{align*}
\item Similarly for fourth power:
\begin{align*}
\left(x+y\right)^4&=x^4+4x^3y+6x^2y^2+4xy^3+y^4\\
\only<+->{&=\alert{1}x^4+\alert{4}x^3y+\alert{6}x^2y^2
+\alert{4}xy^3+\alert{1}y^4}
\end{align*}
\item Note that coefficients $1,4,6,4,1$ the numbers
in fourth row of Pascal's triangle
\end{itemize}
\end{frame}



\end{document}
