\documentclass[answers,12pt]{exam}
\usepackage{color,rotating}
\CorrectChoiceEmphasis{\color{red}\bfseries}
\everymath{\displaystyle}
\usepackage{amsmath,charter}
\usepackage{eulervm}
\title{Math 104 Outcomes Assessment}\author{}\date{Spring 2015}
\usepackage{tensor}
\newcommand\npr[2]{\tensor[_{#1}]P{_{#2}}}
\newcommand\ncr[2]{\tensor[_{#1}]C{_{#2}}}
\begin{document}
\maketitle
\begin{questions}

\uplevel{The first question measures the students' ability
to use binomial coefficients to solve problems.
Correct answers are shown in red.}
\question
\begin{parts}
\part What is $\ncr{12}{5}$?\\
\begin{oneparchoices}
\choice $60$ % 12 * 5
\correctchoice $792$
\choice $248,832$ % 12^5
\choice $95,040$ % P(12,5)
\choice $3,991,680$ % P(12,7)
\end{oneparchoices}
\part A poker hand is called a {\em heart flush}
if all five cards have suit $\heartsuit$. What is
the number of possible heart flushes?\\
\begin{oneparchoices}
\correctchoice $1287$
\choice $5108$ % number of mere flushes
\choice $5148$ % number of flushes
\choice $154,440$ % P(13,5)
\choice $2,598,960$ % C(52,5)
\end{oneparchoices}
\part In how many ways can a four-member subcommittee
of a eight-member committee be chosen?\\
\begin{oneparchoices}
%\choice $4$ % duh!
\choice $24$ % 4!
\choice $32$ % 4 x 8
\correctchoice $70$
\choice $256$ % 2^8
\choice $1680$ % P(8,4)
\end{oneparchoices}
\part Playing poker, you have the cards
$K\heartsuit,K\clubsuit,2\diamondsuit,2\clubsuit,8\heartsuit$.
If you exchange $2\diamondsuit,2\clubsuit,8\heartsuit$ for three
new cards, then what is the probability that your new hand
will have four-of-a-kind?\\
\begin{oneparchoices}
\choice $0.0025$ % something-over-aces full house
\correctchoice $0.0028$
\choice $0.0851$ % aces-over-fours full house
\choice $0.1221$ % exactly one ace 
\choice $0.1274$ % three-of-a-kind or better
\end{oneparchoices}
\end{parts}

\uplevel{Each part worth one point, the average score
on the question was $2.85$ out of 3 points.
The percents of students correctly
answering parts (a), (b), (c), and (d) were
$94\%$, $67\%$, $75\%$, and $50\%$ respectively.}

\uplevel{The second question measures the students' ability
to calculate the expected value of a simple experiment.}

\question This question deals with the game of farkle.
You initially roll $1,2,2,2,5,6$.
\begin{parts}
\part What is the value of your initial roll?\\
\begin{oneparchoices}
\choice $150$ % didn't count 2s
\choice $200$ % only counted 2s
\choice $300$ % didn't count 5
\correctchoice $350$
\choice $1800$ % sum times 100
\end{oneparchoices}
\part You decide to reroll the $6$. What are your expected
winnings for this turn? Round your answer to the nearest whole number.\\
\begin{oneparchoices}
\choice $50$ % didn't add previous score to outcomes
\correctchoice $142$
\choice $283$ % 850/3 rather than 850/6
\choice $350$ % previous score
\choice $450$ % outcome if 1 rolled
\end{oneparchoices}
\end{parts}

\uplevel{Each part worth 1~point, the average score on the
question was $0.96$ out of two~points. The percents of students
correctly answering parts (a) and (b) were
$75\%$ and $21\%$ respectively.}

\end{questions}
\end{document}
