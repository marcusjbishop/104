\documentclass[answers,12pt]{exam}
\usepackage{color,rotating}
\CorrectChoiceEmphasis{\color{red}\bfseries}
\usepackage[euler-digits]{eulervm}
\everymath{\displaystyle}
\usepackage{amsmath,charter}
\title{Math 104 Quiz 4\\Red Form}
\date{Wednesday 19 November}
\usepackage{tensor}
\newcommand\npr[2]{\tensor[_{#1}]P{_{#2}}}
\newcommand\ncr[2]{\tensor[_{#1}]C{_{#2}}}
\begin{document}
\maketitle
\begin{center}
\fbox{\fbox{\parbox{5.5in}{\centering
Using a Number~2 pencil record your student ID on your answer sheet.
Then answer the questions below on the answer sheet.
This quiz has \numquestions~questions printed on \numpages~pages.
All answer choices have been rounded to four decimal places.
}}}
\end{center}

\begin{questions}

\question What is $\ncr{12}{5}$?\\
\begin{oneparchoices}
\choice $60$ % 12 * 5
\correctchoice $792$
\choice $248,832$ % 12^5
\choice $95,040$ % P(12,5)
\choice $3,991,680$ % P(12,7)
\end{oneparchoices}

\question How many subsets does the set
$\left\{a,b,c,d,e,f,g\right\}$ have?\\
\begin{oneparchoices}
\choice $7$ % size of set
\choice $21$ % C(7,2)
\choice $14$ % 2 * 7
\choice $49$ % 7 * 7
\correctchoice $128$
\end{oneparchoices}

\question In how many ways can a four-member subcommittee
of a seven-member committee be chosen?\\
\begin{oneparchoices}
\choice $4$ % duh!
\choice $24$ % 4!
\correctchoice $35$
\choice $128$ % 2^7
\choice $840$ % P(7,4)
\end{oneparchoices}

\question Customers order pizzas at Marcus's Pizza
by selecting any number of five possible toppings.
How many different ways to order a pizza are there?\\
\begin{oneparchoices}
\choice $5$ % number of toppings
\choice $10$ % C(5,2) or 2*5
\choice $20$ % P(5,2)
\choice $25$ % 5 * 5
\correctchoice $32$
\end{oneparchoices}

\question You have ten books, but only six
of them will fit on your shelf. In how many 
ways can six of the books be arranged on the shelf?\\
\begin{oneparchoices}
\choice $210$ % C(10,6)
\choice $720$ % 6!
\choice $5040$ % P(10,4)
\correctchoice $151,200$
\choice $3,628,800$ % 10!
\end{oneparchoices}

\question A poker hand is called a {\em spade flush}
if all five cards have suit $\spadesuit$. What is
the number of possible spade flushes?\\
\begin{oneparchoices}
\correctchoice $1287$
\choice $5148$ % number of flushes
\choice $5108$ % number of mere flushes
\choice $154,440$ % P(13,5)
\choice $2,598,960$ % C(52,5)
\end{oneparchoices}

\question 
How many five-card poker hands have two-pair but nothing better?\\
\begin{oneparchoices}
\choice $2,808$ % didn't choose last card
\correctchoice $123,552$
\choice $247,104$ % used 13*12 rather than C(13,2)
\choice $2,598,960$ % C(52,5)
\choice $65,812,032$ % C(52,2)C(48,2)C(44,1)
\end{oneparchoices}

\question Playing poker, you have the cards
$K\heartsuit,K\clubsuit,2\diamondsuit,2\clubsuit,8\heartsuit$.
If you exchange $8\heartsuit$ for a new card,
then what is the probability that new card
will complete your hand to a full house?\\
\begin{oneparchoices}
\choice $\frac{1}{52}$
\choice $\frac{1}{26}$ % only one of K,2 and didn't eliminate
\choice $\frac{2}{47}$ % counted only one of K,2
\choice $\frac{1}{13}$ % didn't eliminate 5 known cards from total
\correctchoice $\frac{4}{47}$
\end{oneparchoices}

\question Playing poker, you have the cards
$K\heartsuit,K\clubsuit,2\diamondsuit,2\clubsuit,8\heartsuit$.
If you exchange $2\diamondsuit,2\clubsuit,8\heartsuit$ for three
new cards, then what is the probability that the new hand
will have four-of-a-kind?\\
\begin{oneparchoices}
\choice $0.0025$ % something-over-kings full house
\correctchoice $0.0028$
\choice $0.0851$ % kings-over-twos full house
\choice $0.1221$ % exactly one king
\choice $0.1274$ % three-of-a-kind or better
\end{oneparchoices}

\question Two of the $24$ sodas in the refrigerator
have winning bottlecaps. If you randomly select two
bottles, then what is the probability that at least one bottle
will have a winning bottlecap?\\
\begin{oneparchoices}
\choice $0.0036$ % exactly two winners
\choice $0.1594$ % exactly one winner
\correctchoice $0.163$
\choice $0.0417$ % 1/24
\choice $0.837$ % no winners
\end{oneparchoices}

\end{questions}
\end{document}
