\documentclass[11pt]{article}
\usepackage{mdwlist}
\usepackage[euler-digits]{eulervm}
\usepackage{charter,amsmath,amssymb,breakurl}
\usepackage[letterpaper,margin=1in]{geometry}
\usepackage{multicol}
\everymath{\displaystyle}
\author{}\date{Due in class Monday 20 October}
\title{Math 104 Worksheet 2}\author{}
\begin{document}
\maketitle
\pagestyle{empty}
Let $E=\left\{a,b,d,g\right\}$,
$F=\left\{b,c,f,g,h\right\}$, and
$G=\left\{a,c,d,g,h,i\right\}$.
Calculate the following.
\begin{enumerate}
\item $E\cup F$\vspace{.25in}
\item $E\cap G$\vspace{.25in}
\item $F\cap G$\vspace{.25in}
\item $E\cup F\cup G$\vspace{.25in}
\item $\left(E\cup F\right)\cap G$\vspace{.25in}
\item $\left(E\cap F\right)\cup G$\vspace{.25in}
\item $\left(E\cup F\right)\cap\left(E\cup G\right)$\vspace{.25in}
\item $E\cap\left(F\cup G\right)$\vspace{.25in}
\suspend{enumerate}

The Inclusion-Exclusion Theorem for three sets $E,F,G$ reads
\[\left|E\cup F\cup G\right|
=\left|E\right|+\left|F\right|+\left|G\right|
-\left|E\cap F\right|
-\left|E\cap G\right|
-\left|F\cap G\right|
+\left|E\cap F\cap G\right|\]
Use the Inclusion-Exclusion Theorem to answer the following questions.

\resume{enumerate}
\item Use a Venn Diagram to explain why
the Inclusion-Exclusion Theorem is true.
\item Suppose $5$ students are in band, orchestra, and chorus.
Suppose further that 
\end{enumerate}
\end{document}
