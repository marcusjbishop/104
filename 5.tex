\documentclass[handout]{beamer}
\usepackage{multicol}
\usepackage{xy}
\everymath{\displaystyle}
\mode<presentation>
{\usetheme{Warsaw}\setbeamercovered{dynamic}}
\usecolortheme{crane}
\usepackage{beamerfoils}
\pgfdeclareimage[height=1in]{university-logo}{ISULogo}
\logo{\pgfuseimage{university-logo}}
\setbeamertemplate{navigation symbols}{}
\title[\S5]{Section 5\\Tree diagrams}
\author{Dr Marcus Bishop}
\subject{Math 104}
\beamerdefaultoverlayspecification{<+->}
\theoremstyle{definition}
\newtheorem{remark}{Remark}
\newtheorem{impact}{Impact}
\newtheorem{notation}{Notation}
\usepackage{arev}
\begin{document}
\begin{frame}\titlepage\end{frame}
\LogoOff

\begin{frame}{Coin toss and paper-rock-scissors}
\begin{itemize}
\item Experiment: toss coin \alert{and} randomly choose
one of paper, rock, scissors
\item Can enumerate possible outcomes using tree
\[\begin{xy}<1cm,0cm>:
(0,0)="0";
(-2,-1)="-21";
(2,-1)="21";
"0";"-21"*+!D{H}**\dir{-};
"0";"21"*+!D{T}**\dir{-};
"-21";(-3,-2)*{P}**\dir{-};
"-21";(-2,-2)*{R}**\dir{-};
"-21";(-1,-2)*{S}**\dir{-};
"21";(1,-2)*{P}**\dir{-};
"21";(2,-2)*{R}**\dir{-};
"21";(3,-2)*{S}**\dir{-};
\end{xy}\]
\item Possible outcomes correspond with paths from root to leaves
\item Can read outcome by reading letters along path:
\[HP,HR,HS,TP,TR,TS\]
\item Outcomes appear in alphabetical order
\end{itemize}
\end{frame}

\begin{frame}{Marble draw with replacement}
\[\begin{xy}<1cm,0cm>:
(0,0)="0";
(-3,-1)="-31";
(0,-1)="01";
(3,-1)="31";
"0";"-31"*+!D{B}**\dir{-};
"0";"01"*+!D{G}**\dir{-};
"0";"31"*+!D{R}**\dir{-};
"-31";(-4,-2)*{B}**\dir{-};
"-31";(-3,-2)*{G}**\dir{-};
"-31";(-2,-2)*{R}**\dir{-};
"01";(-1,-2)*{B}**\dir{-};
"01";(0,-2)*{G}**\dir{-};
"01";(1,-2)*{R}**\dir{-};
"31";(2,-2)*{B}**\dir{-};
"31";(3,-2)*{G}**\dir{-};
"31";(4,-2)*{R}**\dir{-};
\end{xy}\]
\begin{itemize}
\item Experiment: randomly select marble from bag
containing one each of blue, green, red,
\alert{replace marble}, select another
\item Nine possible outcomes
\item $P\left\{\text{two blues}\right\}
\only<+->{=1/9}$
\item $P\left\{\text{at least one blue}\right\}
\only<+->{=5/9}$
\item Outcomes with at least one blue:
$BB,BG,BR,GB,RB$
\end{itemize}
\end{frame}

\begin{frame}{Marble draw without replacement}
\[\begin{xy}<1cm,0cm>:
(0,0)="0";
(-3,-1)="-31";
(0,-1)="01";
(3,-1)="31";
"0";"-31"*+!D{B}**\dir{-};
"0";"01"*+!D{G}**\dir{-};
"0";"31"*+!D{R}**\dir{-};
"-31";(-3,-2)*{G}**\dir{-};
"-31";(-2,-2)*{R}**\dir{-};
"01";(-1,-2)*{B}**\dir{-};
"01";(1,-2)*{R}**\dir{-};
"31";(2,-2)*{B}**\dir{-};
"31";(3,-2)*{G}**\dir{-};
\end{xy}\]
\begin{itemize}
\item Experiment: randomly select marble from bag
then select another \alert{without replacing} first
\item Six possible outcomes
\item $P\left\{\text{two blues}\right\}
\only<+->{=0}$
\item $P\left\{\text{at least one blue}\right\}
\only<+->{=4/6=2/3}$
\item Outcomes with at least one blue:
$BG,BR,GB,RB$
\end{itemize}
\end{frame}

\begin{frame}
\[\begin{xy}<1cm,0cm>:
(0,0)="0";
(-3,-1)="-31";
(0,-1)="01";
(3,-1)="31";
"0";"-31"*+!D{B}**\dir{-};
"0";"01"*+!D{G}**\dir{-};
"0";"31"*+!D{R}**\dir{-};
"-31";(-3,-2)*{G}**\dir{-};
"-31";(-2,-2)*{R}**\dir{-};
"01";(-1,-2)*{B}**\dir{-};
"01";(1,-2)*{R}**\dir{-};
"31";(2,-2)*{B}**\dir{-};
"31";(3,-2)*{G}**\dir{-};
\end{xy}\]
\begin{itemize}
\item Experiment essentially equivalent to simply
selecting two marbles from bag
\item Outcomes of selecting two balls:
\begin{equation}\label{RGBWithout}
\left\{B,G\right\},\left\{B,R\right\},\left\{G,R\right\}
\end{equation}
\item Outcomes of choosing one after another without replacement:
\begin{equation}\label{RGBWith}
BG,BR,GB,GR,RB,RG
\end{equation}
\item Each set in (\ref{RGBWithout}) appears in
(\ref{RGBWith}) \alert{twice}
\item Order of marble listing irrelevant in
(\ref{RGBWithout}), matters in (\ref{RGBWith})
\end{itemize}
\end{frame}

\begin{frame}{Observations}
\begin{itemize}
\item $6=2\cdot 3$ possible outcomes in coin/paper/rock/scissors example:
\[\begin{xy}<.75cm,0cm>:
(0,0)="0";
(-2,-1)="-21";
(2,-1)="21";
"0";"-21"*+!D{H}**\dir{-};
"0";"21"*+!D{T}**\dir{-};
"-21";(-3,-2)*{P}**\dir{-};
"-21";(-2,-2)*{R}**\dir{-};
"-21";(-1,-2)*{S}**\dir{-};
"21";(1,-2)*{P}**\dir{-};
"21";(2,-2)*{R}**\dir{-};
"21";(3,-2)*{S}**\dir{-};
\end{xy}\]
\item Reason: tree has two \alert{identical} branches, each with three leaves
\item $9=3\cdot 3$ outcomes in replaced marble example:
\[\begin{xy}<.75cm,0cm>:
(0,0)="0";
(-3,-1)="-31";
(0,-1)="01";
(3,-1)="31";
"0";"-31"*+!D{B}**\dir{-};
"0";"01"*+!D{G}**\dir{-};
"0";"31"*+!D{R}**\dir{-};
"-31";(-4,-2)*{B}**\dir{-};
"-31";(-3,-2)*{G}**\dir{-};
"-31";(-2,-2)*{R}**\dir{-};
"01";(-1,-2)*{B}**\dir{-};
"01";(0,-2)*{G}**\dir{-};
"01";(1,-2)*{R}**\dir{-};
"31";(2,-2)*{B}**\dir{-};
"31";(3,-2)*{G}**\dir{-};
"31";(4,-2)*{R}**\dir{-};
\end{xy}\]
\item Reason: tree has three \alert{identical} branches,
each with three leaves
\end{itemize}
\end{frame}

\begin{frame}{Counting principle}
\begin{lemma}[Counting Principle]
If an experiment consists of two parts, first
part has $m$ outcomes, second part
has $n$ outcomes, then experiment
has $mn$ outcomes
\end{lemma}
\begin{definition}
\begin{itemize}
\item Set of all possible outcomes of experiment called
its \alert{sample space}
\item Each possible outcome of experiment called
a \alert{sample point}
\end{itemize}
\end{definition}
\end{frame}

\begin{frame}{Example}
\begin{itemize}
\item Experiment: flip two coins
\item First coin has two outcomes, $H,T$
\item Second coin has two outcomes, $H,T$
\item Thus $2\cdot 2=4$ possible outcomes
\item To construct sample space, use tree diagram:
\end{itemize}
\end{frame}

\end{document}
