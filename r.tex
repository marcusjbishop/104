\documentclass[12pt]{exam}
\usepackage{mdwlist}
\usepackage{graphicx,multicol}
\usepackage[euler-digits]{eulervm}
\usepackage{charter,amsmath,amssymb,breakurl}
\usepackage[letterpaper,margin=1in]{geometry}
\author{}\date{}
\title{Review problems}
\begin{document}
\maketitle
\pagestyle{empty}
\begin{questions}
\uplevel{\bf Basic properties of probability.}
\uplevel{\bf Empirical probability.}
\uplevel{\bf Probabilities and sample spaces of simple events.}
\uplevel{\bf Odds in favor of and against an event.}
\uplevel{\bf Expectation.}

\uplevel{\bf Fair ticket price.}
\question (From exam 1)
To raise money the marching band sells 500~raffle tickets
for $\$3$ each. They will randomly select a ticket holder to receive
a prize of $\$200$.
\begin{parts}
\part Calculate the fair ticket price.
\part Calculate the expected proceeds for the band.
\end{parts}
\begin{solution}
The fair price $P$ is the price for which the expectation
for the ticket holder is $0$. This can be found by solving
$\left(200-P\right)\frac{1}{500}-P\left(\frac{499}{500}\right)=0$
for $P$, which gives $P=\frac{2}{5}$, or 40~cents.
Another way to calculate the fair price
is to add the actual ticket price $\$3$
to the actual expectation $\$-2.60$, which also gives $\$0.40$.
The proceeds for the band are $3\left(\$500\right)-\$200=\$1300$.
From their point of view this is the only outcome, so $\$1300$
is also the expected proceeds for the band.
\end{solution}

\uplevel{\bf Tree diagrams.}
\question You need to hire a database administrator
(DBA) and two programmers. You would like to offer the DBA
position to one of Alice, Barrett, Carl and the programming
positions to two of Quinn, Riley, Steven, Tamara.
However Alice is married to Quinn and they can only accept
positions if {\em both} are offered positions, since they will have
to move.
\begin{parts}
\part In how many ways can the jobs be assigned?
\part In how many ways is Alice offered a job?
\part What is the probability that Riley will be offered a job?
\end{parts}
\begin{solution} $18,6,\frac{5}{9}$\end{solution}

\uplevel{\bf Counting principle.}
\uplevel{\bf Inclusion-exclusion formula.}
\uplevel{\bf Venn diagrams.}
\uplevel{\bf Independence.}
\question You draw a card from a shuffled deck. You don't replace
the card. Then your opponent draws a card.
\begin{parts}
\part Are the events that
you drew a king and that she drew a king independent?
\part Are the events that
you drew a king and that she drew a heart independent?
\end{parts}
\begin{solution} No, no\end{solution}

\question You roll a regular six-sided die twice. Calculate
the probability that you roll a 1 or 2 the first time and a
3, 4, 5, or 6 the second time.
\begin{solution} The probability of rolling 1 or 2 the first
time is $\frac{2}{6}=\frac{1}{3}$.
The probability of rolling 3, 4, 5, or 6 the second time
is $\frac{4}{6}=\frac{2}{3}$.
Since the two events are independent, the probability of both
events occurring is $\frac{1}{3}\cdot\frac{2}{3}=\frac{2}{9}$. \end{solution}

\uplevel{\bf Craps.}
\question Do the pass bettors or the don't pass bettors have
a greater probability of winning?
\begin{solution} The pass bettors win with probability
$\frac{244}{495}$ while the don't pass bettors win with probability
$\frac{251}{495}$. Thus the don't pass bettors have a slight edge.
\end{solution}

\question Suppose you are the shooter and you roll 
a sum of eight on your opening
roll. What is the probability that you win the game?
\begin{solution} At this stage you continue rolling until
a sum of either seven or eight comes up, with you winning
in the latter case. Since there are six ways for a sum of
seven to come up and five ways for a sum of eight to come up,
you will win the game with probability $\frac{5}{11}$.
\end{solution}

\uplevel{\bf Union, intersection, size.}
\question If $E=\left\{\alpha,\gamma,\zeta\right\}$,
$F=\left\{\alpha,\beta,\delta,\xi\right\}$, and
$G=\left\{\beta,\zeta,\xi\right\}$, then calculate the following.
\begin{parts}
\part $E\cap F$
\part $E\cup G$
\part $\left|E\cup\left(F\cap G\right)\right|$
\end{parts}
\begin{solution} $\left\{\alpha\right\},\left\{
\alpha,\beta,\delta,\zeta,\xi\right\},3$
\end{solution}

\uplevel{\bf Conditional probability.}
\question Your opponent draws a card from a full shuffled deck.
She doesn't replace the card.
Then you draw a card. Calculate the following.
\begin{parts}
\part The probability that you draw a queen given that she drew a queen.
\begin{solution} $\frac{3}{51}$ \end{solution}
\part The probability that she drew a queen and you drew a queen.
\begin{solution} $\frac{4}{52}\cdot\frac{3}{51}=\frac{1}{221}$ \end{solution}
\part The probability that you draw a heart given that she drew a queen.
\begin{solution} This is impossible to calculate without knowing
whether she drew the card $Q\heartsuit$. \end{solution}
\end{parts}

\question\label{Spinner} Suppose the sample space of an experiment
consists of three independent
outcomes $\omega_1,\omega_2,\omega_3$ and that
$P\left(\omega_1\right)=\frac{2}{7}$, $P\left(\omega_2\right)=\frac{1}{7}$,
and $P\left(\omega_3\right)=\frac{4}{7}$. Consider the events
$E=\left\{\omega_1,\omega_2\right\}$ and $F=\left\{\omega_2\right\}$.
Calculate $P\left(E\mid\text{not $F$}\right)$.
\begin{solution}
\[\frac{P\left(E\cap\text{not $F$}\right)}
{P\left(\text{not $F$}\right)}
=\frac{P\left\{\omega_1\right\}}{P\left\{\omega_1,\omega_3\right\}}
=\frac{2/7}{2/7+4/7}=\frac{1}{3}\]
\end{solution}

\begin{multicols}{2}
\question A spinner for a game is shown at the right.
Calculate the probability that the spinner will stop
on a primary color given that it does not stop on an ISU color.
\begin{center}\includegraphics[scale=.6]{ReviewSpinner}\end{center}
\end{multicols}
\begin{solution} This is the same problem as Exercise~\ref{Spinner}
with $E$ the event that the spinner lands on a primary
color and $F$ the event that the spinner lands on an ISU color.
\end{solution}

\question You roll two regular six-sided dice. Calculate the probability
that the sum of the numbers showing is greater than 9 given that
the first die shows 5.
\begin{solution} The sum will be greater than 9 if the second die
shows 5 or 6.
This will occur with probability $\frac{2}{6}=\frac{1}{3}$.
\end{solution}

\question The National Vital Statistics Report estimates
that a US resident will live at least 80~years with probability~$.091$
and at least 90~years with probability~$0.047$.
What is the probability that a man lives at least 90~years
given that he is now 80~years old?
\begin{solution}
Let $E$ be the event that the resident lives at least 80~years
and $F$ the event that the resident lives at least 90~years.
Note that $E\cap F=F$. Thus
\[P\left(F\mid E\right)
=\frac{P\left(F\cap E\right)}{P\left(E\right)}
=\frac{P\left(F\right)}{P\left(E\right)}
=\frac{0.047}{0.091}\approx 0.516.\]
\end{solution}

\uplevel{\bf Permutations.}
\question In a swimming race involving eight swimmers, only the three
with the fastest times are observed,
namely, first, second, third places.
How many possible outcomes of the race are there?
\begin{solution} $8\cdot 7\cdot 6=336$ \end{solution}
\question A {\em byte} is a sequence of eight {\em bits}, each
of which representing either zero or one. How many bytes are possible?
\begin{solution} $2^8=256$\end{solution}

\question I have five cars, but only a two-stall garage, so I
park three of the cars on the street. In how many
ways can I park two of the cars in the garage?
\question Five children arrive at my house on Halloween,
but I only have three pieces of fruit remaining to distribute, namely
a banana, a pear, and an apple. In how many ways can I distribute
fruit to the children, leaving two without fruit?
\begin{solution} $5\cdot 4\cdot 3=60$\end{solution}

\uplevel{\bf Combinations, binomial coefficients, number of subsets.}
\question How many subsets of a set of size ten are there?
\begin{solution} $2^{10}=1024$\end{solution}
\uplevel{\bf Definitions of all the poker hands, and the calculation of the probabilities of receiving them.}
\uplevel{\bf Problems involving improving poker hands by exchanging cards.}

\uplevel{\bf Binomial probability.}
\question If a baseball player has a batting average
of $0.350$ what is the probability that the player
will hit the ball the following number of times in his next four
times at bat?
\begin{parts}
\part Exactly twice
\part At least twice
\end{parts}

\question If $60\%$ of the electorate supports the mayor
what is the probability that in a random sample
of 10~voters, fewer than half support her?

\uplevel{\bf Blackjack.}
\end{questions}
\end{document}
