\documentclass[answers,12pt]{exam}
\usepackage{mdwlist}
\usepackage[euler-digits]{eulervm}
\usepackage{charter,amsmath,amssymb,breakurl}
\usepackage[letterpaper,margin=1in]{geometry}
\usepackage{multicol}
\author{}\date{}
\title{Review problems}
\begin{document}
\maketitle
\pagestyle{empty}
\begin{questions}
\uplevel{Basic properties of probability}
\uplevel{Empirical probability}
\uplevel{Probabilities and sample spaces of simple events (coin tosses, dice rolls, drawing particular combinations of cards from deck, etc)}
\uplevel{Odds in favor of and against an event}
\uplevel{Calculation of expection}

\uplevel{Fair ticket price}
\question (From exam 1)
To raise money the marching band sells 500~raffle tickets
for $\$3$ each. They will randomly select a ticket holder to receive
a prize of $\$200$.
\begin{parts}
\part Calculate the fair ticket price.
\part Calculate the expected proceeds for the band.
\end{parts}
\begin{solution}
The fair price $P$ is the price for which the expectation
for the ticket holder is $0$. This can be found by solving
$\left(200-P\right)\frac{1}{500}-P\left(\frac{499}{500}\right)=0$
for $P$, which gives $P=\frac{2}{5}$, or 40~cents.
Another way to calculate the fair price
is to add the actual ticket price $\$3$
to the actual expectation $-\$2.60$, which also gives $\$0.40$.
The proceeds for the band are $3\left(\$500\right)-\$200=\$1300$
\end{solution}

\uplevel{Tree diagrams}
\question You need to hire a database administrator
(DBM) and two programmers. You would like to offer the DBM
position to either Alice, Barrett, or Carl and the programming
positions to two of Quinn, Riley, Steven, or Tamara.
However Alice is married to Quinn and the two can only accept
jobs if they are {\em both} offered positions, since they will have
to move if they accept new jobs.
\begin{parts}
\part In how many ways can the jobs be assigned?
\part In how many ways is Alice offered a job?
\part What is the probability that Riley will be offered a job?
\end{parts}
\uplevel{Counting principle}
\uplevel{Inclusion-exclusion formula (including the form in the case of mutually exclusive events)}
\uplevel{Venn diagrams}
\uplevel{Independence}
\question You draw a card from a shuffled deck. You don't replace
the card. Then your opponent draws a card. Are the events that
you drew a king and that she drew a king independent?
\question You draw a card from a shuffled deck. You don't replace
the card. Then your opponent draws a card. Are the events that
you drew a king and that she drew a heart independent?

\uplevel{Craps (including the rules and the calculation
of the probabilities of the various outcomes)}
\question Do the pass bettors or the don't pass bettors have
a greater probability of winning, or are the probabilities the same?
\question Suppose you are the shooter and you roll four on your opening
roll. What is the probability you will win the game?

\uplevel{Union, intersection, size}
\question If $E=\left\{\alpha,\gamma,\zeta\right\}$,
$F=\left\{\alpha,\beta,\delta,\xi\right\}$, and
$G=\left\{\beta,\zeta,\xi\right\}$, then calculate the following.
\begin{parts}
\part $E\cap F$
\part $E\cup G$
\part $\left|E\cup\left(F\cap G\right)\right|$
\end{parts}

\uplevel{Conditional probability (including the definition)}
\question Your opponent draws a card from a shuffled deck. She doesn't
replace the card. Then you draw a card. Calculate the following.
\begin{parts}
\part The probability that you draw a queen given that she drew a queen.
\begin{solution} $\frac{3}{51}$ \end{solution}
\part The probability that she drew a queen and you drew a queen.
\begin{solution} $\frac{4}{52}\cdot\frac{3}{51}=\frac{1}{221}$ \end{solution}
\part The probability that you draw a heart given that she drew a queen.
\begin{solution} This is impossible to calculate without knowing
whether she drew the card $Q\heartsuit$. \end{solution}
\end{parts}

\uplevel{Permutations (including partial permutations and anagrams)}
\question In a swimming race involving eight swimmers, only the three
with the fastest times are observed, namely, first, second, third places.
How many possible outcomes of the race are there?
\begin{solution} $8\cdot 7\cdot 6=336$ \end{solution}

\uplevel{Combinations, binomial coefficients, the number of subsets of a set}
\question How many subsets of a set of size ten are there?
\begin{solution} $2^{10}=1024$\end{solution}
\question 
\uplevel{Definitions of all the poker hands, and the calculation of the probabilities of receiving them}
\uplevel{Problems involving improving poker hands by exchanging cards}
\uplevel{Binomial probability}
\uplevel{Blackjack, including the rules and simple problems}
\end{questions}
\end{document}

\uplevel{Craps (including the rules and the calculation of the probabilities of the various outcomes)}
\question Do the pass bettors or the don't pass bettors have
a greater probability of winning, or are the probabilities the same?
\question Suppose you are the shooter and you roll four on your opening
roll. What is the probability you will win the game?
\uplevel{Union, intersection, size}

\uplevel{Conditional probability (including the definition)}
\uplevel{Permutations (including anagrams)}
\uplevel{Combinations, binomial coefficients, the number of subsets of a set}
\uplevel{Definitions of all the poker hands, and the calculation of the probabilities of receiving them}
\uplevel{Problems involving improving poker hands by exchanging cards}
\uplevel{Binomial probability}
\uplevel{Blackjack, including the rules and simple problems}
\end{questions}
\end{document}
