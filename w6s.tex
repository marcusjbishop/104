\documentclass[12pt]{article}
\everymath{\displaystyle}
\usepackage{mdwlist}
\usepackage[euler-digits]{eulervm}
\usepackage{charter,amsmath,amssymb,breakurl}
\usepackage[letterpaper,margin=1in]{geometry}
\usepackage{multicol}
\author{}\date{}
\title{Worksheet 6 solutions}\author{}
\begin{document}
\maketitle
\pagestyle{empty}
The following questions involve the game of blackjack,
with the dealer dealing from several decks of cards.
You may therefore assume that $\frac{1}{52}$ is the probability
of being dealt any particular card, irrespective of any cards
known to have already been dealt. 
\begin{enumerate}

\item{\em You received the cards $9\clubsuit,6\diamondsuit$
initially. What is the probability you will bust if
you request another card?}\\
{\bf Solution.} You will bust if you receive a card of rank
 $7,8,9,10,J,Q$ or $K$. Since there are seven such cards,
your probability is $\frac{7}{13}$ of busting if you take another card.

\item{\em You received the cards $K\heartsuit,8\diamondsuit$
initially. What is the probability you will bust if
you request another card?}\\
{\bf Solution.} You will bust if you receive a card of rank
 $4,5,6,7,8,9,10,J,Q$ or $K$. Since there are ten such cards,
your probability is $\frac{10}{13}$ of busting if you take another card.

\item{\em Suppose your cards total $18$ and you decide to stand.
What is the probability you will win the game?}\\
{\bf Solution.} You will win if the dealer busts or has a total of $17$.
Consulting the table on slide~12 of the blackjack slides, these
two possibilities happen with probability $0.28$ and $0.15$ respectively.
Thus you win with probability $0.28+0.15=0.43$.

\item{\em Suppose your cards total $20$ and you decide to stand.
What is the probability you will win the game?}\\
{\bf Solution.} You will win if the dealer busts or has a total of $17,18$
or $19$.These
possibilities happen with probability $0.28,0.15,0.15,0.14$ respectively.
Thus you win with probability $0.72$.

\item{\em A {\em royal hand} consists of a king and a queen
of the same suit. Compute the probability of being
dealt a royal hand in the first two cards.}\\
{\bf Solution.} Your probability of receiving a king is $\frac{1}{13}$.
Your probability of receiving a queen of the same suit as the king
is $\frac{1}{52}$. Since these are independent events,
the probability of both happening is
$\frac{1}{13}\cdot\frac{1}{52}=\frac{1}{676}$.
Bear in mind that the solution to this problem would be very
different if the cards were coming from only one deck. However,
as explained in the instructions above, we assume that the cards
come from a mixture of several decks.

\item{\em Calculate the probability of receiving two
cards adding up to exactly $20$.}\\
{\bf Solution.} This can happen in exactly two ways. Namely,
two cards of value ten, or one ace and one $9$.
Since there are sixteen cards of value ten, namely,
the tens, jacks, queens, and kings, the probability of receiving
a card of value ten is $\frac{16}{52}=\frac{4}{13}$.
So the probability of receiving two such cards is
$\left(\frac{4}{13}\right)^2=\frac{16}{169}$.
The probability of receiving an ace is $\frac{1}{13}$.
This is also the probability of receiving a nine.
Thus $\left(\frac{1}{13}\right)^2=\frac{1}{169}$.
Thus $\frac{17}{169}$ is the probability of receiving two
cards adding up to $20$.

\end{enumerate}
\end{document}
