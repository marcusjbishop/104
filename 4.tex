\documentclass[handout]{beamer}
\usepackage{multicol}
\everymath{\displaystyle}
\mode<presentation>
{\usetheme{Warsaw}\setbeamercovered{dynamic}}
\usecolortheme{crane}
\usepackage{beamerfoils}
\pgfdeclareimage[height=1in]{university-logo}{ISULogo}
\logo{\pgfuseimage{university-logo}}
\setbeamertemplate{navigation symbols}{}
\title[\S4]{Section 4\\Expectation}
\author{Dr Marcus Bishop}
\subject{Math 104}
\beamerdefaultoverlayspecification{<+->}
\theoremstyle{definition}
\newtheorem{remark}{Remark}
\newtheorem{impact}{Impact}
\newtheorem{notation}{Notation}
\newtheorem{principle}{Principle}
\usepackage{arev}
\begin{document}
\begin{frame}\titlepage\end{frame}
\LogoOff

\begin{frame}{Numerical outcomes}
Section deals with experiments
that have \alert{numerical outcomes}
\begin{example}
Following experiments have numerical outcomes
\begin{itemize}
\item Number of minutes randomly selected
customer waits in line at Hy-Vee
\item Number of olives in jar
\item Number of minutes randomly selected
student takes to complete quiz
\item Lifetime in hours of light bulb
\end{itemize}
\end{example}
\end{frame}

\begin{frame}
\begin{example}
Following experiments \alert{don't} have numerical outcomes
\begin{itemize}
\item Flipping coin
\item Color of shirt randomly selected student wearing
\item Day of week my package from Amazon arrives
\end{itemize}
\end{example}
\begin{remark}
\begin{itemize}
\item Outcomes of rolling die might or might not be numerical
\item Outcomes $1,2,3,4,5,6$  often have no significance as numbers
\item Can often be replaced with non-numerical outcomes,
such as colors, letters, etc.
\item In contrast, in Monopoly outcome of rolling dice
determines \alert{number} of spaces to advance
\end{itemize}
\end{remark}
\end{frame}

\begin{frame}{Expectation}
\begin{definition}
\begin{itemize}
\item Suppose experiment has outcomes $E_1,E_2,\ldots$
\item $P_1=P\left(E_1\right)$
\item $P_2=P\left(E_2\right),\ldots$
\item Then
\[P_1E_1+P_2E_2+\cdots\]
the \alert{expectation} of experiment
\end{itemize}
\end{definition}
\begin{remark}
\begin{itemize}
\item Expectation gives
\alert{expected value} of experiment \alert{in long run}
\item Doesn't give exact value we expect
next time experiment repeated
\item Expectation used in same way averages used
\end{itemize}
\end{remark}
\end{frame}

\begin{frame}{Time to take quiz}
\begin{itemize}
\item Time it took calculus students to complete quiz
shown in table:
\[\begin{array}{r|ccc}
\text{Time in minutes}&5&8&40\\\hline
\text{Number of students}&25&13&2
\end{array}\]
\item So average time
\[\frac{5\left(25\right)+8\left(13\right)+40\left(2\right)}{40}
\only<+->{=\frac{309}{40}}
\only<+->{\approx 7.73}\]
\item But observe that
\[\frac{5\left(25\right)+8\left(13\right)+40\left(2\right)}{40}
=5\left(\frac{25}{40}\right)+8\left(\frac{13}{40}\right)
+40\left(\frac{2}{40}\right)\]
\item So average and expectation coincide in this example
\end{itemize}
\end{frame}

\begin{frame}{Multiple choice strategy (Example 2)}
\begin{itemize}
\item Suppose multiple choice question has five answer choices
\item $2$~points for correct response
\item $-1/2$~point for incorrect response
\item $0$~points for not answering
\item Should you guess if you don't know answer?
\item Outcomes:
\begin{itemize}
\item $2$ occurs with probability $1/5$ when correct answer chosen
\item $-1/2$ occurs with probability $4/5$ when incorrect answer chosen
\end{itemize}
\item So $2\cdot\frac{1}{5}+\left(-\frac{1}{2}\right)\frac{4}{5}
\only<+->{=\frac{2}{5}-\frac{4}{10}}
\only<+->{=\frac{4}{10}-\frac{4}{10}}
\only<+->{\alert{=0}}$
\only<+->{the expectation}
\item Conclusion: guessing doesn't hurt \alert{in this case}
\end{itemize}
\end{frame}

\begin{frame}
\begin{itemize}
\item Suppose multiple choice question has five answer choices
\item \alert{$1$~point} for correct response
\item $-1/2$~point for incorrect response
\item $0$~points for not answering
\item Should you guess if you don't know answer?
\item Outcomes:
\begin{itemize}
\item $1$ occurs with probability $1/5$ when correct answer chosen
\item $-1/2$ occurs with probability $4/5$ when incorrect answer chosen
\end{itemize}
\item So $1\cdot\frac{1}{5}+\left(-\frac{1}{2}\right)\frac{4}{5}
\only<+->{=\frac{1}{5}-\frac{4}{10}}
\only<+->{=\frac{1}{5}-\frac{2}{5}}
\only<+->{=\alert{-\frac{1}{5}}}$
\only<+->{the expectation}
\item Conclusion: guessing a bad idea!
\end{itemize}
\end{frame}

\begin{frame}{Card game (Exercise 14)}
\begin{itemize}
\item Marcus and Dave play following game:
\item Marcus randomly selects card from standard deck
\item If $\clubsuit$ selected Dave gives Marcus $\$4$
\item Otherwise Marcus gives Dave $\$2$
\item Marcus's expected proceeds:
\[4\left(\frac{1}{4}\right)+\left(-2\right)\left(\frac{3}{4}\right)
\only<+->{\alert{=-\frac{1}{2}}}\]
\item Dave's expected proceeds:
\[\left(-4\right)\left(\frac{1}{4}\right)
+2\left(\frac{3}{4}\right)
\only<+->{\alert{=\frac{1}{2}}}\]
\end{itemize}
\end{frame}

\begin{frame}
\begin{itemize}
\item To make \alert{fair}, change Dave's payment to
\alert{$\$3$}
and Marcus's payment to \alert{$\$1$}
\item Marcus's new expected proceeds:
\[3\left(\frac{1}{4}\right)
+\left(-1\right)\left(\frac{3}{4}\right)\alert{=0}\]
\item Indeed, $3:1$ the odds against Marcus winning
\item So on average, Marcus loses three times for every
time he wins
\item So if he pays $\$1$ when loses
and gains $\$3$ when he wins, he breaks even
\item Same principle used in casino games:
\begin{principle}
If $X:1$ the
odds against player, prize \alert{should} be $X$ times
amount player paid
\end{principle}
\item However, to give casino slight edge, prize always
slightly less than $X$ times amount bet
\end{itemize}
\end{frame}

\begin{frame}{Door Prize}
\begin{itemize}
\item At charity event can purchase one of $100$ tickets for~$\$2$
\item Prize is $\$50$
\item Calculate expected proceeds of participant who buys one ticket
\item Outcomes:
\begin{itemize}
\item $48=50-2$ with probability $1/100$ if wins
\item $-2$ with probability $99/100$ if loses
\end{itemize} 
\item $48\left(\frac{1}{100}\right)+\left(-2\right)
\left(\frac{99}{100}\right)
\only<+->{=-\frac{3}{2}}$
\item So $\$-1.50$ the expected proceeds 
\item If participant buys \alert{two} tickets:
\item $46\left(\frac{2}{100}\right)+\left(-4\right)
\left(\frac{98}{100}\right)
\only<+->{=-3}$
\end{itemize}
\end{frame}

\begin{frame}{Fair price}
\begin{itemize}
\item Ignoring fact that event a charity, what would be \alert{fair}
ticket price?
\begin{definition}
An experiment called \alert{fair} if zero its expectation
\end{definition}
\item Can figure out fair price $F$ by setting expectation
equal to zero:
\begin{align*}
0&=\left(50-F\right)\left(\frac{1}{100}\right)
-F\left(\frac{99}{100}\right)\\
\only<+->{&=50-F-99F\\}
\only<+->{\iff 100F&=50\\}
\only<+->{\iff F&=\frac{50}{100}=0.50}
\end{align*}
\end{itemize}
\end{frame}

\begin{frame}
\begin{itemize}
\item So $\$0.50$ the fair price
\item Check by calculating expectation:
\[49.50\left(0.01\right)-.50\left(.99\right)=0\]
\item Expectation from point of view of charity:
\[-50+100\left(0.50\right)=0\]
\item Note that only one outcome:
\begin{itemize}
\item Someone wins with probability $1$
\item Charity pays this person $\$50$ and collects $100\left(0.50\right)
=\$50$ in ticket money
\end{itemize}
\item So $\$0.50$ tickets also \alert{fair} to charity
\item Generally \alert{fair} not a good business model!
\end{itemize}
\end{frame}

\begin{frame}{Calculating fair price}
\begin{itemize}
\item But there's an easier way to calculate fair price
\item First calculate expected proceeds of (possibly unfair) game
\item If expected proceeds \alert{negative} then
\[\text{fair price}=\text{expected proceeds}+\text{cost to play}\]
\item Formula obviates having to solve for fair price
\begin{example}
\begin{itemize}
\item Original scenario: 100~tickets, $\$2$ each, $\$50$ prize
\item $-\$1.50$ the expected proceeds
\item So $2+\left(-1.50\right)=.50$ the fair price
\item Agrees with algebra calculation above
\end{itemize}
\end{example}
\end{itemize}
\end{frame}

\begin{frame}{Yet another way}
\begin{itemize}
\item (Slide inspired by conversation with T. Jager)
\item Jager's idea to view situation from point of view
of charity rather than ticket holder
\item Observe that if expectation zero for ticket holder,
then expectation also zero for charity 
\item Thus revenue from tickets must be $\$50$
\item So ticket price must be $\frac{50}{100}=0.50$
\item Agrees with both calculations above
\end{itemize}
\begin{remark}
\begin{itemize}
\item Reason fair price important:
\item Fair price the \alert{minimum} charity should
charge for tickets, in order to profit
\end{itemize}
\end{remark}
\end{frame}

\begin{frame}{Another raffle example}
\begin{itemize}
\item Marching band sells $500$ raffle tickets for $\$3$
\item One ticket holder randomly selected to receive $\$150$
\item Find fair ticket price
\item Can calculate as usual:
\[E=147\left(\frac{1}{500}\right)-3\left(\frac{499}{500}\right)=-2.70\]
so $3-2.70=.30$ the fair price
\item Observe that if expectation zero for ticket holder,
then expectation also zero for band
\item Thus revenue from tickets must be $\$150$
\item So fair ticket price $\frac{150}{500}=0.30$
\item Agrees with calculation above
\end{itemize}
\end{frame}

\begin{frame}{Term life insurance (Exercise 55)}
\begin{itemize}
\item Insurance company offers \alert{ten year life
insurance policy}:
\begin{itemize}
\item Customer pays company $\$1500$ (the \alert{premium})
\item If customer dies during the ten years
(the \alert{term}) company pays customer's estate
$\$40,000$ but keeps premium
\item If customer survives ten years, company pays customer
nothing and keeps premium
\end{itemize}
\item Suppose $0.97$ the
probability that 30~year old male lives to age $40$ (or longer)
\item Morbid as it sounds, calculating such probabilities
the goal of \alert{actuarial science}
\item Calculate expected proceeds of company
\end{itemize}
\end{frame}

\begin{frame}
\begin{itemize}
\item $1500\left(.97\right)+\left(1500-40000\right)\left(.03\right)
\only<+->{=300}$
\item Instead of $\$1500$ calculate \alert{minimum}
premium $P$ such that company profits in long run
\item[]
\begin{align*}
P\left(.97\right)+\left(P-40000\right)\left(.03\right)&>0\\
\only<+->{
.97P+.03P-40000\left(.03\right)&>0\\}
\only<+->{
P&>40000\left(.03\right)}
\only<+->{=1200}
\end{align*}
\item So for any $P>1200$ company expects to profit
\item Indeed, $300$ the expectation when $P=1500$
\end{itemize}
\end{frame}

\begin{frame}{Lawsuit (Exercise 50)}
\begin{itemize}
\item Client considering bringing lawsuit against a chemical company
\item Her lawyer estimates $70\%$ chance she wins $\$60,000$
\item $10\%$ chance she wins nothing
\item $20\%$ chance she loses case, pays $\$30,000$ in legal fees
\item Calculate expected gain or loss if client proceeds with case
\item $60000\left(0.7\right)
+0\left(0.1\right)-30000\left(0.2\right)
\only<+->{\alert{=36000}}$
\end{itemize}
\begin{remark}
\begin{itemize}
\item Experiment has \alert{three} outcomes, unlike previous examples
\item Expectation gives net proceeds \alert{in long run}
\item However, in lawsuit example, experiment only run once,
so expectation not particularly helpful here
\end{itemize}
\end{remark}
\end{frame}

\begin{frame}{More complicated card choices}
\begin{itemize}
\item Probability of \alert{either} rank $2$ or 
suit $\alert{\vardiamond}$?
\item Four cards have rank $2$, namely
$2\clubsuit, 2\spadesuit, 2\alert{\vardiamond}, 2\alert{\varheart}$
\item Thirteen cards have suit $\alert{\vardiamond}$, namely
$\text{A}\alert{\vardiamond},2\alert{\vardiamond},\ldots,
\text{K}\alert{\vardiamond}$
\item However, $2\alert{\vardiamond}$ appears in \alert{both} lists
\item So \alert{sixteen} cards satisfy \alert{either} $2$ or $\alert{\vardiamond}$?
\item Thus $\frac{16}{52}=\frac{4}{13}$ the probability of
\alert{either} $2$ or $\alert{\vardiamond}$
\end{itemize}
\end{frame}

\end{document}
