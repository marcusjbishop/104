\documentclass[handout]{beamer}
\usepackage{multicol}
\everymath{\displaystyle}
\mode<presentation>
{\usetheme{Warsaw}\setbeamercovered{dynamic}}
\usecolortheme{crane}
\usepackage{beamerfoils}
\pgfdeclareimage[height=1in]{university-logo}{ISULogo}
\logo{\pgfuseimage{university-logo}}
\setbeamertemplate{navigation symbols}{}
\title[\S4]{Section 4\\Expectation}
\author{Dr Marcus Bishop}
\subject{Math 104}
\beamerdefaultoverlayspecification{<+->}
\theoremstyle{definition}
\newtheorem{remark}{Remark}
\newtheorem{impact}{Impact}
\newtheorem{notation}{Notation}
\usepackage{arev}
\begin{document}
\begin{frame}\titlepage\end{frame}
\LogoOff

\begin{frame}{Numerical outcomes}
Section deals with experiments
that have \alert{numerical outcomes}
\begin{example}
Following experiments have numerical outcomes
\begin{itemize}
\item Number of minutes a randomly selected
customer waits in line at Hy-Vee
\item Number of olives in jar
\item Number of minutes it takes randomly selected
student to complete quiz
\item Lifetime in hours of light bulb
\end{itemize}
\end{example}
\end{frame}

\begin{frame}
\begin{example}
Following experiments \alert{don't} have numerical outcomes
\begin{itemize}
\item Flipping coin
\item Color of shirt randomly selected student wearing today
\item Day of week my package from Amazon arrives
\end{itemize}
\end{example}
\begin{remark}
\begin{itemize}
\item Outcomes of rolling die might or might not be numerical
\item Outcomes $1,2,3,4,5,6$  often have no significance as numbers
\item Can often be replaced with non-numerical outcomes,
such as colors, letters, etc.
\item In contrast, in Monopoly outcome of rolling dice
determines \alert{number} of spaces to advance
\end{itemize}
\end{remark}
\end{frame}

\begin{frame}{Expectation}
\begin{definition}
\begin{itemize}
\item Suppose experiment has outcomes $E_1,E_2,\ldots$
\item $P_1=P\left(E_1\right)$
\item $P_2=P\left(E_2\right),\ldots$
\item Then
\[P_1E_1+P_2E_2+\cdots\]
the \alert{expectation} of experiment
\end{itemize}
\end{definition}
\begin{remark}
\begin{itemize}
\item Expectation gives
\alert{expected value} of experiment \alert{in long run}
\item Does not give the exact value we expect
next time experiment repeated
\item Expectation used in same way averages used
\end{itemize}
\end{remark}
\end{frame}

\begin{frame}{Time to take quiz}
\begin{itemize}
\item Time it took calculus students to complete quiz
shown in table:
\[\begin{array}{r|ccc}
\text{Time in minutes}&5&8&40\\\hline
\text{Number of students}&25&13&2
\end{array}\]
\item So average time
\[\frac{5\left(25\right)+8\left(13\right)+40\left(2\right)}{40}
\only<+->{=\frac{309}{40}}
\only<+->{\approx 7.73}\]
\item But observe that
\[\frac{5\left(25\right)+8\left(13\right)+40\left(2\right)}{40}
=5\left(\frac{25}{40}\right)+8\left(\frac{13}{40}\right)
+40\left(\frac{2}{40}\right)\]
\item So average and expectation coincide in this example
\end{itemize}
\end{frame}

\begin{frame}{Multiple choice strategy (Example 2)}
\begin{itemize}
\item Suppose multiple choice question has five answer choices
\item $2$~points for a correct response
\item $-1/2$~point for incorrect response
\item $0$~points for not answering
\item Should you guess if you don't know answer?
\item Outcomes:
\begin{itemize}
\item $2$ occurs with probability $1/5$ when correct answer chosen
\item $-1/2$ occurs with probability $4/5$ when incorrect answer chosen
\end{itemize}
\item So $2\cdot\frac{1}{5}+\left(-\frac{1}{2}\right)\frac{4}{5}
\only<+->{=\frac{2}{5}-\frac{4}{10}}
\only<+->{=\frac{4}{10}-\frac{4}{10}}
\only<+->{=0}$ the expectation
\item Conclusion: guessing doesn't hurt \alert{in this case}
\end{itemize}
\end{frame}

\begin{frame}
\begin{itemize}
\item Suppose multiple choice question has five answer choices
\item \alert{$1$~point} for a correct response
\item $-1/2$~point for incorrect response
\item $0$~points for not answering
\item Should you guess if you don't know answer?
\item Outcomes:
\begin{itemize}
\item $1$ occurs with probability $1/5$ when correct answer chosen
\item $-1/2$ occurs with probability $4/5$ when incorrect answer chosen
\end{itemize}
\item So $1\cdot\frac{1}{5}+\left(-\frac{1}{2}\right)\frac{4}{5}
\only<+->{=\frac{1}{5}-\frac{4}{10}}
\only<+->{=\frac{1}{5}-\frac{2}{5}}
\only<+->{=-\frac{1}{5}}$ the expectation
\item Conclusion: guessing a bad idea!
\end{itemize}
\end{frame}

\begin{frame}{Door Prize}
\begin{itemize}
\item At charity event can purchase one of $100$ tickets for~$\$2$
\item Prize is $\$50$
\item Calculate expected winnings of participant who buys one ticket
\item Outcomes:
\begin{itemize}
\item $48=50-2$ with probability $1/100$ if wins
\item $-2$ with probability $99/100$ if loses
\end{itemize} 
\item $48\left(\frac{1}{100}\right)+\left(-2\right)
\left(\frac{99}{100}\right)
\only<+->{=-\frac{3}{2}}$
\item So $\$-1.50$ the expected winnings
\item If participant buys \alert{two} tickets:
\item $46\left(\frac{2}{100}\right)+\left(-4\right)
\left(\frac{98}{100}\right)
\only<+->{=-3}$
\end{itemize}
\end{frame}

%XXX Insert fairness slides here
\begin{frame}{Calculating fair price}
\begin{itemize}
\item But there's an easier way to calculate fair price
\item First calculate expected proceeds of (possibly unfair) game
\item If expected proceeds \alert{negative} then
\[\text{fair price}=\text{expectated proceeds}+\text{cost to play}\]
\item Formula obviates having to solve for fair price
\begin{example}
\begin{itemize}
\item Original senario: 100~tickets, $\$2$ each, $\$50$ prizze
\item $-\$1.50$ the expected proceeds
\item So $2+\left(-1.50\right)=-.50$ the fair price
\item Agrees with algebra calculation above
\end{itemize}
\end{example}
\end{itemize}
\end{frame}

\end{document}
