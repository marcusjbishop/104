\documentclass[11pt]{article}
\usepackage[euler-digits]{eulervm}
\usepackage[colorlinks=true, linkcolor=blue, breaklinks=true]{hyperref} 
\def\sectionautorefname~#1\null{\S#1\null}
\usepackage{charter,amsmath,amssymb,breakurl,booktabs}
\usepackage[letterpaper,margin=.9in]{geometry}
\title{Syllabus for Math~104}
\author{Dr Marcus Bishop, Iowa State University}
\begin{document}
\maketitle

\section{Time, place}\label{Time} Class meets at 12:10--13:00
in Coover~2245 on Mondays, Wednesdays, and Fridays.
The final exam is at 14:15--16:15 on Wednesday 17~December.

\section{Requirements, prerequisites, materials}
To take this course you should have taken two
years of high school algebra and one year of high school geometry.
In addition, you should
have received a satisfactory score on your mathematics placement exam.
For the course you will need a copy of the textbook
and a scientific calculator.
I use a \href{http://en.wikipedia.org/wiki/TI-83}{TI-83} or a
\href{http://en.wikipedia.org/wiki/TI-84}{TI-84}, although
the graphing and algebraic facilities of these devices are
not needed for Math~104.

\section{Instructor information} You may visit me during my office 
hours at 14:00--15:00 on Mondays, Tuesdays, Thursdays, and Fridays
in Carver~418. If my office hours are inconvenient for you
please contact me at 
\href{mailto:mbishop@iastate.edu}{\tt mbishop@iastate.edu} or at 
515-294-0027 to arrange an alternate time.
You can also visit the Math Help Room (see \autoref{MathCenter}).

\section{Course goals, content}
In this course we will learn the rudiments of combinatorics,
including combinations, permutations, and the binomial theorem,
as well as various counting techniques.
Our primary motivation for learning combinatorics is that it
plays an important role in the theory of probability, the main
topic of the course. We will also explore some
applications of probability in gambling. Indeed, probability
arose in the seventeenth century in response to the need
to understand the likelihood of winning at games of chance.
Students should not presume that the course is intended to
promote gambling. On the contrary, we expect that most
students will abstain from gambling after learning the mathematics
behind a number of games.

\section{Calculation of grades}\label{Assessment}
Grades will be assigned by a spreadsheet formula, which will count
the three exams for 15\% each,
the written assignments for 20\% (see \autoref{Written}),
the quizzes for 15\%, and the final exam for 20\%.
You will receive an A, B, C, D, F if your total lies in the range
90--100\%, 80--89\%, 70--79\%, 60--69\%, 0--60\% respectively.
All the scores mentioned above will be available to you in
\href{https://bb.its.iastate.edu}{Blackboard}. Using your scores
and the formula above, you should be able to calculate your own final grade,
and even estimate your final grade midway through the semester.
It is therefore unnecessary for you to ever ask the instructor
what your grade in the course is, which we kindly ask you to refrain from
doing.

We emphasize that the grades will be calculated by a computer program in the manner described
above, not ``curved''. However, in the unlikely event that an exam or quiz question is determined
to be unfair or overly difficult, it will be dropped from the total score of
every student.

\section{Written assignments}\label{Written}
There is a written assignment due {\em every 
class period}, including class periods
in which quizzes or exams occur.
The assignments will be posted on \href{https://bb.its.iastate.edu}{Blackboard}.
They will be collected at the beginning
of class on randomly selected days, approximately
once per week. Collected assignments
will be graded and returned shortly thereafter.
The course is designed in such a way that if you complete each assignment and 
correct the mistakes your make on them,
then you will be very well prepared for all the quizzes and exams.
While we encourage you to {\em discuss} homework problems with other
students, we advise you to work alone most of the time,
as working alone most closely reflects the context of an exam.
In order to accommodate unforeseen events that would cause you
to pass up the opportunity submit assignments on time,
we will exclude from your final grade
the three written assignments with the lowest scores among
all your written assignments.

\section{Expectations, suggestions} Naturally we expect you to attend 
class meetings, complete and submit assignments on 
time, prepare for quizzes and exams, and participate in classroom 
activities. We also strongly encourage you to read the textbook
in addition to attending lectures.
In addition to being generally prudent to 
supplement your learning materials, reading the text has a number of 
advantages over attending lectures alone. Namely, the text is beautifully 
typeset and edited by professionals, very clearly written with readers 
of exactly your level in mind, and virtually free of mistakes.

\section{Course schedule}\label{Schedule} While the exact subject
matter to be covered in class shown in the following schedule is
subject to change slightly, the quizzes and exams will be conducted
on {\em exactly} the dates shown.

\begin{tabular}{c|cl|cl|cl}
&\multicolumn{2}{c|}{\bf Monday}
&\multicolumn{2}{c|}{\bf Wednesday}
&\multicolumn{2}{c}{\bf Friday}\\
{\bf Week}&{\bf Date}&{\bf In class}
&{\bf Date}&{\bf In class}&{\bf Date}&{\bf In class}\\\toprule
1&25 Aug&\S1&27 Aug&\S1&29 Aug&\S2\\\midrule
2&1 Sep&Holiday&3 Sep&\S2&5 Sep&\S3, {\bf Quiz 1}\\\midrule
3&8 Sep&\S3&10 Sep&\S4&12 Sep&\S4\\\midrule
4&15 Sep&\S5&17 Sep&Review&19 Sep&{\bf Exam 1}\\\midrule
5&22 Sep&\S6&24 Sep&\S6&26 Sep&\S6\\\midrule
6&29 Sep&\S7&1 Oct&\S7&3 Oct&\S8, {\bf Quiz 2}\\\midrule
7&6 Oct&\S8&8 Oct&Review&10 Oct&{\bf Exam 2}\\\midrule
8&13 Sep&\S9&15 Sep&\S9&17 Sep&\S9\\\midrule
9&20 Sep&\S10&22 Sep&\S10&24 Sep&\S11, {\bf Quiz 3}\\\midrule
10&27 Sep&\S11&29 Sep&Review&31 Sep&{\bf Exam 3}\\\midrule
11&3 Nov&Blackjack&5 Nov&Blackjack&7 Nov&Blackjack\\\midrule
12&10 Nov&Craps&12 Nov&Craps&14 Nov&Craps\\\midrule
13&17 Nov&Poker&19 Nov&Poker&21 Nov&Poker, {\bf Quiz 4}\\\midrule
14&1 Dec&Lottery&3 Dec&Lottery&5 Dec&\\\midrule
15&8 Dec&&10 Dec&Review&12 Dec&Review\\\midrule
16&&&17 Dec&{\bf Final Exam}&&
\end{tabular}

\section{Math Help Room}\label{MathCenter}
We strongly encourage you to visit the Math Help Room
in Carver~385 for additional help.
Open 9:00--16:00 Monday through Friday the Math Help Room
has tutors specifically intended to address Math~104 questions.

\section{Students with disabilities, academic honesty, disruptive behavior}
The Department of Mathematics complies with the 
\href{http://www.ada.gov}{American Disabilities Act} in making reasonable 
accommodations for qualified students with disabilities.  Students with 
special needs should call the 
\href{http://www.dso.iastate.edu/dr}{Office of Student Disability Resources} at
515-294-7220.
We strictly enforce the University's policies on 
\href{http://www.dso.iastate.edu/ja/academic/misconduct}{academic misconduct}
and \href{http://www.dso.iastate.edu/sa/issuesconcerns/disruption}
{disruptive behavior}.
\end{document}
