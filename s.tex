\documentclass[11pt]{article}
\usepackage[euler-digits]{eulervm}
\usepackage[colorlinks=true, linkcolor=blue, breaklinks=true]{hyperref} 
\def\sectionautorefname~#1\null{\S#1\null}
\usepackage{charter,amsmath,amssymb,breakurl,booktabs}
\usepackage[letterpaper,margin=.9in]{geometry}
\title{Syllabus for Math~104}
\author{Dr Marcus Bishop, Iowa State University}
\begin{document}
\maketitle

\section{Time, place, credits}\label{Time} Class meets at 10:00--10:50
in Lagomar~W0142 on Mondays, Wednesdays, and Fridays.
%The final exam is at 14:15--16:15 on Wednesday 17~December.
Math~104 is a three-credit course.

\section{Requirements, prerequisites, materials}
To take this course you should have taken two
years of high school algebra and one year of high school geometry.
In addition, you should
have received a satisfactory score on your mathematics placement exam.
For the course you will need a copy of the textbook
and a scientific calculator.

\section{Instructor and TA information} You may visit me during my office 
hours at 9:00--12:00 on Tuesdays and Thursdays
in Carver~418. If my office hours are inconvenient for you
please email me at 
\href{mailto:mbishop@iastate.edu}{\tt mbishop@iastate.edu} or call me at 
515-294-0027 to arrange an alternate time.

You may also visit
\href{mailto:mws@iastate.edu}{Minwoo Shin},
the graduate assistant for the course,
in Carver~244 on Mondays and Wednesdays at 12:00--14:00.
Minwoo will handle homework collection and grading, so all questions
concerning homework should directed to Minwoo at
\href{mailto:mws@iastate.edu}{\tt mws@iastate.edu}.

Finally, you can visit the Math Help Room (see \autoref{MathCenter})
for additional help.

\section{Course goals, content}
In this course we will learn the rudiments of combinatorics,
including combinations, permutations, and the binomial theorem,
as well as various counting techniques.
Our primary motivation for learning combinatorics is that it
plays an important role in the theory of probability, the main
topic of the course. We will also explore some
applications of probability in gambling. Indeed, probability
arose in the seventeenth century in response to the need
to understand the likelihood of winning at games of chance.
Students should not presume that the course is intended to
promote gambling. On the contrary, we expect that most
students will abstain from gambling after learning the mathematics
behind a number of games.

\section{Course objectives}
\subsection{Basic Rules of Discrete Probability}
\begin{enumerate}
\item Understand the concept of {\em sample space} and be able to construct sample spaces for simple experiments
\item Understand the definition of {\em probability} in the case where all outcomes are equally likely
\item Understand and be able to apply the basic rules for probability,
namely the union, intersection, and complement of an event
\item Understand the concept of {\em conditional probability}
and be able to compute the conditional probability of simple events
\item Understand the significance of tree diagrams
when computing conditional probability
\item Understand the concept of {\em independent events}
\item To have some exposure to real-life cases, such as court decisions,
where the concept of independent events was misapplied
\item To understand the concept of {\em expected value}
and be able to compute the expected value in cases where the
outcomes of an experiment and their probabilities are known
\item To have some idea of how {\em expected value}
is applied in certain real-life situations,
including insurance policies and other forms of risk-taking
\end{enumerate}
\subsection{Basic Principles of Combinatorics}
\begin{enumerate}
\item To understand the meaning of combinations and permutations,
and the difference between them
\item To know the definition of the binomial coefficients
and to be able to compute simple examples of them
\item To use the concept of combinations and permutations
to solve simple counting problems
\end{enumerate}
\subsection{Applications to Games and Gambling}
\begin{enumerate}
\item To understand the basic rules and bets for poker
(video, Texas Hold Em, and other variants)
and to be able to compute the probabilities of various hands
\item To understand the basic rules of the game of Blackjack,
to be able to compute simple probabilities associated with the game,
and to understand the mathematics behind {\em card counting}
\item To understand the basic rules of roulette,
to be able to analyze the various bets using the concept of expected value,
to understand why there is no such thing as a {\em perfect system}
for winning this and other gambling games,
to understand why European poker (with no 00)
is a better bet for the gambler than American poker
\item To understand the rules of some of the various lottery games,
including Powerball,
to be able to compute the probability of winning the jackpot,
and to appreciate why lottery tickets are such poor bets
\item To understand the rules of Backgammon and to be able to use
probability to determine the best moves in certain situations
\item To apply conditional probability to analyze
and understand the game of craps
\item To use probability to analyze other games,
including Farkle, Keno or Chuck-a-Luck
\end{enumerate}

\section{Online exercises}\label{Online}
The online exercises will be delivered through the
\href{http://iastate.mylabsplus.com}{MyLabsPlus} system.
Your subscription to 
\href{http://iastate.mylabsplus.com}{MyLabsPlus}
comes with the purchase of your textbook.
However, if you purchased the book from some other source,
then you can purchase a 
\href{http://iastate.mylabsplus.com}{MyLabsPlus}
subscription at the bookstore or directly at
\href{http://iastate.mylabsplus.com}{MyLabsPlus}.
To access
\href{http://iastate.mylabsplus.com}{MyLabsPlus}
sign in at
\begin{center}
\href{http://iastate.mylabsplus.com}{\tt http://iastate.mylabsplus.com}
\end{center}
with your NetID and password.
You should also run the
\href{https://www.mathxl.com/BrowserCheck/BrowserCheck.aspx?appproductid=3&courseid=2744761&handler_urn=pearson%2fmlp_mml_xl%2fslink%2fx-pearson-mlp_mml_xl&productid=ccng}{Browser Checker}
to ensure that all the exercises will appear correctly.
In addition to providing your online assignments
\href{http://iastate.mylabsplus.com}{MyLabsPlus}
also contains a number of useful resources, not the least of which is an
electronic copy of the entire text, which could obviate
the somewhat strenuous task of transporting the book.

\section{Written assignments}\label{Written}
There is a written assignment due {\em every 
Friday} except 16 January and 1 May
but including class periods
after which quizzes or exams occur.
The assignments will be posted on \href{https://bb.its.iastate.edu}{Blackboard}.
They will be collected at the beginning
of class. An assignment cannot be submitted after that deadline
because we will usually discuss the solutions to the assignment
in class after collecting it.
However, in cases of prearranged absences, assignments may be submitted
early to \href{mailto:mws@iastate.edu}{Minwoo} in Carver~244.
Please do not slide early assignments under office doors or attempt
to submit them to me in in class.

Collected assignments will be graded and returned the following Friday.
It is important to collect your graded homework and confirm that the correct
score for the assignment was entered into Blackboard.
\href{https://bb.its.iastate.edu}{Blackboard}. Please contact
\href{mailto:mws@iastate.edu}{Minwoo} immediately in case your homework scores
were incorrectly entered.

The course is designed in such a way that if you complete each assignment and 
correct the mistakes your make on them,
then you will be very well prepared for all the quizzes and exams.
While we encourage you to {\em discuss} homework problems with other
students, we advise you to work alone most of the time,
as working alone most closely reflects the context of an exam.
In order to accommodate unforeseen events that would cause you
to pass up the opportunity submit assignments on time,
we will exclude from your final grade
the three written assignments with the lowest scores among
all your written assignments.

\section{Calculation of grades}\label{Assessment}
Grades will be assigned by a spreadsheet formula, which will count
the three midterm exams for 15\% each,
the written assignments for 10\% (see \autoref{Written}),
the online assignments for 10\% (see \autoref{Online}),
the quizzes for 15\%, and the final exam for 20\%.
This means that your grand total for the course out of 1000
can be calculated by the formula
\footnote{The denominators of each term are the number of points possible
for the corresponding assessment. Written assignments and quizzes will be marked
out of 10~points. Midterm exams will be marked out of 100~points. The final exam
will be marked out of 40~points.}
\[\frac{450e}{300}+\frac{100w}{10W}
+\frac{100m}{M}+\frac{150q}{50}+\frac{200f}{40}\]
where $e$ is the sum of of your midterm exam scores,
$w$ is the sum of your written homework scores, $W$ is the
number of written assignments (which won't be known until
the end of the course but is likely to be around 10),
$m$ is the sum of your
\href{http://iastate.mylabsplus.com}{MyLabsPlus} scores,
$M$ is the number of 
\href{http://iastate.mylabsplus.com}{MyLabsPlus} points possible
(which won't be known until the end of the course but is likely
to be around 450), $q$ is the sum
of your quiz scores, and $f$ is your final exam score.
You will receive an A, B, C, D, F if your total lies in the range
850--1000, 750--849, 650--749, 550--649, 0--549 respectively.
All the scores mentioned above will be available to you in
\href{https://bb.its.iastate.edu}{Blackboard}. Using your scores
and the formula above, you should be able to calculate your own final grade,
and even estimate your final grade midway through the semester.
It is therefore unnecessary for you to ever ask the instructor
what your grade in the course is, which we kindly ask you to refrain from
doing.

We emphasize that the grades will be calculated by a computer program
in the manner described
above, not ``curved''. However, in the unlikely event that an exam
or quiz question is determined
to be unfair or overly difficult, it will be dropped from the total score of
every student.

\section{Quiz and exam dates}
The quizzes and exams will be conducted in class
on {\em exactly} the dates shown below, all of which are Wednesdays.
\begin{center}\begin{tabular}{lrl}
Quiz 1&21&January\\
Exam 1&4&February\\
Quiz 2&18&February\\
Exam 2&4&March\\
Quiz 3&25&March\\
Exam 3&8&April\\
Quiz 4&22&April
\end{tabular}\end{center}
Please mark all of these events on your calendar,
because by the policy of the mathematics department
it is {\em never} possible to reschedule a quiz or exam
except in cases of death, severe illness, military service,
jury duty, or school-sponsored trip.
Appropriate documentation is required in each of these cases.

\section{Math Help Room}\label{MathCenter}
We strongly encourage you to visit the Math Help Room
in Carver~385 for additional help.
Open 9:00--16:00 Monday through Friday the Math Help Room
has tutors specifically intended to address Math~104 questions.

\section{Expectations, suggestions} Naturally we expect you to attend 
class meetings, complete and submit assignments on 
time, prepare for quizzes and exams, and participate in classroom 
activities. We also strongly encourage you to read the textbook
in addition to attending lectures.
In addition to being generally prudent to 
supplement your learning materials, reading the text has a number of 
advantages over attending lectures alone. Namely, the text is beautifully 
typeset and edited by professionals, very clearly written with readers 
of exactly your level in mind, and virtually free of mistakes.
\section{Students with disabilities, academic honesty, disruptive behavior}
The Department of Mathematics complies with the 
\href{http://www.ada.gov}{American Disabilities Act} in making reasonable 
accommodations for qualified students with disabilities.  Students with 
special needs should call the 
\href{http://www.dso.iastate.edu/dr}{Office of Student Disability Resources} at
515-294-7220.
We strictly enforce the University's policies on 
\href{http://www.dso.iastate.edu/ja/academic/misconduct}{academic misconduct}
and \href{http://www.dso.iastate.edu/sa/issuesconcerns/disruption}
{disruptive behavior}.
\end{document}
