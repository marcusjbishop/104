\documentclass[11pt]{article}
\usepackage[euler-digits]{eulervm}
\usepackage[colorlinks=true, linkcolor=blue, breaklinks=true]{hyperref} 
\def\sectionautorefname~#1\null{\S#1\null}
\usepackage{charter,amsmath,amssymb,breakurl,booktabs}
\usepackage[letterpaper,margin=.9in]{geometry}
\title{Syllabus for Math~104}
\author{Dr Marcus Bishop, Iowa State University}
\begin{document}
\maketitle

\section{Time, place}\label{Time} Class meets at 12:10--13:00
in Coover~2245 on Mondays, Wednesdays, and Fridays.
The final exam is at 14:15--16:15 on Wednesday 17~December.

\section{Requirements, prerequisites, materials}\label{Require}
To take this course you should have taken two
years of high school algebra and one year of high school geometry,
as well as achieved a satisfactory score on your placement exam.
For the course you will want a scientific calculator.
I use a \href{http://en.wikipedia.org/wiki/TI-83}{TI-83} or a
\href{http://en.wikipedia.org/wiki/TI-84}{TI-84}, although
the graphing and algebraic facilities of these devices are
not needed for Math~104.

\section{Instructor information}

\section{Course goals, content}
In this course we will learn the rudiments of combinatorics,
which include combinations, permutations, and the binomial theorem.
Our primary motivation for learning combinatorics is that
plays an important role in the theory of probability, the main
topic of the course. We will also explore some
applications of probability in gambling. Indeed, probability
arose in the seventeenty century in response to the need
to understand the likelyhood of winning at games of chance.

Students should not presume that the course is intended to
promote gambling. On the contrary, we expect that most
students will abstain from gambling after learning the mathematics
behind a number of games.

\section{Calculation of grades}\label{Assessment}
Grades will be assigned by a spreadsheet formula, which will count
the three exams together for 40\%, the written assignments for 25\% (see \autoref{Written}),
the quizes for 15\%, and the final exam for 20\%. You will receive an A, B, C, D, F
if your total calculated by this formula is
90--100\%, 80--89\%, 70--79\%, 60--69\%, or below 60\% repsectively.
All your scores mentioned above will be available to you in
\href{https://bb.its.iastate.edu}{Blackboard}. Using your scores
and the formula above, you should be able to calculate your own final grade,
and even estimate your final grade midway through the semester.
It is therefore unnecessary for you to ever ask the instructor
what your grade in the course is, which we kindly ask you to refrain from
doing.

We emphasize that the grades will be calculated by a computer program in the manner described
above, not ``curved''. However, in the unlikely event that an exam or quiz question is determined
to be unfair or overly difficult, it will be dropped from the total score of
every student.

\section{Written assignments}\label{Written}
There is a written assignment due on {\em every 
Friday} that classes are held, including Fridays of weeks
in which exams occur.
These assignments will be collected at the beginning
of class. They will be graded and returned shortly thereafter.
The first eleven assignments
are listed in \autoref{Schedule}. Assignments 11--15
will be posted at a later stage on \href{https://bb.its.iastate.edu}{Blackboard}.
The written assignments comprise the core of the course. In fact, the course is 
designed in such a way that if you complete each assignment and 
correct the mistakes your make on them,
then you will be very well prepared for all the quizzes and exams.

As with all writing, you should compose the solutions to your written
assignments with the primary goal of explaining something to the
reader as painlessly and effortlessly as possible. Obviously this
is even more relevant when the reader is assessing your understanding
of something, which will be the case with all your writing for this
course. Therefore, we advise you to solve each exercise on scratch
paper, write your solution on scratch paper, and finally, copy your
solution onto the page you intend to submit.

While we encourage you to {\em discuss} homework problems with other
students, we advise you to work alone most of the time,
as working alone most closely reflects the context of an exam.
However, if you choose to complete your written assignments in small
groups containing {\bf no more than four members}, then you should
submit {\bf only one assignment} per group, with the names of all
the groups' members clearly written at the top.  Although this
practice is strongly discouraged, the decision to submit assignments
in groups will have no direct effect on your grade. However,
collaborating with other students might diminish your learning,
which {\em will} most likely affect your grade.

\section{Expectations, suggestions} Naturally we expect you to attend 
class meetings, complete and submit assignments on 
time, prepare for quizzes and exams, and participate in classroom 
activities. We also strongly encourage you to read the textbook
in addition to attending lectures.
In addition to being generally prudent to 
supplement your learning materials, reading the text has a number of 
advantages over attending lectures alone. Namely, the text is beautifully 
typeset and edited by professionals, very clearly written with readers 
of exactly your level in mind, and virtually free of mistakes.

\section{Course schedule}\label{Schedule} While the exact subject
matter to be covered in class shown in the following schedule is
subject to change slightly, the quizzes and exams will be conducted
on {\em exactly} the dates shown.

\begin{tabular}{c|cl|cl|cl|l}
&\multicolumn{2}{c|}{\bf Monday}
&\multicolumn{2}{c|}{\bf Wednesday}
&\multicolumn{2}{c}{\bf Friday}&\multicolumn{1}{c}{\bf Homework}\\
{\bf Week}&{\bf Date}&{\bf In class}
&{\bf Date}&{\bf In class}&{\bf Date}&{\bf In class}&\multicolumn{1}{c}{\bf due Friday}\\\toprule
1&25 Aug&\S1&27 Aug&\S1&29 Aug&\S2&\S1: 7,9,15,18,20,21,25,29\\\midrule
2&1 Sep&Holiday&3 Sep&\S2&5 Sep&\S3&\S2: 11,13,25--28,41--44,59--64\\\midrule
3&8 Sep&\S3&10 Sep&\S4&12 Sep&\S4&\S3: 9,13--16,27--32,56,61\\\midrule
4&15 Sep&\S5&17 Sep&Review&19 Sep&{\bf Exam 1}&\S4:7,11,14,17,20,25--32,53\\
&&&&&&&\S5: 5,9,13,17,21,25\\\midrule
5&22 Sep&\S6&24 Sep&\S6&26 Sep&\S6&\S6: 11,15,17--26,35--38,\\
&&&&&&&49--52,61,85--88\\\midrule
6&29 Sep&\S7&1 Oct&\S7&3 Oct&\S8&\S7:5--10,25--28,35--40,59--64\\\midrule
7&6 Oct&\S8&8 Oct&Review&10 Oct&{\bf Exam 2}&\S8: 21,25,27,31,33,51,54,65\\\midrule
8&13 Sep&\S9&15 Sep&\S9&17 Sep&\S9&\S9: 15,23,29,33,39,41,45,47,51\\\midrule
9&20 Sep&\S10&22 Sep&\S10&24 Sep&\S11&\S10: 1,5,9,19--22,31,41\\\midrule
10&27 Sep&\S11&29 Sep&Review&31 Sep&{\bf Exam 3}&\S11: 5,7,11,15,21\\\midrule
11&3 Nov&Blackjack&5 Nov&Blackjack&7 Nov&Blackjack&Assignment 11\\\midrule
12&10 Nov&Craps&12 Nov&Craps&14 Nov&Craps&Assignment 12\\\midrule
13&17 Nov&Poker&19 Nov&Poker&21 Nov&Poker&Assignment 13\\\midrule
14&1 Dec&Lottery&3 Dec&Lottery&5 Dec&&Assignment 14\\\midrule
15&8 Dec&&10 Dec&Review&12 Dec&Review&Assignment 15\\\midrule
16&&&17 Dec&{\bf Final Exam}&&
\end{tabular}

\section{Math Help Room}\label{MathCenter}
\end{document}
