\documentclass[11pt]{article}
\usepackage[euler-digits]{eulervm}
\usepackage[colorlinks=true, linkcolor=blue, breaklinks=true]{hyperref} 
\def\sectionautorefname~#1\null{\S#1\null}
\usepackage{charter,amsmath,amssymb,breakurl}
\usepackage[letterpaper,margin=.9in]{geometry}
\title{Syllabus for Math~104}
\author{Dr Marcus Bishop, Iowa State University}
\begin{document}
\maketitle

\section{Time, place}\label{Time} Section~29 meets at 15:10--16:00

\section{Requirements, prerequisites, materials}\label{Require}
To take this course, you should have taken Math~165
or the equivalent. You will need the twelth edition of 
\href{http://wps.aw.com/aw_thomas_calculus_series}
{\em Thomas' Calculus, Early Trancendentals}
by Thomas, Weir, and Hass, hereafter referenced by
\href{http://wps.aw.com/aw_thomas_calculus_series}{Thomas}.
You will also want a 
\href{http://en.wikipedia.org/wiki/TI-83}{TI-83} or
\href{http://en.wikipedia.org/wiki/TI-84}{TI-84}
graphing calculator. Unfortunately, we cannot allow you to use
any device with algebraic capabilities greater than those of the TI-83
or TI-84 on quizzes and exams. However, we encourage
you to supplement your learning with a computer algebra system
of your choice.
In class I will use \href{http://www.sagemath.org}{\textsf Sage},
an open-source computer algebra system that you can install on your
computer for free if you wish.

\section{Instructor information}
\section{Course goals, content}
\section{Calculation of grades}\label{Assessment}
\section{Written assignments}\label{Written}
\section{Expectations, suggestions}
\section{Expected learning outcomes}
\section{Course schedule}\label{Schedule} While the exact subject
matter to be covered in class shown in the following schedule is
subject to change slightly, the quizzes and exams will be conducted
on {\em exactly} the dates shown.

\begin{tabular}{c|cl|cl|cl|cl}
&\multicolumn{2}{c|}{\bf Monday}
&\multicolumn{2}{c|}{\bf Wednesday}
&\multicolumn{2}{c|}{\bf Thursday}
&\multicolumn{2}{c}{\bf Friday}\\
{\bf Week}&{\bf Date}&{\bf In class}&{\bf Date}&{\bf In class}
&{\bf Date}&{\bf In class}&{\bf Date}&{\bf In class}\\\hline
1&13 Jan&\S6.1&15 Jan&\S6.2&16 Jan&&17 Jan&\S7.1\\
2&20 Jan&No class&22 Jan&\S7.1&23 Jan&{\bf Quiz 1}&24 Jan&\S7.2\\
3&27 Jan&\S7.4&29 Jan&\S7.4&30 Jan&&31 Jan&\S7.3\\
4&3 Feb&\S7.3&5 Feb&Review&6 Feb&{\bf Exam 1}&7 Feb&\S6.4\\
5&10 Feb&\S7.5&12 Feb&\S7.6&13 Feb&&14 Feb&\S8.1\\
6&17 Feb&\S8.1&19 Feb&\S8.2&20 Feb&{\bf Quiz 2}&21 Feb&\S8.4\\
7&24 Feb&\S8.4&26 Feb&\S8.5&27 Feb&&28 Feb&\S8.3\\
8&3 Mar&Review&5 Mar&Review&6 Mar&{\bf Exam 2}&7 Mar&\S9.1\\
9&10 Mar&\S9.1&12 Mar&\S9.2&13 Mar&&14 Mar&\S9.3\\
10&24 Mar&\S9.4&26 Mar&\S9.5&27 Mar&{\bf Quiz 3}&28 Mar&\S9.5\\
11&31 Mar&\S10.1&2 Apr&\S10.2&3 Apr&&4 Apr&\S10.2\\
12&7 Apr&\S10.3&9 Apr&\S10.3&10 Apr&{\bf Quiz 4}&11 Apr&\S10.4\\
13&14 Apr&Review&16 Apr&Review&17 Apr&{\bf Exam 3}&18 Apr&\S11.1\\
14&21 Apr&No class&23 Apr&\S11.2&24 Apr&&25 Apr&\S11.4\\
15&28 Apr&\S11.4&30 Apr&Review&1 May&{\bf Quiz 5}&2 May&Review\\
&&&&&8 May&{\bf Final Exam}
\end{tabular}

\section{Math Help Room}\label{MathCenter}
\end{document}
