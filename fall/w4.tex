\documentclass[12pt]{article}
\usepackage{mdwlist}
\usepackage[euler-digits]{eulervm}
\usepackage{charter,amsmath,amssymb,breakurl}
\usepackage[letterpaper,margin=1in]{geometry}
\usepackage{multicol}
\everymath{\displaystyle}
\author{}\date{Due in class Wednesday 12 November}
\title{Math 104 Worksheet 4}\author{}
\begin{document}
\maketitle
\pagestyle{empty}
\begin{enumerate}
\item A poker hand is called a {\em flush} if all five cards
have the same suit.
\begin{enumerate}
\item\label{hf} Calculate the number of possible {\em heart flush}es,
that is, the number of hands where all five cards have suit $\heartsuit$.
\item Use your response to (\ref{hf}) to count the number of
possible flushes.
\item What is the probability of being dealt a flush?
\end{enumerate}
\item A hand is called a {\em straight} if the ranks of the five
cards appear consecutively in the list
\[A,2,3,4,5,6,7,8,9,10,J,Q,K,A.\]
Note that $A$ appears at the beginning and at the end of the list, but
that no other ``wrapping around'' is allowed.
For example, $7\heartsuit,8\spadesuit,9\clubsuit,10\clubsuit,J\heartsuit$
is a straight.
\begin{enumerate}
\item Calculate the number of possible straights.
\item Calculate the probability of being dealt a straight.
\end{enumerate}
\item A hand is called a {\em straight flush} if it is both
a straight and a flush. Calculate the probability of being dealt
a straight flush.
\item A flush which is not also a straight flush is called a {\em mere flush}.
Calculate the probability of being dealt a mere flush.
\item A straight flush whose cards have ranks $A,K,Q,J,10$ is called
a {\em royal flush}. 
Calculate the probability of being dealt a royal flush.
\end{enumerate}
\end{document}

