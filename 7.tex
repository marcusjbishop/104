\documentclass[handout]{beamer}
\usepackage{multicol}
\usepackage{xy}
\everymath{\displaystyle}
\mode<presentation>
{\usetheme{Warsaw}\setbeamercovered{dynamic}}
\usecolortheme{crane}
\usepackage{beamerfoils}
\pgfdeclareimage[height=1in]{university-logo}{ISULogo}
\logo{\pgfuseimage{university-logo}}
\setbeamertemplate{navigation symbols}{}
\title[\S7]{Section 7\\Conditional probability}
\author{Dr Marcus Bishop}
\subject{Math 104}
\beamerdefaultoverlayspecification{<+->}
\theoremstyle{definition}
\newtheorem{remark}{Remark}
\newtheorem{impact}{Impact}
\newtheorem{notation}{Notation}
\newtheorem{argument}{Argument}
\usepackage{arev}
\begin{document}
\begin{frame}\titlepage\end{frame}
\LogoOff

\begin{frame}{Conditional probability}
\begin{itemize}
\item Probability meant to help people make decisions
\item New information might become available
that changes probability of certain events
\item Taking new information into account
might help in estimating probability
\item In turn, new information might influence decisions made
\end{itemize}
\begin{example}
\begin{itemize}
\item Should I study for a possible pop quiz in Math~104 or not?
\item Hard to calculate probability of pop quiz
\item However, if regular quiz occurred in class yesterday,
probability of pop quiz tomorrow reduced
\end{itemize}
\end{example}
\end{frame}

\begin{frame}
\begin{example}
\begin{itemize}
\item Have to run into store for five minutes
\item Should I put money in parking meter?
\item Hard to calculate probability that meter reader
comes in next five minutes
\item However, if it's raining, probability that meter reader
comes reduced
\end{itemize}
\end{example}
\begin{remark}
In instructor's experience, better to always put money
in parking meter, regardless of rain
\end{remark}
\end{frame}

\begin{frame}
\begin{definition}
\begin{itemize}
\item Define $P\left(E\mid F\right)
=\frac{P\left(E\cap F\right)}{P\left(F\right)}$
\item $P\left(E\mid F\right)$ called
\alert{probability of $E$ given $F$}
\end{itemize}
\end{definition}
\begin{itemize}
\item Normally regard $E$ as occurring
\alert{after} $F$ when considering $P\left(E\mid F\right)$
\item $P\left(E\mid F\right)$ gives probability that $E$ occurs,
under further assumption that $F$ already occurred
\end{itemize}
\end{frame}

\begin{frame}{Example}
\begin{itemize}
\item Calculate probability of drawing $\alert{\varheart}$ given
that card is red
\item Take $E=\left\{\text{suit is $\alert{\varheart}$}\right\}$
and $F=\left\{\text{card is red}\right\}$
\item Then question asks for $P\left(E\mid F\right)
=\frac{P\left(E\cap F\right)}{P\left(F\right)}$
\item $P\left(E\right)=\frac{1}{4}$\qquad
\only<+->{$P\left(F\right)=\frac{1}{2}$}
\item $P\left(E\cap F\right)=\frac{1}{4}$\qquad
($E\cap F=E$ since all $\alert{\varheart}$ are red)
\begin{remark}
Note that $P\left(E\right)\cdot P\left(F\right)\alert{\ne}
P\left(E\cap F\right)$
since $E,F$ not independent
\end{remark}
\item So $P\left(E\mid F\right)
=\frac{P\left(E\cap F\right)}{P\left(F\right)}
\only<+->{=\frac{1/4}{1/2}}
\only<+->{=\frac{1}{2}}$
\end{itemize}
\end{frame}

\begin{frame}
\begin{remark}
\begin{itemize}
\item Suppose $S$ the sample space of experiment
\item Then
$P\left(E\mid F\right)=\frac{P\left(E\cap F\right)}{P\left(F\right)}
\only<+->{=\frac{\left|E\cap F\right|/\left|S\right|}
{\left|F\right|/\left|S\right|}}
\only<+->{=\alert{\frac{\left|E\cap F\right|}{\left|F\right|}}}$
\item Generally easier to calculate
$\frac{\left|E\cap F\right|}{\left|F\right|}$
\item Assuming $F$ occurred, only possible outcomes in $F$
\item So $\left|F\right|$ the denominator of $P\left(E\mid F\right)$
\item Essentially $F$ the \alert{new sample space}
\item Further, any outcome in $E$
also in $F$
\item So $\left|E\cap F\right|$ the numerator of
$P\left(E\mid F\right)$
\end{itemize}
\end{remark}
\end{frame}

\begin{frame}
\[\begin{xy}<1cm,0cm>:
(-1,2)*+!D{\text{E}};
(1,2)*+!D{\text{F}};
(-1,0)*\cir<2cm>{};
(1,0)*\cir<2cm>{};
(0,0)*{E\cap F};
\end{xy}\]
\begin{itemize}
\item We imagine all outcomes lying inside
or outside of circles above
\item Knowing that outcome lies in $F$
eliminates all outcomes lying outside circle $F$
\item Now for outcome to be in $E$, has to be in $E\cap F$
\item Thus $P\left(E\mid F\right)=
\frac{\left|E\cap F\right|}{\left|F\right|}$
\end{itemize}
\end{frame}

\begin{frame}
\begin{example}
\begin{itemize}
\item Calculate probability of drawing $\alert{\varheart}$ given
that card is red
\item $13$ cards are red and $\alert{\varheart}$
\item $26$ cards are red
\item Thus $P\left(\alert{\varheart}\mid\text{red}\right)
=\frac{13}{26}\only<+->{=\frac{1}{2}}$
\end{itemize}
\end{example}
\end{frame}

\begin{frame}{Exercise 53}
\[\begin{array}{r|llll|l}
&\text{same}&\text{same}&\text{different}&&\\
\text{education}&\text{county}
&\text{state}&\text{state}&\text{abroad}&\text{total}\\\hline
\text{no degree}&2247&470&254&98&3069\\
\text{high school}&3842&1074&712&145&5773\\
\text{some college}&3319&1020&707&112&5158\\
\text{bachelor's}&2072&760&667&182&3681\\
\text{graduate}&913&383&461&118&1875\\\hline
\text{total}&12,393&3707&2801&655&19,556
\end{array}\]
\begin{itemize}
\item Table shows where people who moved in 2008
moved to, organized by educational attainment
\item If someone who moved selected at random, find probability
that he/she moved within same county
\item $\frac{\left\{\text{same county total}\right\}}
{\left\{\text{grand total}\right\}}
\only<+->{=\frac{12,393}{19,556}}\only<+->{\approx 0.63}$
\item Not a conditional probability problem
\end{itemize}
\end{frame}

\begin{frame}{Exercise 54}
\[\begin{array}{r|llll|l}
&\text{same}&\text{same}&\text{different}&&\\
\text{education}&\text{county}
&\text{state}&\text{state}&\text{abroad}&\text{total}\\\hline
\text{no degree}&2247&470&254&98&3069\\
\text{high school}&3842&1074&712&145&5773\\
\text{some college}&3319&1020&707&112&5158\\
\text{bachelor's}&2072&760&667&182&3681\\
\text{graduate}&913&383&461&118&1875\\\hline
\text{total}&12,393&3707&2801&655&19,556
\end{array}\]
\begin{itemize}
\item If someone who moved selected at random, find probability
that he/she moved within same county given that he/she has bachelor's
\item $\frac{\left\{\text{same county and bachelor's}\right\}}
{\left\{\text{bachelor's}\right\}}
\only<+->{=\frac{2072}{3681}}\only<+->{\approx 0.56}$
\end{itemize}
\end{frame}

\begin{frame}{Exercise 58 (modified)}
\[\begin{array}{r|llll|l}
&\text{same}&\text{same}&\text{different}&&\\
\text{education}&\text{county}
&\text{state}&\text{state}&\text{abroad}&\text{total}\\\hline
\text{no degree}&2247&470&254&98&3069\\
\text{high school}&3842&1074&712&145&5773\\
\text{some college}&3319&1020&707&112&5158\\
\text{bachelor's}&2072&760&667&182&3681\\
\text{graduate}&913&383&461&118&1875\\\hline
\text{total}&12,393&3707&2801&655&19,556
\end{array}\]
\begin{itemize}
\item If someone who moved selected at random, find probability
that he/she has graduate degree, given that he/she moved abroad
\item $\frac{\left\{\text{graduate and abroad}\right\}}
{\left\{\text{abroad}\right\}}
\only<+->{=\frac{118}{655}}\only<+->{\approx 0.18}$
\end{itemize}
\end{frame}

\begin{frame}{Marble example}
\begin{itemize}
\item Bag contains $5$~red and $7$~blue marbles
\item Two marbles drawn without replacement
\item Calculate probability that second marble red
given that first marble blue
\item No need to use use formula
$\frac{P\left(E\cap F\right)}{P\left(F\right)}$ here
\item Can argue directly:
\begin{itemize}
\item Since first marble was blue, $5$~red and \alert{$6$~blue} remain
\item Thus probability of red on second draw $\frac{5}{5+6}
\only<+->{=\frac{5}{11}}$
\end{itemize}
\end{itemize}
\end{frame}

\begin{frame}
\begin{itemize}
\item Using formula
$\frac{P\left(E\cap F\right)}{P\left(F\right)}$ instead:
\item $P\left(\text{blue first and red second}\right)
=\frac{7}{12}\cdot\frac{5}{11}$
\item $P\left(\text{blue first}\right)=\frac{7}{12}$
\item So $P\left(\text{red second given blue first}\right)
=\frac{\frac{7}{12}\cdot\frac{5}{11}}{\frac{7}{12}}
\only<+->{=\frac{5}{11}}$
\item Same answer because $\frac{7}{12}$ cancels
\item But $\frac{5}{11}$ arises in same way as in previous slide
\item Thus easier to argue as in previous slide
\end{itemize}
\end{frame}


\begin{frame}{Spanish and physics}
\begin{itemize}
\item Ames High School has $245$ students enrolled in Spanish
\item Has $131$ students in physics
\item Has $29$ students enrolled in both Spanish and physics
\item Probability that randomly selected student
enrolled in physics, given that student enrolled in Spanish:
\item $\frac{\left|\text{Spanish and physics}\right|}
{\left|\text{Spanish}\right|}
=\frac{29}{245}\approx 0.118$
\item Probability that student in Spanish given that student in physics:
\item $\frac{\left|\text{Spanish and physics}\right|}
{\left|\text{physics}\right|}
=\frac{29}{131}\approx 0.221$
\end{itemize}
\end{frame}

\begin{frame}{Spanish and physics revisited}
\begin{itemize}
\item Ames High School has 1422 students
\item Has $245$ students enrolled in Spanish
\item Has $131$ students in physics
\item Has $29$ students enrolled in both Spanish and physics
\item Calculate probability that randomly selected student
enrolled in physics, given that student \alert{not} enrolled in Spanish
\end{itemize}
\end{frame}

\begin{frame}
\begin{itemize}
\item 1422 students
\item $245$ in Spanish
\item $131$ in physics
\item $29$ in both Spanish and physics
\item So $245-29=216$ in Spanish but not physics
\item $131-29=102$ in physics but not Spanish
\item $1422-216-29-102=1075$ in neither
\end{itemize}
\[\begin{xy}<1cm,0cm>:
(-1,2)*+!D{\text{S}};
(1,2)*+!D{\text{P}};
(-1,0)*\cir<2cm>{};
(1,0)*\cir<2cm>{};
(0,0)*{29};
(-2,0)*{216};
(2,0)*{102};
(4,0)*{1075};
\end{xy}\]
\end{frame}

\begin{frame}
\begin{itemize}
\item Question asks for $P\left(\text{physics}\mid
\text{\alert{not} Spanish}\right)$
\[\begin{xy}<1cm,0cm>:
(-1,2)*+!D{\text{S}};
(1,2)*+!D{\text{P}};
(-1,0)*\cir<2cm>{};
(1,0)*\cir<2cm>{};
(0,0)*{29};
(-2,0)*{216};
(2,0)*{102};
(4,0)*{1075};
\end{xy}\]
\item $\left|\text{not Spanish}\right|
\only<+->{=102+1075}\only<+->{=1177}$
\item $\left|\text{physics and not Spanish}\right|
\only<+->{=102}$
\item Thus
$P\left(\text{physics}\mid
\text{not Spanish}\right)\only<+->{=\frac{102}{1177}}
\only<+->{\approx 0.086}$
\end{itemize}
\end{frame}

\begin{frame}{Pop quiz}
\[\begin{array}{r|llll|l}
&\text{same}&\text{same}&\text{different}&&\\
\text{education}&\text{county}
&\text{state}&\text{state}&\text{abroad}&\text{total}\\\hline
\text{no degree}&2247&470&254&98&3069\\
\text{high school}&3842&1074&712&145&5773\\
\text{some college}&3319&1020&707&112&5158\\
\text{bachelor's}&2072&760&667&182&3681\\
\text{graduate}&913&383&461&118&1875\\\hline
\text{total}&12,393&3707&2801&655&19,556
\end{array}\]
Calculate probability that
randomly selected individual
\begin{enumerate}
\item moved to different state given that she
has only high school diploma
\item does \alert{not} have graduate degree given that he moved
within same state
\item moved abroad \alert{and} has only bachlor's
\end{enumerate}
\end{frame}

\end{document}
