\documentclass[handout]{beamer}
\usepackage{multicol}
\usepackage{xy}
\everymath{\displaystyle}
\mode<presentation>
{\usetheme{Warsaw}\setbeamercovered{dynamic}}
\usecolortheme{crane}
\usepackage{beamerfoils}
\pgfdeclareimage[height=1in]{university-logo}{ISULogo}
\logo{\pgfuseimage{university-logo}}
\setbeamertemplate{navigation symbols}{}
\title[\S7]{Section 7\\Conditional probability}
\author{Dr Marcus Bishop}
\subject{Math 104}
\beamerdefaultoverlayspecification{<+->}
\theoremstyle{definition}
\newtheorem{remark}{Remark}
\newtheorem{impact}{Impact}
\newtheorem{notation}{Notation}
\newtheorem{argument}{Argument}
\usepackage{arev}
\begin{document}
\begin{frame}\titlepage\end{frame}
\LogoOff

\begin{frame}{Union, intersection, size}
\begin{itemize}
\item Suppose $E,F$ are sets
\item $E\cup F$ denotes set of all elements of either
$E$ or $F$
\item $E\cup F$ called the \alert{union} of $E$ and $F$
\item $E\cap F$ denotes set of all elements in both
$E$ and $F$
\item $E\cap F$ called the \alert{intersection} of $E$ and $F$
\item $\left|E\right|$ denotes number of elements
or \alert{size} of $E$
\end{itemize}
\begin{example}
\begin{itemize}
\item Suppose $E=\left\{1,2,4\right\}$
\item Suppose $F=\left\{1,3,4\right\}$
\item Then $\left|E\right|=3=\left|F\right|$
\item $E\cup F=\left\{1,2,3,4\right\}$
\item $F\cap F=\left\{1,4\right\}$
\end{itemize}
\end{example}
\end{frame}

\begin{frame}{Example}
\begin{align*}
\only<+->{
\text{Suppose}\qquad E&=\left\{a,b,e,f\right\}\\
F&=\left\{b,c,d,f,g\right\}\\
G&=\left\{a,c,f,g\right\}\\}
\only<+->{
\text{Then}\qquad\left(E\cap F\right)\cup G}
\only<+->{
&=\left\{b,f\right\}\cup G\\}
\only<+->{
&=\left\{a,b,c,f,g\right\}\\}
\only<+->{
\left(E\cup G\right)\cap\left(F\cup G\right)}
\only<+->{
&=\left\{a,b,c,e,f,g\right\}\cap\left(F\cup G\right)\\}
\only<+->{
&=\left\{a,b,c,e,f,g\right\}\cap\left\{a,b,c,d,f,g\right\}\\}
\only<+->{
&=\left\{a,b,c,f,g\right\}}
\end{align*}
\only<+->{
\begin{remark}
In general $\left(E\cap F\right)\cup G
=\left(E\cup G\right)\cap\left(F\cup G\right)$
for any sets $E,F,G$
\end{remark}}
\end{frame}


\begin{frame}{Remarks}
\begin{itemize}
\item Order that elements \alert{of set} listed unimportant
\begin{example}
$\left\{1,2,4\right\}=\left\{4,2,1\right\}=\left\{2,1,4\right\}$
\end{example}
\item Size refers to number of \alert{distinct} elements
of set
\item If an element listed more than once, duplicates can be eliminated
\begin{example}
\begin{itemize}
\item Suppose $E=\left\{1,1,2,4,1\right\}$
\item Then $E$ the same set as $\left\{1,2,4\right\}$
\item So $\left|E\right|=3$, not $5$
\end{itemize}
\end{example}
\item Of course, would be inefficient to list
an element more than once
\end{itemize}
\end{frame}

\begin{frame}
\begin{itemize}
\item Now suppose $E,F$ events
\item Since events are sets of outcomes,
$E\cup F$ the same event as $\alert{\text{$E$ or $F$}}$
\item Similarly
$E\cap F$ the same event as $\alert{\text{$E$ and $F$}}$
\item Thus Inclusion-Exclusion Theorem reads:
\[P\left(E\cup F\right)=P\left(E\right)
+P\left(F\right)-P\left(E\cap F\right)\]
\item Set form reads:
\[\left|E\cup F\right|=\left|E\right|
+\left|F\right|-\left|E\cap F\right|\]
\end{itemize}
\begin{example}
\begin{itemize}
\item Suppose $E=\left\{1,2,4\right\}$
\item Suppose $F=\left\{1,3,4\right\}$
\item Then $\left|E\right|
+\left|F\right|-\left|E\cap F\right|
\only<+->{=3+3-2=4}
\only<+->{=\left|E\cup F\right|}$
\end{itemize}
\end{example}
\end{frame}

\begin{frame}{Conditional probability}
\begin{itemize}
\item Probability meant to help people make decisions
\item New information might become available
that changes probability of certain events
\item Taking new information into account
might help in estimating probability
\item In turn, new information might influence decisions made
\end{itemize}
\begin{example}
\begin{itemize}
\item Should I study for a possible pop quiz in Math~104 or not?
\item Hard to calculate probability of pop quiz
\item However, if regular quiz occurred in class yesterday,
probability of pop quiz tomorrow reduced
\end{itemize}
\end{example}
\end{frame}

\begin{frame}
\begin{example}
\begin{itemize}
\item Have to run into store for five minutes
\item Should I put money in parking meter?
\item Hard to calculate probability that meter reader
comes in next five minutes
\item However, if it's raining, probability that meter reader
comes reduced
\end{itemize}
\end{example}
\begin{remark}
In instructor's experience, better to always put money
in parking meter, regardless of rain
\end{remark}
\end{frame}

\begin{frame}
\begin{definition}
\begin{itemize}
\item Define $P\left(E\mid F\right)
=\frac{P\left(E\cap F\right)}{P\left(F\right)}$
\item $P\left(E\mid F\right)$ called
\alert{probability of $E$ given $F$}
\end{itemize}
\end{definition}
\begin{itemize}
\item Normally regard $E$ as occurring
\alert{after} $F$ when considering $P\left(E\mid F\right)$
\item $P\left(E\mid F\right)$ gives probability that $E$ occurs,
under further assumption that $F$ already occurred
\end{itemize}
\end{frame}

\begin{frame}{Example}
\begin{itemize}
\item Calculate probability of drawing $\alert{\varheart}$ given
that card is red
\item Take $E=\left\{\text{suit is $\alert{\varheart}$}\right\}$
and $F=\left\{\text{card is red}\right\}$
\item Then question asks for $P\left(E\mid F\right)
=\frac{P\left(E\cap F\right)}{P\left(F\right)}$
\item $P\left(E\right)=\frac{1}{4}$\qquad
\only<+->{$P\left(F\right)=\frac{1}{2}$}
\item $P\left(E\cap F\right)=\frac{1}{4}$\qquad
($E\cap F=E$ since all $\alert{\varheart}$ are red)
\begin{remark}
Note that $P\left(E\right)\cdot P\left(F\right)\alert{\ne}
P\left(E\cap F\right)$
since $E,F$ not independent
\end{remark}
\item So $P\left(E\mid F\right)
=\frac{P\left(E\cap F\right)}{P\left(F\right)}
\only<+->{=\frac{1/4}{1/2}}
\only<+->{=\frac{1}{2}}$
\end{itemize}
\end{frame}

\begin{frame}
\begin{remark}
\begin{itemize}
\item Suppose $S$ the sample space of experiment
\item Then
$P\left(E\mid F\right)=\frac{P\left(E\cap F\right)}{P\left(F\right)}
\only<+->{=\frac{\left|E\cap F\right|/\left|S\right|}
{\left|F\right|/\left|S\right|}}
\only<+->{=\alert{\frac{\left|E\cap F\right|}{\left|F\right|}}}$
\item Generally easier to calculate
$\frac{\left|E\cap F\right|}{\left|F\right|}$
\item Assuming $F$ occurred, only possible outcomes in $F$
\item So $\left|F\right|$ the denominator of $P\left(E\mid F\right)$
\item Essentially $F$ the \alert{new sample space}
\item Further, any outcome in $E$
also in $F$
\item So $\left|E\cap F\right|$ the numerator of
$P\left(E\mid F\right)$
\end{itemize}
\end{remark}
\end{frame}

\begin{frame}
\[\begin{xy}<1cm,0cm>:
(-1,2)*+!D{\text{E}};
(1,2)*+!D{\text{F}};
(-1,0)*\cir<2cm>{};
(1,0)*\cir<2cm>{};
(0,0)*{E\cap F};
\end{xy}\]
\begin{itemize}
\item We imagine all outcomes lying inside
or outside of circles above
\item Knowing that outcome lies in $F$
eliminates all outcomes lying outside circle $F$
\item Now for outcome to be in $E$, has to be in $E\cap F$
\end{itemize}
\end{frame}

\begin{frame}
\begin{example}
\begin{itemize}
\item Calculate probability of drawing $\alert{\varheart}$ given
that card is red
\item $13$ cards are red and $\alert{\varheart}$
\item $26$ cards are red
\item Thus $P\left(\alert{\varheart}\mid\text{red}\right)
=\frac{13}{26}\only<+->{=\frac{1}{2}}$
\end{itemize}
\end{example}
\end{frame}

\begin{frame}{Exercise 53}
\[\begin{array}{r|llll|l}
&\text{same}&\text{same}&\text{different}&&\\
\text{education}&\text{county}
&\text{state}&\text{state}&\text{abroad}&\text{total}\\\hline
\text{no degree}&2247&470&254&98&3069\\
\text{high school}&3842&1074&712&145&5773\\
\text{some college}&3319&1020&707&112&5158\\
\text{bachelor's}&2072&760&667&182&3681\\
\text{graduate}&913&383&461&118&1875\\\hline
\text{total}&12,393&3707&2801&655&19,556
\end{array}\]
\begin{itemize}
\item Table shows where people who moved in 2008
moved to, organized by educational attainment
\item If someone who moved selected at random, find probability
that he/she moved within same county
\item $\frac{12,393}{19,556}\only<+->{\approx 0.63}$
\item Not a conditional probability problem
\end{itemize}
\end{frame}

\begin{frame}{Exercise 54}
\[\begin{array}{r|llll|l}
&\text{same}&\text{same}&\text{different}&&\\
\text{education}&\text{county}
&\text{state}&\text{state}&\text{abroad}&\text{total}\\\hline
\text{no degree}&2247&470&254&98&3069\\
\text{high school}&3842&1074&712&145&5773\\
\text{some college}&3319&1020&707&112&5158\\
\text{bachelor's}&2072&760&667&182&3681\\
\text{graduate}&913&383&461&118&1875\\\hline
\text{total}&12,393&3707&2801&655&19,556
\end{array}\]
\begin{itemize}
\item If someone who moved selected at random, find probability
that he/she moved within same county given that he/she has bachelor's
\item $\frac{2072}{3681}\only<+->{\approx 0.56}$
\end{itemize}
\end{frame}

\begin{frame}{Exercise 58 (modified)}
\[\begin{array}{r|llll|l}
&\text{same}&\text{same}&\text{different}&&\\
\text{education}&\text{county}
&\text{state}&\text{state}&\text{abroad}&\text{total}\\\hline
\text{no degree}&2247&470&254&98&3069\\
\text{high school}&3842&1074&712&145&5773\\
\text{some college}&3319&1020&707&112&5158\\
\text{bachelor's}&2072&760&667&182&3681\\
\text{graduate}&913&383&461&118&1875\\\hline
\text{total}&12,393&3707&2801&655&19,556
\end{array}\]
\begin{itemize}
\item If someone who moved selected at random, find probability
that he/she has graduate degree, given that he/she moved abroad
\item $\frac{118}{655}\only<+->{\approx 0.18}$
\end{itemize}
\end{frame}

\end{document}
