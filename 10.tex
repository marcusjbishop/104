\documentclass[handout]{beamer}
\usepackage{multicol}
\usepackage{xy}
\everymath{\displaystyle}
\mode<presentation>
{\usetheme{Warsaw}\setbeamercovered{dynamic}}
\usecolortheme{crane}
\usepackage{beamerfoils}
\pgfdeclareimage[height=1in]{university-logo}{ISULogo}
\logo{\pgfuseimage{university-logo}}
\setbeamertemplate{navigation symbols}{}
\title[\S10]{Section 10\\Combinatorics applications}
\author{Dr Marcus Bishop}
\subject{Math 104}
\beamerdefaultoverlayspecification{<+->}
\theoremstyle{definition}
\newtheorem{remark}{Remark}
\newtheorem{impact}{Impact}
\newtheorem{notation}{Notation}
\newtheorem{question}{Question}
\usepackage{arev}
\usepackage{tensor}
\newcommand\npr[2]{\tensor[_{#1}]P{_{#2}}}
\newcommand\ncr[2]{\tensor[_{#1}]C{_{#2}}}
\usepackage{cancel}
\newcommand{\hs}{\alert{\varheart}}
\newcommand{\ds}{\alert{\vardiamond}}
\newcommand{\s}{\spadesuit}
\newcommand{\cs}{\clubsuit}
\begin{document}
\begin{frame}\titlepage\end{frame}
\LogoOff

\begin{frame}{Poker outline}
\begin{itemize}
\item Dealer deals five cards from standard deck
to each player
\item Players make bets and/or replace cards
\item Finally, player with best \alert{hand} wins
\item There are nine classes of poker hands:
\begin{tabular}{lll}
Straight flush&Four-of-a-kind&Full house\\
Flush&Straight&Three-of-a-kind\\
Two pairs&One pair&High card
\end{tabular}
\item Best hand determined by probability
of each hand
(although listed from best to worst above)
\item The smaller the probability, the better the hand
\item If two players have same type of hand, tie
broken in manner depending on type of hand
\end{itemize}
\end{frame}

\begin{frame}{Poker basics}
\begin{itemize}
\item Recall that deck has cards of thirteen \alert{ranks},
$A,K,Q,J,10,9,8,7,6,5,4,3,2$
listed from highest to lowest
\item Each rank has four cards, one of each suit $\hs,\ds,\cs,\s$
\item Thus deck has $13\cdot 4=52$ cards
\item So $\ncr{52}{5}=2,598,960$ the number of possible hands
\end{itemize}
\includegraphics[scale=.3]{cards}
\end{frame}

\begin{frame}{Four of a kind}
\begin{itemize}
\item Hand called \alert{four-of-a-kind} if four cards have
same rank
\item Fifth card and common rank of other four cards unimportant
\item However, if another player also has four-of-a-kind,
then player with four cards of highest rank wins
\begin{example} $8\hs,8\ds,8\cs,8\s,10\cs$ a four-of-a-kind\end{example}
\begin{example} $9\hs,9\ds,9\cs,9\s,3\hs$ a four-of-a-kind,
and beats previous example\end{example}
\end{itemize}
\end{frame}

\begin{frame}{Counting four-of-a-kinds }
\begin{itemize}
\item How many ways to form a four-of-a-kind?
\item Thirteen ways to choose common rank of four cards
\item $48$ cards remain after removing four cards of same rank
\item So fifth card can be any of remaining $48$
\item Thus $13\cdot 48=624$ possible four-of-a-kinds
\item So $\frac{624}{2598960}=\frac{1}{4165}\approx 0.00024$
the probability of being dealt four-of-a-kind
\end{itemize}
\end{frame}

\begin{frame}{Full house}
\begin{itemize}
\item Hand called a \alert{full house} if three
cards have same rank and remaining two have same rank
\begin{example} $8\hs,8\ds,8\cs,10\s,10\s$ a full house\end{example}
\item If another player has full house, then player with
three cards of highest rank wins
\begin{example} $9\hs,9\ds,9\cs,7\hs,7\hs$ a full house,
and beats example above\end{example}
\end{itemize}
\end{frame}

\begin{frame}{Counting full houses}
\begin{itemize}
\item First choose two common ranks
\item $13$~ways to choose first rank,
\item $12$~ways to choose second rank
\item Next, choose three cards of first rank
\item $\ncr{4}{3}=4$ ways
\item Finally, choose two cards of second rank
\item $\ncr{4}{2}=6$ ways
\item So $13\cdot 12\cdot 4\cdot 6=3744$ possible full houses
\item So $\frac{3744}{2598960}=\frac{6}{4165}\approx 0.01444$
the probability of begin dealt a full house
\item Note that $\frac{6}{4165}>\frac{1}{4165}$,
so four-of-a-kind ranks \alert{higher} than full house
\end{itemize}
\end{frame}

\begin{frame}{Three-of-a-kind}
\begin{itemize}
\item Hand called \alert{three-of-a-kind} if
three cards have same rank
\item Remaining two cards should have rank different
from first three (else hand is four-of-a-kind)
\item Also, remaining cards should have rank
different from one another (else hand is full house)
\begin{example} $8\hs,8\ds,8\cs,Q\hs,10\cs$ a three-of-a-kind\end{example}
\end{itemize}
\end{frame}

\begin{frame}{Counting three-of-a-kinds}
\begin{itemize}
\item $13$ choices for common rank of three cards
\item $\ncr{4}{3}=4$ choices for three cards of common rank
\item $52-4=48$ cards have rank different from rank chosen above
\item So $48$~choices for fourth card
\item $48-4=44$~cards have suit different from two suits chosen above
\item So $44$~choices for fifth card
\item However, each choice of fourth and fifth cards appears
\alert{twice}
\item Hence divide final answer by $2$
\item So $\frac{13\cdot 4\cdot 48\cdot 44}{2}=54,912$ three-of-a-kinds
\end{itemize}
\end{frame}

\begin{frame}{Another way to count three-of-a-kinds}
\begin{itemize}
\item (Method courtesy of Wikipedia)
\item $13$ choices for common rank of three cards
\item $\ncr{4}{3}=4$ choices for three cards of common rank
\item $\ncr{12}{2}=66$ choices for ranks of two remaining cards
\item $4$~choices for suit of fourth card 
\item $4$~choices for suit of fifth card 
\item So $13\cdot 4\cdot 66\cdot 4\cdot 4=54,912$ three-of-a-kinds
\item Agrees with previous calculation
\end{itemize}
\end{frame}

\end{document}

