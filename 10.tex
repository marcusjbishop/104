\documentclass[handout]{beamer}
\usepackage{multicol}
\usepackage{xy}
\everymath{\displaystyle}
\mode<presentation>
{\usetheme{Warsaw}\setbeamercovered{dynamic}}
\usecolortheme{crane}
\usepackage{beamerfoils}
\pgfdeclareimage[height=1in]{university-logo}{ISULogo}
\logo{\pgfuseimage{university-logo}}
\setbeamertemplate{navigation symbols}{}
\title[\S10]{Section 10\\Combinatorics applications}
\author{Dr Marcus Bishop}
\subject{Math 104}
\beamerdefaultoverlayspecification{<+->}
\theoremstyle{definition}
\newtheorem{remark}{Remark}
\newtheorem{impact}{Impact}
\newtheorem{notation}{Notation}
\newtheorem{question}{Question}
\usepackage{arev}
\usepackage{tensor}
\newcommand\npr[2]{\tensor[_{#1}]P{_{#2}}}
\newcommand\ncr[2]{\tensor[_{#1}]C{_{#2}}}
\usepackage{cancel}
\newcommand{\hs}{\alert{\varheart}}
\newcommand{\ds}{\alert{\vardiamond}}
\newcommand{\s}{\spadesuit}
\newcommand{\cs}{\clubsuit}
\begin{document}
\begin{frame}\titlepage\end{frame}
\LogoOff

\begin{frame}{Poker gameplay}
\begin{itemize}
\item Dealer deals five cards from standard deck
to each player
\item Players make bets and/or replace cards
\item Finally, player with best \alert{hand} wins
\item There are nine classes of poker hands:
\begin{tabular}{lll}
Straight flush&Four-of-a-kind&Full house\\
Flush&Straight&Three-of-a-kind\\
Two pairs&One pair&High card
\end{tabular}
\item Best hand determined by probability
(although listed from best to worst above)
\item The smaller the probability, the better the hand
\item If two players have same type of hand, way
tie broken depends on type of hand
\end{itemize}
\end{frame}

\begin{frame}{Poker basics}
\begin{itemize}
\item Recall that deck has cards of thirteen \alert{ranks},
$A,K,Q,J,10,9,8,7,6,5,4,3,2$
\item Each rank has four cards, one of each suit $\hs,\ds,\cs,\s$
\item Thus deck has $13\cdot 4=52$ cards
\item So $\ncr{52}{5}=2,598,960$ the number of possible hands
\end{itemize}
\end{frame}

\begin{frame}{Four of a kind}
\begin{itemize}
\item Hand called \alert{four-of-a-kind} if four cards have
same rank
\item Fifth card and rank of other four cards unimportant
\item However, if another player also has four-of-a-kind,
then player with four cards of highest rank wins
\begin{example} $8\hs,8\ds,8\cs,8\s,10\cs$ a four-of-a-kind\end{example}
\begin{example} $9\hs,9\ds,9\cs,9\s,3\hs$ a four-of-a-kind
and beats previous example\end{example}
\end{itemize}
\end{frame}

\begin{frame}{Counting four-of-a-kind hands}
\begin{itemize}
\item How many ways to form a four-of-a-kind?
\item Thirteen ways to choose rank of four cards of same rank
\item $48$ cards remain after removing four cards of same rank
\item So fifth card can be any of remaining $48$
\item Thus $13\cdot 48=624$ possible four-of-a-kinds
\item So $\frac{624}{2598960}=\frac{1}{4165}\approx 0.00024$
the probability of being dealt four-of-a-kind
\end{itemize}
\end{frame}

\end{document}

