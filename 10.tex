\documentclass[handout,xcolor=dvipsnames]{beamer}
\usepackage{multicol}
\usepackage{xy}
\everymath{\displaystyle}
\mode<presentation>
{\usetheme{Warsaw}\setbeamercovered{dynamic}}
\usecolortheme{crane}
\usepackage{beamerfoils}
\pgfdeclareimage[height=1in]{university-logo}{ISULogo}
\logo{\pgfuseimage{university-logo}}
\setbeamertemplate{navigation symbols}{}
\title[\S10]{Section 10\\Combinatorics applications}
\author{Dr Marcus Bishop}
\subject{Math 104}
\beamerdefaultoverlayspecification{<+->}
\theoremstyle{definition}
\newtheorem{remark}{Remark}
\newtheorem{impact}{Impact}
\newtheorem{situation}{Situation}
\newtheorem{question}{Question}
\usepackage{arev}
\usepackage{tensor}
\newcommand\npr[2]{\tensor[_{#1}]P{_{#2}}}
\newcommand\ncr[2]{\tensor[_{#1}]C{_{#2}}}
\usepackage{cancel}
\newcommand{\hs}{\alert{\varheart}}
\newcommand{\ds}{\alert{\vardiamond}}
\newcommand{\s}{\spadesuit}
\newcommand{\cs}{\clubsuit}
\begin{document}
\begin{frame}\titlepage\end{frame}
\LogoOff

\begin{frame}{Poker outline}
\begin{itemize}
\item Dealer deals five cards from standard deck
to each player
\item Selection of five cards called a \alert{hand}
\item For simplicity, we imagine all five cards in player's hand
\item In reality, some cards lying face up on table
\item Players make bets and/or replace cards
\item Finally, player with \alert{best hand} wins
\item Nine categories of poker hands:
\begin{tabular}{lll}
Straight flush&Four-of-a-kind&Full house\\
Flush&Straight&Three-of-a-kind\\
Two pairs&One pair&High card
\end{tabular}
\item Strength of hand determined by its probability
%(although listed from best to worst above)
\item The smaller the probability, the stronger the hand
\item If two players have same type of hand, tie
broken in manner depending on type of hand
\end{itemize}
\end{frame}

\begin{frame}{Playing cards}
\begin{itemize}
\item Recall that deck has cards of thirteen \alert{ranks},
$A,K,Q,J,10,9,8,7,6,5,4,3,2$
listed from highest to lowest
\item Each rank has four cards, one of each \alert{suit} $\hs,\ds,\cs,\s$
\item Thus deck has $13\cdot 4=52$ cards
\end{itemize}
\includegraphics[scale=.3]{cards}
\end{frame}

\begin{frame}{Poker basics}
\begin{itemize}
\item So $\ncr{52}{5}=2,598,960$ the number of possible hands
\item Will first calculate probability that player dealt
hand of particular type
\item This probability not altered by fact that other cards
might have already been dealt to other players
\item \dots provided that those cards \alert{not exposed to player}
\item However, some variants of poker involve exposing cards
\end{itemize}
\begin{example}
If card of suit $\hs$ lying face up on table, affects probability
that next card drawn will have suit $\hs$
\end{example}
\end{frame}

\begin{frame}{Four of a kind}
\begin{itemize}
\item Hand called \alert{four-of-a-kind} if four cards have
same rank
\item Fifth card and common rank of first four cards unimportant
\item However, if another player also has four-of-a-kind,
then player with four cards of highest rank wins
\begin{example} $8\hs,8\ds,8\cs,8\s,10\cs$ a four-of-a-kind\end{example}
\begin{example} $9\hs,9\ds,9\cs,9\s,3\hs$ a four-of-a-kind,
beats hand above\end{example}
\end{itemize}
\end{frame}

\begin{frame}{Counting four-of-a-kinds }
\begin{itemize}
\item How many ways to form four-of-a-kind?
\item Thirteen ways to choose common rank of four cards
\item $52-4=48$ cards remain
\item So $48$ ways to choose fifth card
\item Thus $13\cdot 48=624$ possible four-of-a-kinds
\item So $\frac{624}{2598960}=\frac{1}{4165}\approx 0.00024$
the probability of being dealt four-of-a-kind
\end{itemize}
\end{frame}

\begin{frame}{Full house}
\begin{itemize}
\item Hand called a \alert{full house} if three
cards have same rank and remaining two have same rank
\begin{example} $8\hs,8\ds,8\cs,10\s,10\ds$ a full house,
an \alert{eights-over-tens full house}\end{example}
\item If another player has full house, then player with
three cards of highest rank wins
\begin{example} $9\hs,9\ds,9\cs,7\hs,7\s$ a full house,
a \alert{nines-over-sevens full house},
beats hand above\end{example}
\end{itemize}
\end{frame}

\begin{frame}{Counting full houses}
\begin{itemize}
\item First choose two common ranks
\item $13$~ways to choose first rank
\item $12$~ways to choose second rank
\item Next, choose three cards of first rank
\item $\ncr{4}{3}=4$ ways
\item Finally, choose two cards of second rank
\item $\ncr{4}{2}=6$ ways
\item So $13\cdot 12\cdot 4\cdot 6=3744$ possible full houses
\item So $\frac{3744}{2598960}=\frac{6}{4165}\approx 0.00144$
the probability of begin dealt full house
\item Note that $\frac{6}{4165}>\frac{1}{4165}$,
so four-of-a-kind beats full house
\end{itemize}
\end{frame}

\begin{frame}
\newcommand{\bl}{\underline{\hspace{3mm}}}
\begin{example} $8\hs,8\ds,8\cs,10\s,10\ds$ a full house\end{example}
\begin{itemize}
\item $13\cdot 12\cdot\ncr{4}{3}\cdot\ncr{4}{2}$ possible full houses
(see previous slide)
\item $13$ choices for first rank, $8$ in example above
\item At this stage, have chosen
$8\bl,8\bl,8\bl,\bl\;\bl,\bl\;\bl$
\item $12$ choices for second rank, $10$ in example
\item At this stage, have chosen
$8\bl,8\bl,8\bl,10\bl,10\bl$
\item $\ncr{4}{3}$ choices of suits of cards of first
rank, $\hs,\ds,\cs$ in example
\item At this stage, have chosen
$8\hs,8\ds,8\cs,10\bl,10\bl$
\item $\ncr{4}{2}$ choices of suits of cards of second
rank, $\s,\ds$ in example
\item At this stage, have chosen
$8\hs,8\ds,8\cs,10\s,10\ds$
\end{itemize}
\end{frame}

\begin{frame}{Two-pair}
\begin{itemize}
\item Hand called \alert{two-pair} if two
cards have same rank and another two have same rank
\item Rank of second pair should be different than rank of first,
else hand a four-of-a-kind
\item Fifth card should have rank different than ranks
of remaining cards, else hand a full house
\end{itemize}
\end{frame}

\begin{frame}
\begin{itemize}
\begin{example} $2\hs,2\ds,Q\cs,Q\s,10\ds$ a two-pair\end{example}
\item If another player has two-pairs, tie broken
by comparing highest ranking pairs
\begin{example} $K\s,K\hs,7\ds,7\s,A\s$ a two-pair,
beats hand above, since $K$ higher than $Q$\end{example}
\item If two players have
highest ranking pair of same rank, then tie broken
by comparing lowest ranking pairs
\begin{example} $K\cs,K\ds,J\hs,J\cs,3\hs$ a two-pair,
beats both hands above\end{example}
\end{itemize}
\end{frame}

\begin{frame}{Counting two-pairs}
\begin{itemize}
\item First choose two common ranks
\item $\ncr{13}{2}=78$~ways to choose ranks
\item Next, choose two cards of first rank
\item $\ncr{4}{2}=6$ ways
\item Next, choose two cards of second rank
\item $\ncr{4}{2}=6$ ways
\item Finally, choose fifth card
\item $52-8=44$ remaining cards have ranks different
than ranks chosen above
\item So $78\cdot 6\cdot 6\cdot 44=123,552$ possible two-pairs
\item So $\frac{123552}{2598960}=\frac{198}{4165}\approx 0.04754$
the probability of begin dealt two-pair
\end{itemize}
\end{frame}

\begin{frame}
\begin{itemize}
\item But why were common ranks chosen with $\ncr{13}{2}$
but with $13\cdot 12$ in full house calculation?
\item Because in full house, one common rank has \alert{three}
cards, while other common rank has \alert{two}
\item In two-pair, $13\cdot 12$ would have counted
each hand \alert{twice}
\item So we instead used
$\ncr{13}{2}=\frac{13\cdot 12}{2!}$
\item $2!$ in denominator accounts for duplication
\item Analogous to factorials in denominator of
anagram calculation
\begin{example}
\begin{itemize}
\item Word \alert{book} has
$\frac{4!}{2!}=12$ anagrams
\item Role of $2!$ to eliminate duplicate anagrams
resulting from swapping o's
\end{itemize}
\end{example}
\end{itemize}
\end{frame}

\begin{frame}{Three-of-a-kind}
\begin{itemize}
\item Hand called \alert{three-of-a-kind} if
three cards have same rank
\item Remaining two cards should have rank different
than first three (else hand is four-of-a-kind)
\item Also, remaining cards should have rank
different than one another (else hand is full house)
\begin{example} $8\hs,8\ds,8\cs,Q\hs,10\cs$ a three-of-a-kind\end{example}
\end{itemize}
\end{frame}

\begin{frame}{Counting three-of-a-kinds}
\begin{itemize}
\item $13$ choices for common rank of three cards
\item $\ncr{4}{3}=4$ choices for three cards of rank chosen above
\item $52-4=48$ cards have rank different than rank chosen above
\item So $48$~choices for fourth card
\item $48-4=44$~cards have rank different than two ranks chosen above
\item So $44$~choices for fifth card
\item However, order in which fourth and fifth cards selected unimportant
\item $48\cdot 44$ counts each choice \alert{twice}
\item Hence divide final answer by $2$
\item So $\frac{13\cdot 4\cdot 48\cdot 44}{2}=54,912$ three-of-a-kinds
\end{itemize}
\end{frame}

\begin{frame}{Another way to count three-of-a-kinds}
\begin{itemize}
\item $13$ choices for common rank of three cards
\item $\ncr{4}{3}=4$ choices for three cards of rank chosen above
\item $\ncr{12}{2}=66$ choices for ranks of two remaining cards
\item $4$~choices for suit of fourth card 
\item $4$~choices for suit of fifth card 
\item So $13\cdot 4\cdot 66\cdot 4\cdot 4=54,912$ three-of-a-kinds
\item Agrees with previous calculation
\item So $\frac{54912}{2598960}=\frac{88}{4165}\approx 0.02113$ the probability
of being dealt four-of-a-kind
\end{itemize}
\end{frame}

\begin{frame}
\begin{situation}
\begin{itemize}
\item You have $3\hs,8\hs,J\hs,Q\hs,5\s$
\item Would like to replace $5\s$ with card of suit $\hs$ to complete flush
\item You know your opponent has two-pair (by telepathy or otherwise)
\item You surmise she would like to exchange fifth card to complete full house
\end{itemize}
\end{situation}
\begin{itemize}
\item $52-5=47$ cards remain unknown to you
\item You have four $\hs$'s so nine $\hs$'s remain
\item So you receive $\hs$ with probability $\frac{9}{47}$
\item But $\frac{9}{47}$ \alert{not} your probability of winning!
\item Opponent must also \alert{fail} to complete her full house,
since full house beats flush
\end{itemize}
\end{frame}

\begin{frame}
\begin{itemize}
\item Opponent has two pair
\item For concreteness, suppose she has $4\s,4\cs,5\ds,5\s$
\item So she completes full house if she receives $4$ or $5$
\item So she \alert{fails} to complete full house
if rank of card dealt not four \alert{and} not five
\item Since she has two fours, two remain unknown to her
\item Since has two fives, two remain unknown to her
\item So $47-2-2=43$ unknown cards are not $4$ and not $5$
\item Thus $\frac{43}{47}$ her probability of \alert{failing}
\item So $\frac{9}{47}\cdot\frac{43}{47}\approx 0.1752$
your probability of winning
\end{itemize}
\end{frame}

\begin{frame}
\begin{situation}
\begin{itemize}
\item You have $J\hs,J\cs,4\s,4\hs,Q\cs$
\item You could exchange $Q\cs$, hoping to complete full house
\item You could also exchange $4\s,4\hs,Q\cs$, hoping to build better hand
\end{itemize}
\end{situation}
\begin{itemize}
\item Suppose you exchange $Q\cs$
\item Two of $47$ cards unknown to you are jacks, two are fours
\item Thus $\frac{4}{47}\approx 0.0851$ your probability
of completing full house
\end{itemize}
\end{frame}

\begin{frame}
\begin{itemize}
\item Suppose instead you exchange $4\s,4\hs,Q\cs$
\item First calculate probability that exactly one of new cards
a jack
\item Then you have three-of-a-kind
\item Additionally, if last two cards have same rank,
then you have full house, a \alert{jacks-over-something full house}
\item $47$ cards remain unknown to you, two of which jacks
\item So $45$ unknown cards have rank other than jack
\item Want to receive one jack and two non-jacks
\item Could happen in three ways, since jack could be
first, second, or third card
\item So $\frac{45}{47}\cdot\frac{44}{46}\cdot\frac{2}{45}
+\frac{45}{47}\cdot\frac{2}{46}\cdot\frac{44}{45}
+\frac{2}{47}\cdot\frac{45}{46}\cdot\frac{44}{45}\approx 0.1221$
the probability of jacks-over-something full house
\end{itemize}
\end{frame}

\begin{frame}
\begin{itemize}
\item Assuming you exchange $4\s,4\hs,Q\cs$
might instead receive \alert{two} jacks
\item Then you have four-of-a-kind
\item $47$ cards remain unknown to you, two of which jacks
\item So $45$ unknown cards have rank other than jack
\item Then \[\frac{45}{47}\cdot\frac{2}{46}\cdot\frac{1}{45}
+\frac{2}{47}\cdot\frac{45}{46}\cdot\frac{1}{45}
+\frac{2}{47}\cdot\frac{1}{46}\cdot\frac{45}{45}
=\frac{3}{1081}\approx 0.0028\]
your probability of four-of-a-kind
\end{itemize}
\end{frame}

\begin{frame}
An alternate four-of-a-kind calculation:
\begin{itemize}
\item You discard $4\s,4\hs,Q\cs$ and receive three new cards
\item $47$ cards remain unknown to you, two of which jacks
\item $\ncr{47}{3}=16,215$ ways to receive $3$ cards from remaining $47$
\item How many combinations have two jacks?
\item If two cards are jacks, then third card not a jack
\item $45$ non-jacks remain, so $45$ combinations
have two jacks
\item Thus $\frac{45}{16215}=\frac{3}{1081}\approx 0.0028$
the probability of four-of-a-kind
\end{itemize}
\end{frame}

\begin{frame}
\begin{itemize}
\item Assuming you exchange $4\s,4\hs,Q\cs$
might instead receive three cards of same rank
\item Then you have \alert{something-over-jacks full house}
\item First choose common rank of three cards
\item $10$ ways to choose rank other than $J,4,Q$
\item Then choose three cards of this rank
\item $\ncr{4}{3}=4$ ways to choose cards
\item So $10\cdot 4=40$ ways to choose three cards of
same rank other than $J,4,Q$
\item Also, three queens remain, so $41$ ways
to choose three cards of same rank
\item $\ncr{47}{3}=16,215$ possible ways to
choose three of remaining $47$ cards
\item Thus $\frac{41}{16215}\approx 0.0025$
your probability of something-over-jacks full house
\end{itemize}
\end{frame}

\begin{frame}
An alternate something-over-jacks calculation:
\begin{itemize}
\item $40$ of remaining $47$ cards of rank different than $J,4,Q$
\item So $\frac{40}{47}$ the probability of drawing
a card of rank different than $J,4,Q$
\item Afterwards three of remaining $46$ cards have same rank as first card
\item Afterwards two of remaining $45$ cards have same rank
as first two
\item So $\frac{40}{47}\cdot\frac{3}{46}\cdot\frac{2}{45}
=\frac{8}{3243}\approx 0.0025$ the probability
of drawing three cards of same rank
\item Note that calculation on previous slide includes
possibility of receiving $Q\hs,Q\ds,Q\s$, excluded from this calculation
\end{itemize}
\end{frame}

\begin{frame}{Summary}
\begin{itemize}
\item You have $J\hs,J\cs,4\s,4\hs,Q\cs$
\item If you discard $Q\cs$ then $0.0851$
your probability of completing full house
\item If you discard $4\s,4\hs,Q\cs$ then
\begin{itemize}
\item $0.1221$ your probability of three-of-a-kind
(including jacks-over-something full house)
\item $0.0028$ your probability of four-of-a-kind
\item $0.0025$ your probability of something-over-jacks full house
\end{itemize}
\item So $0.1221+0.0028+0.0025=0.1274$ your probability
of three-of-a-kind or better
\item $0.1274\approx\frac{1}{8}$
\item Thus chances of winning much better if you
if you discard $4\s,4\hs,Q\cs$
\item \dots unless original two-pair certain to win
\end{itemize}
\end{frame}

\begin{frame}{Video poker}
\begin{itemize}
\item Coin operated poker machines existed since 1880
\item Quickly became popular in west-coast
saloons and cigar stores
\item Thanks to video chip, video poker appears in late 70's
\item More history:
\begin{itemize}
\item \alert{Space Invaders}, June 1978
\item \alert{Asteroids}, November 1979, Atari's best-selling game of all time
\item \alert{Pac-Man}, May 1980
\item \alert{Draw Poker}, 1979
\end{itemize}
\item Video poker increasingly popular in casinos
throughout 80's
\item Less intimidating than casino table games
\item Today increasingly popular online, one of most popular
forms of gambling
\end{itemize}
\end{frame}

\begin{frame}{Jacks-or-better}
\begin{multicols}{2}
\begin{itemize}
\item We consider \alert{Jacks-or-better},
an implementation of video poker
\item Player makes bet
\item Machine deals five cards
\item Player chooses $0$--$5$ cards to replace with new cards
\item If hand has pair of jacks or better,
player gets $\text{bet}\times\text{payoff}$,
where payoff shown in
following schedule:
\end{itemize}
\setbeamercolor{mycolor}{bg=Lavender}
\begin{beamercolorbox}{mycolor}
\begin{tabular}{lr}
Hand&Payoff\\\hline
Royal flush&2500\\
Straight flush&50\\
Four-of-a-kind&25\\
Full house&8\\
Flush&5\\
Straight&4\\
Three-of-a-kind&3\\
Two-pair&2\\
Jacks or better&1
\end{tabular}
\end{beamercolorbox}
\end{multicols}
\end{frame}

\begin{frame}{Observations}
\begin{multicols}{2}
\begin{itemize}
\item Royal flush its own category
\item Payoff tables vary among implementations
\item Payoff shows amount player receives, not net proceeds
\item Hands listed according to increasing probability
\item However, payoffs not proportional to probabilities
\item Last entry refers to
single pair of jacks, queens, or kings, no better
\item No payoff for other pairs
\end{itemize}
\setbeamercolor{mycolor}{bg=Lavender}
\begin{beamercolorbox}{mycolor}
\begin{tabular}{lr}
Hand&Payoff\\\hline
Royal flush&2500\\
Straight flush&50\\
Four-of-a-kind&25\\
Full house&8\\
Flush&5\\
Straight&4\\
Three-of-a-kind&3\\
Two-pair&2\\
Jacks or better&1
\end{tabular}
\end{beamercolorbox}
\end{multicols}
\end{frame}

\begin{frame}{Surprising example}
\begin{itemize}
\item Suppose original hand $9\hs,10\hs,J\hs,Q\hs,K\hs$
\item Suppose bet $\$1$
\item Payoff $\$50$ for straight flushes
\item However, hand one card away from royal flush
\item By discarding $9\hs$,
might receive $A\hs$, forming royal flush
\item Payoff $\$2500$ for royal flushes!
\item Expected payoff if no cards exchanged: $\$50$
\item Expected payoff if $9\hs$ discarded:
\[\frac{1}{47}\left(2500\right)+\text{positive terms}
\only<+->{=\$53.19}
\only<+->{>\$50}\]
\item \alert{positive terms} the proceeds if
hand not royal flush but better than pair of jacks
\item We don't bother to calculate \alert{positive terms}
since $53.19$ already greater than $50$
\end{itemize}
\end{frame}

\begin{frame}{Remarks}
\begin{itemize}
\item Video poker very simple in comparison with poker
\item No opponents, no bluffing, no bet raising 
\item Player makes single decision, namely how many
cards to exchange
\item Player wins amount specified in table, game is over
\item Game can be played in under 30 seconds
\item So players generally play many times, making small bets
\item Expectation helpful with such games
\item Indeed, expectation predicts what should
happen, on average, if 
experiment repeated long enough
\end{itemize}
\end{frame}

\begin{frame}
\begin{itemize}
\item Straight flush $9$--$K$ very unlikely
\item $\frac{4}{\ncr{52}{5}}\approx 0.00000154$ the probability
\item However, receiving $A\hs$ needed to complete flush
also unlikely
\item $\frac{1}{47}\approx 0.0212$ the probability
\item If playing only one game, player should
keep $9\hs,10\hs,J\hs,Q\hs,K\hs$, win $\$50$
\item However, if repeated play intended, would
be desirable to develop \alert{strategy}, taking
expectation into account
\item A \alert{strategy} a systematic plan specifying 
how player should act in every possible situation
\end{itemize}
\end{frame}

\begin{frame}{Developing video poker strategy}
\begin{itemize}
\item Consider each of $\ncr{52}{5}$ hands separately
\item Given a particular hand, enumerate
all possible actions player could take
\item Multiply payoff by probability
\item Gives expected payoff if player chooses this action
\item Repeat for other possible actions
\item Whichever action has greatest
expected payoff, player should take
\item Repeat for remaining $\ncr{52}{5}-1$ hands
\end{itemize}
\begin{example}
\begin{itemize}
\item Suppose player receives $9\hs,10\hs,J\hs,Q\hs,K\hs$
\item $50$ the expected payoff if she discards no cards
\item $53.19$ the expected payoff if she discards
one card, namely $9\hs$
\item So she should discard $9\hs$
\end{itemize}
\end{example}
\end{frame}

\begin{frame}{Example}
\begin{itemize}
\item Suppose you receive
$3\cs,3\hs,5\cs,J\cs,Q\cs$
\item Plan A: discard $3\hs$
\begin{itemize}
\item Would have flush if you receive $\cs$
\item Nine remain, so $\frac{9}{47}\approx
0.1915$ the probability
\item Payout $\$5$ for flush
\item Would have pair if you receive jack or queen
\item Six remain, so $\frac{6}{47}\approx 0.1277$
the probability
\item Payout $\$1$ for pair
\item So $5\cdot\frac{9}{47}+1\cdot\frac{6}{47}
\approx 1.0851$ the expected proceeds for Plan A
\end{itemize}
\end{itemize}
\end{frame}

\begin{frame}
\begin{itemize}
\item Plan B: keep $3\cs,3\hs$, discard rest
\item Would have three-of-a-kind if you receive
a three and two cards of different ranks
\item Can happen in $\left(2\cdot 45\cdot 41\right)\frac{1}{2}
=1845$ ways,
so probability $\frac{11070}{\ncr{47}{3}}\approx 0.1138$
\item Alternate calculation:
$3\left(\frac{2}{47}\cdot\frac{45}{46}
\cdot\frac{41}{45}\right)$
\item Payout $\$3$ for three-of-a-kind
\item Would have threes-over-something full house if
you receive one three and two cards of same rank
\item Payout $\$8$ for full house
\item Can happen in $2\cdot 10\cdot\ncr{4}{2}=120$ ways
so probability $\frac{120}{\ncr{47}{3}}\approx 0.0074$
\end{itemize}
\end{frame}

\begin{frame}
\begin{itemize}
\item Would have four-of-a-kind if you receive
two threes and one card of different rank
\item Can happen in $45$ ways
so probability $\frac{45}{\ncr{47}{3}}\approx 0.0028$
\item Payout $\$25$ for four-of-a-kind
\item Would have something-over-threes full house
if you receive three cards of same rank
\item Can happen in $10\cdot\ncr{4}{3}+3=43$ ways
so probability $\frac{43}{\ncr{47}{3}}\approx 0.0027$
\item Thus $3\left(0.1138\right)+8\left(0.0074\right)
+25\left(0.0028\right)+8\left(0.0027\right)
\approx 0.4922$ the expected payoff for Plan B
\end{itemize}
\end{frame}

%\begin{frame}
%\begin{itemize}
%\item Plan C: keep $J\cs,Q\cs$

\begin{frame}
\begin{itemize}
\item Similar calculation should be performed for each
of $\ncr{52}{5}$ hands
\item Then we know what action player should take in
every possible situation
\item Assuming player follows strategy, can calculate
overall expectation of game
\item Namely, multiply expected payoff for each hand by
$\frac{1}{\ncr{52}{5}}$ and add
\end{itemize}
\end{frame}

\begin{frame}
\begin{itemize}
\item One such calculation gives $1.0239$
\item Note that $1.0239>0$
\item So video poker the first casino game seen
yet with positive expectation
\item \dots provided that player follows strategy
\item Remains to teach strategy to player
\item Impractical to require player to look up strategy
in long list
\end{itemize}
\end{frame}

\begin{frame}{Example Jacks-or-Better strategy}
\begin{itemize}
\item Player memorizes following list:
\begin{enumerate}
\item Four-of-a-kind, straight flush, royal flush
\item Four to a royal flush
\item Three of a kind, straight, flush, full house
\item Four to a straight flush
\item[]$\vdots$
\setcounter{enumi}{15}
\item Discard everything
\end{enumerate}
\item Full list appears at \href{http://wizardofodds.com/games/video-poker/strategy/jacks-or-better/9-6/simple}{\color{blue}\tt www.wizardofodds.com}
\item Player mentally enumerates all ways to improve hand
\item Whichever improvement appears higher in list,
player should choose
\item Note that any strategy highly dependent
on payoff table
\end{itemize}
\end{frame}

\begin{frame}{Pop quiz}
\setbeamercolor{mycolor}{bg=Salmon}
\begin{beamercolorbox}{mycolor}
\begin{enumerate}
\item How many subsets does a set with eight elements have?
\item I can cook eight different meals; need to cook dinner
three times for house-guest; how many ways can dinner menus be selected,
not repeating meals?
\item Animal rescue shelter has five dogs and six cats; how many
ways can I select two of each to adopt?
\end{enumerate}
\end{beamercolorbox}
\setbeamercolor{mycolor}{bg=Goldenrod}
\begin{beamercolorbox}{mycolor}
\begin{enumerate}
\item How many subsets does a set with nine elements have?
\item Carver Hall has six entrances; how many ways can I
enter and exit building if I enter and exit from different doors?
\item I have six shirts and seven ties; how many ways can I select
two of each to pack for trip?
\end{enumerate}
\end{beamercolorbox}
\end{frame}

\end{document}

