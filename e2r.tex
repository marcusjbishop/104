\documentclass[answers,addpoints,12pt]{exam}
\usepackage{multicol,xy}
\usepackage{graphicx,multicol}
\usepackage[euler-digits]{eulervm}
\usepackage{charter,amsmath,amssymb}
\usepackage[letterpaper,margin=1in]{geometry}
\pagestyle{headandfoot}
\runningheadrule
\firstpageheader{\bf Red Form}{\bf Exam 2}
{\bf 4 March 2015}
\runningheader{\bf Red Form}
{\bf Exam Two, Page \thepage\ of \numpages}
{\bf 4 March 2015}
\firstpagefooter{}{}{}
\runningfooter{}{}{}
\everymath{\displaystyle}
\begin{document}

\ifprintanswers\else
\begin{center}
\fbox{\fbox{\parbox{5.5in}{
This exam has \numquestions~questions.
It has been printed on \numpages~pages and is worth \numpoints~points.
Answer all the questions below in the spaces provided.
In order to receive maximum credit, you must
clearly indicate how you arrived at your answers.
By signing below, you pledge that
\begin{enumerate}
\item you will not communicate to any person in any conceivable way anything
about the contents of this exam
until all students have taken the exam, and
\item in taking this exam now,
you have not been the recipient of such communication from anyone else.
\end{enumerate}}}}
\end{center}
\vspace{.2in}
\makebox[\textwidth]{Your signature:\enspace\hrulefill}\\
\vspace{.2in}\\
\makebox[\textwidth]{Your name:\enspace\hrulefill}\\
\vspace{.2in}\\
\makebox[\textwidth]{Your student ID number:\enspace\hrulefill}
\fi

\begin{questions}

\begin{multicols}{2}
\question[12]
Suppose that $E$, $F$, and $G$ are subsets
of the universal set \[U=\left\{d,e,i,m,n,p,r,s,u,t\right\}.\]
The elements of
$E,F,G$ are as shown in the Venn diagram
at the right.
List all the elements of the following sets.
\begin{parts}
\part $E\cup\left(F\cap G\right)$
\begin{solution}[.5in]
$\left\{u,n,i,t,e,d\right\}$
\end{solution}
\part $\left(E\cup G\right)'$
\begin{solution}[.5in]
$\left\{r,s\right\}$
\end{solution}
\end{parts}
\[\begin{xy}<1cm,0cm>:
(-2.5,2)*{E};
(2.5,2)*{F};
(-2.25,-2.5)*{G};
(-1,0)*\cir<2cm>{};
(1,0)*\cir<2cm>{};
(0,-1.732)*\cir<2cm>{};
(-1.5,1)*{u};
(-1.5,-1)*{i};
(0,1)*{n};
(0,-.5)*{t};
(0,-1)*{e};
(1.25,1)*{r};
(1.25,-1)*{d};
(2,1)*{s};
(-.5,-2.5)*{m};
(.5,-2.5)*{p};
\end{xy}\]
\end{multicols}
\ifprintanswers\else\newpage\fi

\question[24] Two dice are rolled and the
numbers showing on the top faces are observed.
Determine the probability of each of the following events.
\begin{parts}
\part Both numbers are the same
\begin{solution}[.5in]
Can occur in $6$ ways,
so $\frac{6}{36}=\frac{1}{6}$ the probability
\end{solution}
\part The sum of the two numbers is $5$
\begin{solution}[.5in]
Can occur in $4$ ways,
so $\frac{4}{36}=\frac{1}{9}$ the probability
\end{solution}
\part The sum of the two numbers is even
\begin{solution}[.5in]
Can occur in $18$ ways,
so $\frac{18}{36}=\frac{1}{2}$ the probability
\end{solution}
\part At least one of the numbers is $3$
\begin{solution}[.5in]
Can occur in $11$ ways, so $\frac{11}{36}$ the probability
\end{solution}
\end{parts}

\question[15]
A {\em deviant septet} is a roulette bet,
available only in the private casino located
in your instructor's basement, on any of the spaces labeled by
$0,00,1,2,3,4,5$. The payout for a $\$1$~deviant septet
is $\$4$.
Note that this is an {\bf exception} to the
payout rule $\textstyle\frac{36}{n}-1$.
\begin{parts}
\part Calculate the probability of winning a deviant septet.
\begin{solution}[.5in] $\frac{7}{38}$
\end{solution}
\part Calculate the {\bf odds against}
winning a deviant septet.
\begin{solution}[.5in]
Since $38-7=31$~spaces are losing, the odds against winning
are $31:7$.
\end{solution}
\part Calculate the expectation on a $\$1$~deviant septet.
Round your answer to the nearest cent.
\begin{solution}[1in]
$4\left(\frac{7}{38}\right)-1\left(\frac{31}{38}\right)=-\frac{3}{38}
\approx -\$0.08$
\end{solution}
\part How could the $\$4$~payout
be changed so that the deviant septet would be fair?
\begin{solution}[.5in]
Change the payout to $\frac{31}{7}\approx\$ 4.43$
\end{solution}
\end{parts}
\ifprintanswers\else\newpage\fi

\question[16] Your itinerary for Saturday should
include a trip to Target, Staples, Hy-Vee, and the bank.
However, since Target is across the street from Staples,
you should visit those two stores consecutively. In other
words, you should visit Target immediately after Staples, or else
Staples immediately after Target. In how many ways
can you plan your itinerary?
\begin{solution}[4in]12\end{solution}

\question[15] Suppose that $E$ and $F$ are events
with $P\left(E\right)=0.65$ and $P\left(F\right)=0.25$.
Do not round your answers to the following questions.
\begin{parts}
\part Calculate $P\left(\text{$E$ or $F$}\right)$
assuming that $E$ and $F$ are mutually exclusive.
\begin{solution}[.75in]
$0.65+0.25=0.9$
\end{solution}
\part Calculate $P\left(\text{$E$ and $F$}\right)$
assuming $E$ and $F$ are independent.
\begin{solution}[.75in]
$0.65\cdot 0.25=0.1625$
\end{solution}
\part Calculate $P\left(\text{$E$ or $F$}\right)$
assuming $E$ and $F$ are independent.
\begin{solution}[.75in]
$0.65+0.25-0.1625=0.7375$
\end{solution}
\end{parts}
\ifprintanswers\else\newpage\fi

\question[18] Ames High School has $1422$ students.
$102$ students are in band but not chorus while
$176$ students are in chorus but not band.
If $1093$ students are in neither band nor chorus,
then how many students are in both band and chorus?
\begin{solution}[4in]
\[\begin{xy}<.5cm,0cm>:
(-1,2)*+!D{\text{B}};
(1,2)*+!D{\text{C}};
(-1,0)*\cir<1cm>{};
(1,0)*\cir<1cm>{};
(0,0)*{51};
(-2,0)*{102};
(2,0)*{176};
(4,0)*{1093};
\end{xy}\]
\end{solution}

\end{questions}

\vfill\ifprintanswers\else
\begin{center}\gradetable[h][questions]\end{center}\fi

\end{document}
