\documentclass[handout]{beamer}
\usepackage{multicol}
\usepackage{xy}
\everymath{\displaystyle}
\mode<presentation>
{\usetheme{Warsaw}\setbeamercovered{dynamic}}
\usecolortheme{crane}
\usepackage{beamerfoils}
\pgfdeclareimage[height=1in]{university-logo}{ISULogo}
\logo{\pgfuseimage{university-logo}}
\setbeamertemplate{navigation symbols}{}
\title[\S9]{Section 9\\Combinations}
\author{Dr Marcus Bishop}
\subject{Math 104}
\beamerdefaultoverlayspecification{<+->}
\theoremstyle{definition}
\newtheorem{remark}{Remark}
\newtheorem{impact}{Impact}
\newtheorem{notation}{Notation}
\newtheorem{question}{Question}
\usepackage{arev}
\usepackage{tensor}
\newcommand\npr[2]{\tensor[_{#1}]P{_{#2}}}
\newcommand\ncr[2]{\tensor[_{#1}]C{_{#2}}}
\usepackage{cancel}
\begin{document}
\begin{frame}\titlepage\end{frame}
\LogoOff

\begin{frame}{Factorial}
\begin{itemize}
\item Denote $n\cdot\left(n-1\right)\cdots 3\cdot 2\cdot 1$
by $n!$ for any whole number $n$
\item By convention $0!=1$
\item Useful since factorial often appears in denominator
\begin{example}
\begin{itemize}
\item Recall that $\npr{n}{r}=\frac{n!}{\left(n-r\right)!}$
\item $\npr{n}{r}$ gives number of permutations of $n$
objects taken $r$ at a time
\item So $\npr{5}{5}
\only<+->{=\frac{5!}{\left(5-5\right)!}}
\only<+->{=\frac{5!}{0!}}
\only<+->{=5!}$
\end{itemize}
\end{example}
\end{itemize}
\end{frame}

\begin{frame}{Questions}
\begin{question}
How many finishes among first $3$~places
can occur in $8$-horse race (excluding ties)?
\end{question}
\begin{question}
How many different foursomes can be formed
from among $7$~golfers?
\end{question}
\begin{question}
How many different $13$-card bridge hands are there?
\end{question}
\begin{question}
How many different ways to arrange a particular $13$-card
bridge hand?
\end{question}
\end{frame}

\begin{frame}
\begin{question}
How many finishes among first $3$~places
can occur in $8$-horse race (excluding ties)?
\end{question}
\begin{itemize}
\item $8$~choices for first place
\item $7$~choices for second place
\item $6$~choices for third place
\item So $8\cdot 7\cdot 6=336$ possible finishes
\item Note that $8\cdot 7\cdot 6=\npr{8}{3}$
\item Indeed, problem asks for number of permutations
of $8$~objects taken $3$ at a time
\item \dots dressed up with intriguing story about horse racing
\end{itemize}
\end{frame}

\begin{frame}
\begin{question}
How many different foursomes can be formed
from among $7$~golfers?
\end{question}
\begin{itemize}
\item Members of foursome meant to sign names on scorecard
\item To make problem more concrete,
count number of ways foursome can fill out scorecard
\item $7$~choices for first name
\item $6$~choices for second, $5$ for third, $4$ for fourth
\item Thus $7\cdot 6\cdot 5\cdot 4
\only<+->{=\npr{7}{4}}$\only<+->{$=840$ ways for
players to fill out scorecard}
\end{itemize}
\end{frame}
\begin{frame}
\begin{itemize}
\item However, suppose same foursome signed scorecard in different order
\item Then foursome would be counted as \alert{different foursome}
\item Thus $840$ includes considerable duplication
\item In how many ways could particular foursome fill out scorecard?
\item $4$~choices for first name, $3$ for second, etc
\item So $4!\only<+->{=\npr{4}{4}}$
\only<+->{$=24$ ways for particular foursome to fill out scorecard}
\item So $840$ counts each possible foursome $24$~times 
\item Thus $\frac{840}{24}=35$ possible foursomes
\end{itemize}
\end{frame}

\begin{frame}
\begin{question}
How many different ways to arrange a particular $13$-card
bridge hand?
\end{question}
\begin{itemize}
\item $13!\only<+->{=\npr{13}{13}}$
\only<+->{$=6,227,020,800$}
\begin{question}
How many different $13$-card bridge hands are there?
\end{question}
\item Imagine hand arranged from left to right in player's hand
\item $52$~choices for first card, $51$ for second, etc
\item So $52\cdot 51\cdots 40=\npr{52}{13}$ arrangements
\item However, every distinct hand counted $13!$~times (see question above)
\item Thus $\frac{\npr{52}{13}}{13!}
=635,013,559,600$ possible hands
\end{itemize}
\end{frame}

\begin{frame}{Remarks}
\begin{itemize}
\item All questions above involve selection \alert{without replacement}
\item So after each selection, number of choices decreases by one
\item Accounts for appearance of consecutive numbers decreasing
by one, as in $8\cdot 7\cdot 6$
\item If replacement had been possible, numbers would remain same
instead of decrease
\item For example, spinning American roulette wheel three times
has $38\cdot 38\cdot 38$ possible outcomes
\end{itemize}
\end{frame}
\begin{frame}
\begin{itemize}
\item In horse and bridge hand arrangement examples, order of selections
significant
\item In golf and bridge hand examples, order of choices irrelevant
\item Determining whether order matters requires careful reading of question
and realistic thought about problem
\item Note that word \alert{arrange} in bridge hand arrangement
question a \alert{cue}
\item But no such cue present in horse question
\end{itemize}
\end{frame}

\begin{frame}{Combinations}
\begin{itemize}
\item Want to count number
of ways to select $r$~objects from set of $n$~objects
without replacement, with order irrelevant
\item $\npr{n}{r}=\frac{n!}{\left(n-r\right)!}$
accomplishes selection of $r$~objects from set of $n$
without replacement
\item But $\npr{n}{r}$ multiply counts
every distinct selection
\item Each selection of $r$~objects can be arranged
in $\npr{r}{r}=r!$ ways.
\item Therefore number of selections
given by \[\frac{\npr{n}{r}}{r!}
\only<+->{=\frac{n!}{\left(n-r\right)!r!}}\]
\item
$\frac{n!}{\left(n-r\right)!r!}$ denoted by $\ncr{n}{r}$
\item $\ncr{n}{r}$ called a \alert{binomial coefficient}
and often pronounced \alert{$n$ choose $r$}
\end{itemize}
\end{frame}

\begin{frame}{Examples}
\begin{question}
How many subsets of $\left\{a,b,c,d,e,f\right\}$ of size two?
\end{question}
\only<+->{
\[\ncr{6}{2}
\only<+->{=\frac{6!}{4!2!}}
\only<+->{=\frac{6\cdot 5\cdot 4\cdot 3\cdot 2\cdot 1}
{4\cdot 3\cdot 2\cdot 1\cdot 2\cdot 1}}
\only<+->{=\frac{6\cdot 5\cdot \cancel{4}
\cdot\cancel{3}\cdot\cancel{2}\cdot\cancel{1}}
{\cancel{4}\cdot\cancel{3}\cdot\cancel{2}\cdot\cancel{1}\cdot 2\cdot\cancel{1}}}
\only<+->{=\frac{6\cdot 5}{2}}
\only<+->{=15}\]}
\begin{remark}
As in example above, massive cancellation often occurs in binomial coefficients
\end{remark}
\begin{question}
How many two-topping pizzas possible from Joe's Pizza, which
offers six different toppings?
\end{question}
\only<+->{\[\ncr{6}{2}=15\]}
\end{frame}

\end{document}
