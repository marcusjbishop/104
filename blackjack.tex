\documentclass[handout,xcolor=dvipsnames]{beamer}
\usepackage{multicol}
\usepackage{xy}
\everymath{\displaystyle}
\mode<presentation>
{\usetheme{Warsaw}\setbeamercovered{dynamic}}
\usecolortheme{crane}
\usepackage{beamerfoils}
\pgfdeclareimage[height=1in]{university-logo}{ISULogo}
\logo{\pgfuseimage{university-logo}}
\setbeamertemplate{navigation symbols}{}
\title[Blackjack]{Blackjack}
\author{Dr Marcus Bishop}
\subject{Math 104}
\beamerdefaultoverlayspecification{<+->}
\theoremstyle{definition}
\newtheorem{remark}{Remark}
\newtheorem{impact}{Impact}
\newtheorem{situation}{Situation}
\newtheorem{question}{Question}
\usepackage{arev}
\usepackage{tensor}
\newcommand\npr[2]{\tensor[_{#1}]P{_{#2}}}
\newcommand\ncr[2]{\tensor[_{#1}]C{_{#2}}}
\usepackage{cancel}
\newcommand{\hs}{\alert{\varheart}}
\newcommand{\ds}{\alert{\vardiamond}}
\newcommand{\s}{\spadesuit}
\newcommand{\cs}{\clubsuit}
\begin{document}
\begin{frame}\titlepage\end{frame}
\LogoOff

\begin{frame}{Blackjack}
\begin{itemize}
\item We consider casino version
\item Also exists friendly version
\item In casino version player play against dealer, not against one another
\item Each player makes bet
\item Dealer deals each player two face-up cards
\item She also deals herself one face-up, one face-down
\item $J,K,Q$ worth $10$~points
\item $A$ worth either $1$ or $11$~points, whichever convenient
\item Remaining cards worth value shown on card
\item Players can request further cards, as many as desired
\item Accomplished by saying \alert{hit me}
\item Sum of player's cards should be as close to $21$ as possible,
but not $>21$
\item If sum $>21$ player loses bet (a \alert{bust})
\end{itemize}
\end{frame}

\begin{frame}
\begin{itemize}
\item When remaining players (if any) happy with their sums,
dealer hits herself as needed
\item Dealer \alert{must} hit herself if her sum $\le 16$
\item However, she must stop hitting herself when her sum $\ge 17$
\item If her sum $>21$ (she \alert{busts}), player wins amount bet
\item Suppose her sum $\le 21$
\item If her sum closer to $21$ than player's sum,
player loses bet
\item If player's sum closer to $21$ than dealer's sum,
player wins amount bet
\item \dots unless player's sum $21$ (a \alert{blackjack}
\item Then player wins $\frac{3}{2}$ amount bet
\item If dealer's sum equals player's sum, no money exchanged
\end{itemize}
\end{frame}

\begin{frame}{Why blackjack favors casino}
\begin{itemize}
\item Game sounds perfectly symmetric: dealer as likely
as player to arrive at sum closest to $21$
\item Why does game favor casino?
\item If player busts, player loses automatically
\item Dealer doesn't need to hit herself if all players bust
\item If she does hit and busts, players who bust earlier still lose
\end{itemize}
\end{frame}

\begin{frame}{Probability of blackjack}
\begin{itemize}
\item Two cards whose sum $21$ called a \alert{blackjack}
\item If player receives blackjack on first two cards,
wins unless dealer also has blackjack
\item Calculate probability of receiving blackjack on first two cards
\item Can only happen if one card is $A$ and other
has value $10$
\item So other card must be $10$, $J$, $Q$, or $K$
\item Four choices for $A$
\item $16$~choices for other card
\item So $4\cdot 16=64$ blackjacks possible
\item $\ncr{52}{2}=1326$ possible two-card hands
\item So $\frac{64}{1326}\approx 0.0483$ the probability of blackjack
\item We'll round answer to $\frac{1}{20}$
\end{itemize}
\end{frame}

\begin{frame}{Remarks}
\begin{itemize}
\item All our blackjack calculations only approximate
\item One issue: dealer often uses several decks of cards, rather than one
\item So $\approx\frac{1}{52}$ the
probability of receiving any particular card
\item \dots even though card might already appear on table
\item $\frac{1}{13}$ the probability of receiving any particular
\item Casino's use of several decks probably intended to foil
intentions of \alert{card counters} (discussed later)
\item Blackjack can be played virtually at
\href{http://wizardofodds.com/play/blackjack}{\color{blue}\tt wizardofodds}
\end{itemize}
\end{frame}

\begin{frame}{Probability of winning with blackjack}
\begin{itemize}
\item If player receives blackjack
%(with probability $\approx\frac{1}{20}$
what is probability he wins?
\item Player fails to win only if dealer also has blackjack
\item Dealer has blackjack with probability $\textstyle\frac{1}{20}$
\item So player wins with probability $\textstyle\frac{19}{20}$
\item So $\frac{1}{20}\cdot\frac{19}{20}
\approx\frac{1}{20}$ the probability of getting blackjack and winning
\end{itemize}
\begin{remark}
\begin{itemize}
\item Player receiving blackjack and dealer not receiving blackjack
not independent
\item So strictly speaking 
$\frac{1}{20}\cdot\frac{19}{20}$ incorrect
\item However, $\textstyle\frac{1}{20}$
only an approximation
\end{itemize}
\end{remark}
\end{frame}

\begin{frame}{Probability of busting if hit}
\begin{itemize}
\item Most important question: should player take another card?
\item Want to calculate probability that player busts
if hit
\item Depends on cards player has
\item Will assume several decks being used
\end{itemize}
\begin{example}
\begin{itemize}
\item Suppose player has $J\hs,5\cs$
\item Ranks that \alert{won't} bust player:
$A,2,3,4,5,6$
\item $\frac{6}{13}$ the probability player won't bust if hit
\item So $\frac{7}{13}$ the probability player busts if hit
\end{itemize}
\end{example}
\end{frame}

\end{document}
