\documentclass[handout]{beamer}
\everymath{\displaystyle}
\mode<presentation>
{\usetheme{Warsaw}\setbeamercovered{dynamic}}
\usecolortheme{crane}
\usepackage{beamerfoils}
\pgfdeclareimage[height=1in]{university-logo}{ISULogo}
\logo{\pgfuseimage{university-logo}}
\setbeamertemplate{navigation symbols}{}
\title[\S2]{Section 2\\Theoretical probability}
\author{Dr Marcus Bishop}
\subject{Math 104}
\beamerdefaultoverlayspecification{<+->}
\theoremstyle{definition}
\newtheorem{remark}{Remark}
\newtheorem{impact}{Impact}
\newtheorem{notation}{Notation}
\begin{document}
\begin{frame}\titlepage\end{frame}
\LogoOff

\begin{frame}{Equally likely outcomes}
\begin{definition} If each outcome of experiment
as likely to occur as any other, outcomes called
\alert{equally likely}
\end{definition}
\begin{example}
Outcomes $1,2,3,4,5,6$ of rolling die equally likely
\end{example}
\begin{example}
Outcomes H,T of flipping coin equally likely
\end{example}
\begin{example}
\begin{itemize}
\item Experiment: roll two dice and add results
\item Outcomes 2,3,\ldots,12 \alert{not} equally likely!
\end{itemize}
\end{example}
\end{frame}

\begin{frame}{Example}
\[\begin{array}{cccccc}
\left(1,1\right)&\quad
\left(1,2\right)&\quad
\left(1,3\right)&\quad
\left(1,4\right)&\quad
\left(1,5\right)&\quad
\left(1,6\right)
\end{array}\]
\end{frame}
\end{document}
