\documentclass[12pt]{article}
\usepackage[colorlinks=true, linkcolor=blue, breaklinks=true]{hyperref} 
\usepackage{graphicx}
\usepackage[euler-digits]{eulervm}
\usepackage{charter,amsmath,amssymb,breakurl}
\usepackage[letterpaper,margin=1in]{geometry}
\usepackage{multicol}
\everymath{\displaystyle}
\author{}\date{}
\title{Solutions to Review Problems for Exam 1}\author{}
\begin{document}
\maketitle
\pagestyle{empty}
\begin{enumerate}
\setcounter{enumi}{1}
\item\begin{enumerate}\item From the Punnett square
$\begin{array}{c|cc}
&\mathbold{a}&\mathbold{a}\\\hline
\mathbold{A}&\mathbold{Aa}&\mathbold{Aa}\\
\mathbold{A}&\mathbold{Aa}&\mathbold{Aa}\end{array}$
we see that {\em all} possible offspring have
genotype $\mathbold{Aa}$, so $0$ is the probability
of an offspring having albinism.
\item $\mathbold{Aa}$ is the only possibility.
\item From the Punnett square
$\begin{array}{c|cc}
&\mathbold{a}&\mathbold{A}\\\hline
\mathbold{A}&\mathbold{Aa}&\mathbold{AA}\\
\mathbold{A}&\mathbold{Aa}&\mathbold{AA}\end{array}$
we see that the possible genotypes are
$\mathbold{Aa}$ and $\mathbold{AA}$.
\item Two of the four possible outcomes have genotype
$\mathbold{Aa}$ and none have
$\mathbold{aA}$. So $\frac{2}{4}=\frac{1}{2}$ is the probability
that an offspring will be a carrier.
\end{enumerate}
\item\begin{enumerate}
\item $\frac{1}{8}$ is the area of sector~3, so $\frac{1}{8}$
is the probability of the spinner stopping in~3.
\item $\frac{5/8}{3/8}=\frac{5}{3}$ so $5:3$ are the odds
against the spinner stopping in~4.
\item $\frac{1/4}{3/4}=\frac{1}{3}$ so $1:3$ are the odds
in favor of the spinner stopping in~2.
\end{enumerate}
\item\begin{enumerate}\item The {\em net} proceeds
are $4-2=2$ in the case that the player wins, which occurs
with probability $\frac{3}{8}$. Similarly
the {\em net} proceeds are $-2$ in the case that the player loses, which occurs
with probability $\frac{5}{8}$. Thus the expectation is
\[2\left(\frac{3}{8}\right)-2\left(\frac{5}{8}\right)=-\frac{1}{2}.\]
\item One way to do this is to change the {\em net} proceeds
to $5$ and $-3$. This causes the expectation to be
\[5\left(\frac{3}{8}\right)-3\left(\frac{5}{8}\right)=0.\]
Now since $-3$ is the {\em net} proceeds in the case that the player loses,
this means that the player must have paid $\$3$ to play.
Similarly, since $5$ is the {\em net} proceeds in the case that the player 
wins, this means that the prize must have been $\$8$.
In summary, one way to make the game fair is to change the ticket price
to $\$3$ and the prize to $\$8$.
\end{enumerate}
\end{enumerate}
\end{document}
