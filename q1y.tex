\documentclass[answers,12pt]{exam}
\everymath{\displaystyle}
\usepackage{amsmath,charter}
\title{Math 104 Quiz 1\\Yellow Form}
\begin{document}
\maketitle
\begin{center}
\fbox{\fbox{\parbox{5.5in}{\centering
Using a Number~2 pencil record your student ID on your answer sheet.
Then answer the questions below on the answer sheet.
}}}
\end{center}

\begin{questions}
\question Math~104 currently has 122~women and 93~men enrolled.
If a student is selected at random, what is the probability
that the student will be a woman?\\
\begin{oneparchoices}
\choice $\frac{93}{215}$
\choice $\frac{122}{215}$
\choice $\frac{93}{122}$
\choice $\frac{93}{104}$
\choice $\frac{122}{93}$
\end{oneparchoices}

\question If an event has probability $0.65$
what is the probability that the event will {\em not} occur?\\
\begin{oneparchoices}
\choice $0.0$
\choice $0.35$
\choice $0.65$
\choice $0.7$
\choice $1.0$
\end{oneparchoices}

\question If a card is selected at random from a deck of
playing cards, what is the probability that
the card will be a King?\\
\begin{oneparchoices}
\choice $\frac{1}{52}$
\choice $\frac{1}{13}$
\choice $\frac{1}{4}$
\choice $\frac{3}{4}$
\choice $\frac{12}{13}$
\end{oneparchoices}

\question Probability computed by studying
the possible outcomes of an experiment is called \dots\\
\begin{oneparchoices}
\choice computational.
\choice conditional.
\choice empirical.
\choice relative.
\choice theoretical.
\end{oneparchoices}

\question An event which cannot occur has probability \dots\\
\begin{oneparchoices}
\choice $0.0$.
\choice $0.1$.
\choice $0.25$.
\choice $0.5$.
\choice $1.0$.
\end{oneparchoices}

\newpage
\question Consider the experiment consisting
of rolling two dice and adding the numbers
appearing on the top faces of the dice.
The experiment is repeated $200$ times and the
results shown in the table below. Namely, 
for each possible outcome, the table shows the number
of times the outcome occurred directly below the outcome.
\[\begin{array}{r|ccccccccccc}
\text{outcome}&2&3&4&5&6&7&8&9&10&11&12\\\hline
\text{frequency}&6&16&22&20&27&27&25&25&9&19&4
\end{array}\]
Based on these results, what is the {\em empirical} probability
that the sum will be two if the experiment is performed again?\\
\begin{oneparchoices}
\choice $\frac{2}{200}$
\choice $\frac{1}{36}$
\choice $\frac{3}{100}$
\choice $\frac{1}{6}$
\choice $\frac{1}{2}$
\end{oneparchoices}

\question If an event $E$ has probability $0.45$
calculate $P\left(E\right)+P\left(\text{not $E$}\right)$.\\
\begin{oneparchoices}
\choice $0.0$
\choice $0.45$
\choice $0.5$
\choice $0.55$
\choice $1.0$
\end{oneparchoices}

\question If two parents have genotypes $cc$ and $cC$
with respect to the gene for Cystic Fibrosis, what
is the probability that one of their children will be
a carrier of the Cystic Fibrosis gene?\\
\begin{oneparchoices}
\choice $0$
\choice $\frac{1}{4}$
\choice $\frac{1}{3}$
\choice $\frac{1}{2}$
\choice $\frac{3}{4}$
\end{oneparchoices}

\question Probability calculated by relative frequency is called \dots\\
\begin{oneparchoices}
\choice conditional.
\choice empirical.
\choice random.
\choice relative.
\choice theoretical.
\end{oneparchoices}

\question An event which is as likely to occur as to not occur has probability \dots\\
\begin{oneparchoices}
\choice $0.0$.
\choice $0.1$.
\choice $0.25$.
\choice $0.5$.
\choice $1.0$.
\end{oneparchoices}

\end{questions}
\end{document}
