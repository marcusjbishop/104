\documentclass[12pt]{article}
\usepackage[colorlinks=true, linkcolor=blue, breaklinks=true]{hyperref} 
\usepackage{graphicx}
\usepackage[euler-digits]{eulervm}
\usepackage{charter,amsmath,amssymb,breakurl}
\usepackage[letterpaper,margin=1in]{geometry}
\usepackage{multicol}
\everymath{\displaystyle}
\author{}\date{Due in class Friday 30 January}
\title{Math 104 Written Assignment 1}\author{}
\begin{document}
\maketitle
\pagestyle{empty}
\begin{enumerate}
\item Consider the experiment of flipping three coins
and recording the number of heads facing up.
\begin{enumerate}
\item What are all the possible outcomes of the experiment?
\vspace{1cm}
\item\label{ThreeCoins} Repeat the experiment at least 30~times,
either with real coins or with a coin simulator, such
as
\href{https://www.random.org/coins/?num=3&cur=60-usd.0025c-nj}{\tt www.random.org/coins}
and record the results in the space below.
\vspace{3cm}
\item Use your response to (\ref{ThreeCoins}) to calculate
the empirical probability of flipping exactly two heads.
\vspace{1cm}
\item Calculate the theoretical probability
of flipping exactly two heads.
\vspace{1cm}
\end{enumerate}

\item Consider the experiment of rolling three dice
and recording the sum of the numbers showing on the top faces
of the dice.
\begin{enumerate}
\item What are all the possible outcomes of the experiment?
\vspace{1cm}
\item\label{ThreeDice} Repeat the experiment at least 30 times,
either with real dice or with a dice simulator, such
as
\href{https://www.random.org/dice/?num=3}{\tt www.random.org/dice}
and record the results in the space below.
\vspace{3cm}
\item\label{Sum10} Use your response to (\ref{ThreeDice}) to calculate
the empirical probability of rolling a sum of 10.
\vspace{1cm}
\item\label{Sum16} Use your response to (\ref{ThreeDice}) to calculate
the empirical probability of rolling a sum of 16.
\vspace{1cm}
\item Comment on the accuracy of your responses to
(\ref{Sum10}) and (\ref{Sum16}).
\vspace{1cm}
\end{enumerate}
\end{enumerate}
\end{document}
