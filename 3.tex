\documentclass[handout]{beamer}
\usepackage{multicol}
\everymath{\displaystyle}
\mode<presentation>
{\usetheme{Warsaw}\setbeamercovered{dynamic}}
\usecolortheme{crane}
\usepackage{beamerfoils}
\pgfdeclareimage[height=1in]{university-logo}{ISULogo}
\logo{\pgfuseimage{university-logo}}
\setbeamertemplate{navigation symbols}{}
\title[\S3]{Section 3\\Odds}
\author{Dr Marcus Bishop}
\subject{Math 104}
\beamerdefaultoverlayspecification{<+->}
\theoremstyle{definition}
\newtheorem{remark}{Remark}
\newtheorem{impact}{Impact}
\newtheorem{notation}{Notation}
\usepackage{arev}
\begin{document}
\begin{frame}\titlepage\end{frame}
\LogoOff

\begin{frame}{Introduction}
\begin{itemize}
\item \alert{Odds} express same information as probability
\item Odds more common than probability in non-mathematical contexts
\end{itemize}
\begin{example}
\begin{itemize}
\item The odds of winning the Powerball are \alert{$1$ in $175,223,510$}
\item Keselowski, who had a career-best Atlanta finish of third-place
during his 2012 Championship season, is one of the \alert{$6$-to-$1$} favorites
\end{itemize}
\end{example}
\begin{remark}
\begin{itemize}
\item Odds consist of two numbers, seperated by preposition \alert{in} or \alert{to}
\item One number often $1$
\item Larger number can come either first or last
\item Hyphens in second example make easier to read
\end{itemize}
\end{remark}
\end{frame}

\end{document}
