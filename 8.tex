\documentclass[handout]{beamer}
\usepackage{multicol}
\usepackage{xy}
\everymath{\displaystyle}
\mode<presentation>
{\usetheme{Warsaw}\setbeamercovered{dynamic}}
\usecolortheme{crane}
\usepackage{beamerfoils}
\pgfdeclareimage[height=1in]{university-logo}{ISULogo}
\logo{\pgfuseimage{university-logo}}
\setbeamertemplate{navigation symbols}{}
\title[\S8]{Section 8\\The Counting Principle and permutations}
\author{Dr Marcus Bishop}
\subject{Math 104}
\beamerdefaultoverlayspecification{<+->}
\theoremstyle{definition}
\newtheorem{remark}{Remark}
\newtheorem{impact}{Impact}
\newtheorem{notation}{Notation}
\newtheorem{argument}{Argument}
\usepackage{arev}
\begin{document}
\begin{frame}\titlepage\end{frame}
\LogoOff

\begin{frame}
{What I do outside of class that helps me learn}
(Selected comments from course evaluations
paraphrased slightly to improve humor and style)
\begin{itemize}
\item Homework, study guide problems, past exams, slides,
YouTube videos
\item Math help room, Josh's office hours, Bishop's office hours,
private tutors
\item Study with friends, organize or attend study groups
\item Since the lectures are useless and demoralizing, I learn the
material by myself. Then I tutor my friends, who
find the instructor flippant and condescending.
This helps me to understand the material better because
it forces me to know it well enough to teach others.
\item I always travel with friends at night, because I'm afraid
Bishop will jump out from behind a tree and make me do Venn diagram
problems
\end{itemize}
\end{frame}

\begin{frame}{What I like about Math 104}
\begin{itemize}
\item Professor is funny, entertaining, energetic, upbeat
\item Frequency, easiness, amount of homework
\item I don't enjoy anything about Math 104
\item I enjoy learning about casino games
\item Math 104 lectures resemble
getting a root canal, except that it only lasts 50 minutes, hallelujah!
\item Material interesting, useful, applicable to real world
\item Class might feel like sitting in a sinking ship,
but at least my friends are all here with me. I love you guys!
\end{itemize}
\end{frame}

\begin{frame}{What I would change about Math 104}
\begin{itemize}
\item The instructor; I'd say my cat could teach the
course better, and he's actually looking for a job!
\item Instructor better prepared, organized, competent
\item More examples, explained slower, in greater depth
\item Homework submission, retrieval, exam distribution
\item Lectures too slow
\item Lectures too fast
\item Use Blackboard calender
\item Exams and quizzes harder than homework
\item Instructor reads slides aloud
\end{itemize}
\end{frame}

\begin{frame}{Permuations}
\begin{itemize}
\item Ways of arranging set of distinct objects called
\alert{permutations}, \alert{arrangements}, or \alert{anagrams}
\item Objects meant to be arranged linearly
\item Order matters in listing permutations
\begin{example} Six permutations of $\left\{1,2,3\right\}$:
\[123\quad 132\quad 213\quad 231\quad 312\quad 321\]
\end{example}
\item Permutations can be generated using tree diagram:
\[\begin{xy}<1cm,0cm>:
(0,0)="0";
(-3,-1)="-31";
(0,-1)="01";
(3,-1)="31";
"0";"-31"*+!D{1}**\dir{-};
"0";"01"*+!D{2}**\dir{-};
"0";"31"*+!D{3}**\dir{-};
"-31";(-4,-2)*+!D{2}**\dir{-};
(-4,-3)*+!D{3}**\dir{-};
"-31";(-2,-2)*+!D{3}**\dir{-};
(-2,-3)*+!D{2}**\dir{-};
"01";(-1,-2)*+!D{1}**\dir{-};
(-1,-3)*+!D{3}**\dir{-};
"01";(1,-2)*+!D{3}**\dir{-};
(1,-3)*+!D{1}**\dir{-};
"31";(2,-2)*+!D{1}**\dir{-};
(2,-3)*+!D{2}**\dir{-};
"31";(4,-2)*+!D{2}**\dir{-};
(4,-3)*+!D{1}**\dir{-};
\end{xy}\]
\end{itemize}
\end{frame}

\begin{frame}
\begin{itemize}
\item But drawing tree inefficient when only
\alert{number} of permutations needed
\item Observe that tree has $3$~branches,
each with $2$~branches, each with $1$~leaf
\item So $3\cdot 2\cdot 1=6$ leaves
\[\begin{xy}<1cm,0cm>:
(0,0)="0";
(-3,-1)="-31";
(0,-1)="01";
(3,-1)="31";
"0";"-31"*+!D{1}**\dir{-};
"0";"01"*+!D{2}**\dir{-};
"0";"31"*+!D{3}**\dir{-};
"-31";(-4,-2)*+!D{2}**\dir{-};
(-4,-3)*+!D{3}**\dir{-};
"-31";(-2,-2)*+!D{3}**\dir{-};
(-2,-3)*+!D{2}**\dir{-};
"01";(-1,-2)*+!D{1}**\dir{-};
(-1,-3)*+!D{3}**\dir{-};
"01";(1,-2)*+!D{3}**\dir{-};
(1,-3)*+!D{1}**\dir{-};
"31";(2,-2)*+!D{1}**\dir{-};
(2,-3)*+!D{2}**\dir{-};
"31";(4,-2)*+!D{2}**\dir{-};
(4,-3)*+!D{1}**\dir{-};
\end{xy}\]
\item Could use same mechanism to count permutations
of $\left\{1,2,3,4\right\}$ without drawing tree
\item Tree would have $4$~branches, each with $3$~branches,
each with $2$~branches, each with $1$~leaf
\item So tree would have $4\cdot 3\cdot 2\cdot 1=24$ leaves
\item So $24$ the number of permutations of four objects
\end{itemize}
\end{frame}

\end{document}
