\documentclass{ximera}
\usepackage{booktabs}
\title{\S1 The nature of probability}
\begin{abstract}
Introduction to probability, empirical probability, Law of large numbers
\end{abstract}
\begin{document}
\maketitle
\begin{enumerate}

\item{\bf Basics.}
\begin{enumerate}
\item Probability arose in 17th century
\item Measures likelyhood of future events
\end{enumerate}

\item{\bf Terminology.}
\begin{enumerate}
\item An {\em experiment} a controlled operation,
generally in future
\item Possible results of experiment called {\em outcomes}
\item An {\em event} a set of outcomes of an experiment
\item Events the basic object of study in probability
\end{enumerate}

\begin{example}
\begin{enumerate}
\item Rolling a die an experiment
\item Possible outcomes: 1, 2, 3, 4, 5, 6
\item Examples of events:
\begin{enumerate}
\item Die shows even number, or $E=\left\{2,4,6\right\}$
\item Die shows odd number, or $E=\left\{1,3,5\right\}$
\item Die shows 5, or $E=\left\{5\right\}$
\end{enumerate}
\end{enumerate}
\end{example}

\begin{example}
\begin{enumerate}
\item Rolling {\em two} dice an experiment
\item Possible outcomes: $\left(1,1\right),\left(1,2\right),\ldots$
\item Examples of events:
\begin{enumerate}
\item Both dice show even number, or $E=\left\{\left(2,2\right),
\left(2,4\right),\ldots\right\}$
\item One die shows even number and one shows odd number,
or $E=\left\{\left(1,2\right),\left(2,1\right),\ldots\right\}$
\end{enumerate}
\end{enumerate}
\end{example}

\begin{example}
More abstract events:
\begin{enumerate}
\item $E:$ An insurance customer will be in a car accident in the next year
\item $E:$ It rains tomorrow
\item $E:$ Coop will be out of kale when I arrive
\end{enumerate}
Not obvious what experiments are in these cases
\end{example}

\item{\bf Probability}
\begin{enumerate}
\item {\em Probability of event $E$}, denoted by $P\left(E\right)$,
a number between $0$ and $1$
\item $P\left(E\right)$ measures likelyhood that $E$ occurs
\item The greater $P\left(E\right)$, the more likely $E$ to occur
\item If $P\left(E\right)=0$ then $E$ will not occur
\item If $P\left(E\right)=1$ then $E$ will occur
\item If $P\left(E\right)=\frac{1}{2}$ then $E$ 
as likely to occur as not occur
\item Since $0\le P\left(E\right)\le 1$ can also express
$P\left(E\right)$ as percent
\item Common with weather: {\em $30\%$ chance of rain tomorrow}
\item $P\left(E\right)$ calculated either {\em theoretically}
or {\em empirically}
\item Theoretical probability the subject of \S2 and remainder of course
\end{enumerate}

\item{\bf Empirical probability}
\begin{enumerate}
\item To calculate $P\left(E\right)$ {\em empirically}
repeat experiment several times, recording how
many times $E$ occured
\item Each repitition of experiment called a {\em trial}
\item Then $P\left(E\right)=
\frac{\text{Number of times $E$ occured}}
{\text{Number of trials}}$
\item $\frac{\text{Number of times $E$ occured}}
{\text{Number of trials}}$ also called
{\em relative frequency of~$E$}
\end{enumerate}

\begin{question}
\begin{enumerate}
\item Experiment: flip coin and observe side facing up
\item Possible outcomes: heads (H), tails (T)
\item $E:$ event that coin shows heads
\item Repeat experiment ten times
\item Results: H,T,H,H,T,H,H,T,T,H
\item Calculate $P\left(E\right)$ empirically
\end{enumerate}
\begin{solution} Since H occurs $6$ times
and experiment repeated $10$ times
$P\left(E\right)=\answer{\frac{5}{10}=\frac{1}{2}}$
\end{solution}
\end{question}

\begin{question}
Table shows number of visitors to travel websites in October 2010
\begin{center}\begin{tabular}{cr}
Website&Visitors\\\toprule
Trip Advisor&24,000,000\\
Yahoo! Travel&23,250,000\\
Expedia&23,000,000\\
Travelocity&16,000,000\\
Priceline&15,000,000\\\midrule
Totals&101,250,000
\end{tabular}\end{center}
\begin{parts}
\item Calculate probability that next visitor
visits Expedia
\begin{solution}\answer{\frac{23,000,000}{101,250,000}}\end{solution}
\item Calculate probability that next visitor
visits Priceline 
\begin{solution}\answer{\frac{15,000,000}{101,250,000}}\end{solution}
\end{parts}
\end{question}

\begin{remark}
\begin{enumerate}
\item Empirical probability na\"ive
\item Results vary each time $P\left(E\right)$ calculated
\item Theoretical probability better, but not always possible to calculate
\item However, empirical probability improves
as number of trials increases
\end{enumerate}
\end{remark}

\begin{theorem}[Law of Large Numbers]
As number of trials increases, empirical probability
approaches theoretical probability
\end{theorem}

\begin{example}
\begin{enumerate}
\item Experiment: flip coin
\item $E:$ coin shows heads
\item Table gives $P\left(E\right)$ with different numbers of trails:
\end{enumerate}
\begin{center}\begin{tabular}{rrr}
Number trials&Number heads&Relative frequency\\\toprule
10&4&0.4\\
100&45&0.45\\
1000&546&0.546\\
10,000&4852&.4852\\
100,000&49,770&0.4977
\end{tabular}\end{center}

\end{enumerate}
\end{document}
