\documentclass{beamer}
\everymath{\displaystyle}
\mode<presentation>
{\usetheme{Warsaw}\setbeamercovered{dynamic}}
\usecolortheme{crane}
\usepackage{beamerfoils}
\pgfdeclareimage[height=1in]{university-logo}{ISULogo}
\logo{\pgfuseimage{university-logo}}
\setbeamertemplate{navigation symbols}{}
\title[\S1]{Section 1\\The nature of probability}
\author{Dr Marcus Bishop}
\subject{Math 104}
\beamerdefaultoverlayspecification{<+->}
\theoremstyle{definition}
\newtheorem{remark}{Remark}
\newtheorem{impact}{Impact}
\newtheorem{notation}{Notation}
\begin{document}
\begin{frame}\titlepage\end{frame}
\LogoOff

\begin{frame}{Terminology}
\begin{itemize}
\item Probability deals with \alert{experiments}
\item An \alert{experiment} is
\begin{itemize}
\item A controlled operation
\item Generally in future
\item Results of interest to us, but unknown until experiment
carried out
\item Has specific set of \alert{outcomes}, exactly
one of which will occur
\end{itemize}
\item An \alert{event} a set of outcomes of an experiment
\item Events the basic object of study in probability
\item Since outcome of experiment either in event or not,
event will either occur or not
\item Question arises: how likely to occur is event?
\item Aim of probability: assign number to event
reflecting its likelihood
\item Then people can use that number to help make decisions
\end{itemize}
\end{frame}

\begin{frame}{Examples}
\begin{example}
\begin{itemize}
\item Rolling a die an experiment
\item Possible outcomes: 1, 2, 3, 4, 5, 6
\item Examples of events:
\begin{itemize}
\item Die shows even number, or $E=\left\{2,4,6\right\}$
\item Die shows odd number, or $E=\left\{1,3,5\right\}$
\item Die shows 5, or $E=\left\{5\right\}$
\end{itemize}
\end{itemize}
\end{example}
\begin{example}
\begin{itemize}
\item Rolling \alert{two} dice an experiment
\item So possible outcomes: $\left(1,1\right),\left(1,2\right),\ldots$
\item Examples of events:
\begin{itemize}
\item Both dice show even number, or $E=\left\{\left(2,2\right),
\left(2,4\right),\ldots\right\}$
\item One die shows even number and one shows odd number,
or $E=\left\{\left(1,2\right),\left(2,1\right),\ldots\right\}$
\end{itemize}
\end{itemize}
\end{example}
\end{frame}

\begin{frame}
\begin{example}
\begin{itemize}
\item ISU men's basketball game against Baylor an experiment
\item Possible outcomes: ISU wins, Baylor wins
\end{itemize}
\end{example}
\begin{itemize}
\item Ultimately interested in more abstract events:
\begin{itemize}
\item $E:$ It rains tomorrow
\item $E:$ grocery store will be out of kale when I arrive
\item $E:$ An insurance customer will be in a car accident in the next year
\end{itemize}
\item Not obvious what experiments are in these cases
\item Math~104 focuses on events occurring in framework of experiments
\end{itemize}
\end{frame}

\begin{frame}{Probability}
\begin{itemize}
\item \alert{Probability of event $E$}, denoted by $P\left(E\right)$,
a number between $0$ and $1$ (inclusive)
\item $P\left(E\right)$ measures likelihood that $E$ will occur
\item The greater $P\left(E\right)$, the more likely $E$ to occur
\item If $P\left(E\right)=0$ then $E$ will not occur
\item If $P\left(E\right)=1$ then $E$ will occur
\item If $P\left(E\right)=\frac{1}{2}$ then $E$ 
as likely to occur as not occur
\item Since $0\le P\left(E\right)\le 1$ can also express
$P\left(E\right)$ as percent
\item Common with weather: \alert{$30\%$ chance of rain tomorrow}
\item $P\left(E\right)$ calculated either \alert{theoretically}
or \alert{empirically}
\item Theoretical probability the subject of \S2 and remainder of course
\item Involves systematically studying all possible outcomes of experiment
\end{itemize}
\end{frame}

\begin{frame}{Empirical probability}
\begin{itemize}
\item To calculate $P\left(E\right)$ \alert{empirically}
repeat experiment several times, recording how
many times $E$ occurred
\item Each repetition of experiment called a \alert{trial}
\item Then $P\left(E\right)=
\frac{\text{Number of times $E$ occurred}}
{\text{Number of trials}}$
\item $\frac{\text{Number of times $E$ occurred}}
{\text{Number of trials}}$ also called
\alert{relative frequency}
\item Measures the \alert{proportion} or \alert{percentage}
of times $E$ occurred
\end{itemize}
\begin{remark}
\begin{itemize}
\item Neither numerator or denominator alone meaningful
\item Idea of relative frequency: consolidate two numbers into one
\end{itemize}
\end{remark}
\end{frame}

\begin{frame}{Example}
\begin{itemize}
\item Experiment: flip coin and observe side facing up
\item Possible outcomes: heads, tails
\item $E:$ event that coin shows heads
\item Repeat experiment ten times
\item Results: H,T,H,H,T,H,H,T,T,H
\item Want to calculate $P\left(E\right)$ empirically
\item Since H occurs $6$ times
and experiment repeated $10$ times
$P\left(E\right)=\frac{6}{10}=\frac{3}{5}$
\end{itemize}
\end{frame}

\begin{frame}{Exercise 16}
Table shows number of visitors to travel websites in October~2010
\begin{center}\begin{tabular}{cr}
Website&Visitors\\\hline
Trip Advisor&24,000,000\\
Yahoo! Travel&23,250,000\\
Expedia&23,000,000\\
Travelocity&16,000,000\\
Priceline&15,000,000\\\hline
Totals&101,250,000
\end{tabular}\end{center}
\begin{itemize}
\item Calculate probability that next visitor visits Expedia
\item $\frac{23,000,000}{101,250,000}\approx 0.227$
\item Calculate probability that next visitor visits Priceline 
\item $\frac{15,000,000}{101,250,000}\approx 0.148$
\end{itemize}
\end{frame}

\begin{frame}{Remarks}
\begin{itemize}
\item Benefit of empirical probability: very simple
\item Drawback: results vary each time $P\left(E\right)$ calculated
\item Rationale in travel example:
\begin{itemize}
\item The travelers visiting websites in October 2010 a \alert{sample}
of all travelers, essentially \alert{randomly} selected
\item We assume proportion of travelers preferring Travelocity
in sample same as proportion among all travelers
\item Assumption usually reasonable
\end{itemize}
\item Theoretical probability more precice,
but not always possible to calculate
\end{itemize}
\begin{theorem}[Law of Large Numbers]
As number of trials increases, empirical probability
approaches theoretical probability
\end{theorem}
\end{frame}

\begin{frame}{Illustration}
\begin{itemize}
\item Experiment: flip coin
\item $E:$ coin shows heads
\item Table gives $P\left(E\right)$ with different numbers of trails:
\begin{center}\begin{tabular}{rrr}
Number trials&Number heads&$P\left(E\right)$\\\hline
10&4&0.4\\
100&45&0.45\\
1000&546&0.546\\
10,000&4852&.4852\\
100,000&49,770&0.4977
\end{tabular}\end{center}
\item Note that relative frequencies approach $0.5$
\item Indeed, $0.5$ the theoretical probability of $E$
(see \S2)
\end{itemize}
\end{frame}

\begin{frame}{Application: Mendelian Genetics}
\begin{itemize}
\item Mendel (1822--1884) crossed pea plants 
\item Certain plants observed to be \alert{pure}:
\begin{itemize}
\item When pure plants with same traits crossed,
offspring also have same traits
\end{itemize}
\item Mendel crossed pure plants producing yellow peas
with pure plants producing green peas
\item Offspring (\alert{first generation}) all had yellow peas
\item However, first generation plants no longer pure
\item Indeed, crossing first generation plants resulted in 
6022 yellow and 2001 green (\alert{second generation})
\item So 
$\frac{6022}{6022+2001}=\frac{6022}{8023}\approx 0.75$
the probability that randomly chosen second generation plant yellow

\end{itemize}
\end{frame}

\begin{frame}
\begin{itemize}
\item Similarly Mendel crossed pure plants
producing round pea plants with pure plants producing wrinkled peas
\item Offspring all round, but not pure
\item However, in second generation 5474 round and 1850 wrinkled
\item So probability that randomly chosen second generation plant round
$\frac{5474}{7324}\approx 0.75$
\end{itemize}
\end{frame}

\begin{frame}
\begin{itemize}
\item Appearance of $0.75$ twice can't be coincidence!
\item Mendel's results led scientists to discover mechanism
\item Namely, each plant has two \alert{genes} controlling color
\item One gene inherited from mother, one from father
\item If both genes green, then plant green
\item However, if one or both genes yellow, then plant yellow
\item Explains possibility of green offspring of yellow plants
\item Plants identified as \alert{pure} have two green or two yellow genes
\end{itemize}
\end{frame}

\end{document}
