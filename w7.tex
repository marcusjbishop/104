\documentclass[12pt]{article}
\usepackage{eulervm}
\usepackage{charter,amsmath,amssymb,breakurl}
\usepackage[letterpaper,margin=.8in]{geometry}
\usepackage{multicol}
\everymath{\displaystyle}
\author{}\date{Due in class Friday 27 March}
\title{Math 104 Worksheet 7}\author{}
\begin{document}
\maketitle
\thispagestyle{empty}
\begin{enumerate}
\item Calculate the value of each of the following
initial farkle rolls.
\begin{enumerate}
\item $1,1,2,3,5,6$
\item $1,1,1,2,2,3$
\item $2,2,3,3,5,5$
\item $2,2,2,2,3,5$
\item $1,2,3,4,5,6$
\end{enumerate}

\item Suppose you are playing farkle with your
friends. You initially roll $1,2,2,2,5,6$.
\begin{enumerate}
\item According to the rules, can you end your turn now?
Explain why or why not.
\vspace{1cm}
\item One possible choice is to reroll the die showing 6.
Calculate your expected winnings for this turn in
this situation.
This means that you should list all the possible {\em net}
outcomes, multiply each outcome by its probability,
and add the results.
\vspace{3cm}
\item Another possible choice is to reroll
the dice showing $5,6$.
Calculate your expected winnings for this turn in
this situation.
\vspace{3cm}
\item Using your results above, decide which
choice you should make.
\end{enumerate}
\end{enumerate}
\end{document}
