\documentclass{beamer}
\usepackage{multicol}
\usepackage{xy}
\everymath{\displaystyle}
\mode<presentation>
{\usetheme{Warsaw}\setbeamercovered{dynamic}}
\usecolortheme{crane}
\usepackage{beamerfoils}
\pgfdeclareimage[height=1in]{university-logo}{ISULogo}
\logo{\pgfuseimage{university-logo}}
\setbeamertemplate{navigation symbols}{}
\title[Farkle]{Farkle}
\author{Dr Marcus Bishop}
\subject{Math 104}
\beamerdefaultoverlayspecification{<+->}
\theoremstyle{definition}
\newtheorem{remark}{Remark}
\newtheorem{impact}{Impact}
\newtheorem{notation}{Notation}
\newtheorem{argument}{Argument}
\begin{document}
\begin{frame}\titlepage\end{frame}
\LogoOff

\begin{frame}{Facebook rules}
\begin{itemize}
%\item Players do the following in turn
\item Player rolls six dice
\item Results of roll scored according to next slide
\item If roll worth zero points (player \alert{farkles})
then player's turn over
\item Otherwise, dice not contributing points 
(and some but not all dice contributing points) \alert{may}
be rerolled
\item Player \alert{must} reroll dice
unless roll worth $\ge 300$ points
\begin{remark}
This rule not present in street version of farkle
\end{remark}
\item Results of second roll scored according to next slide
\item If second roll worth zero points
(player \alert{farkles}) then turn over and
player loses all points arising in first roll
\item Player continues rerolling dice not contributing
points until she farkles or accumulates $\ge 300$ points and
chooses to end turn, banking points accumulated
\item Player gets another turn if all six
dice contribute points
\end{itemize}
\end{frame}

\begin{frame}{Scoring first roll}
\begin{itemize}
\item Combination $1,2,3,4,5,6$
(in any order) called a \alert{straight}
\item A straight worth 1500~points
\item Combination of the form $x,x,y,y,z,z$
(in any order, $x,y,z$ distinct) called \alert{three-pair}
\item Three~pair worth 750~points
\item Each die showing 1 worth 100~points, each die showing 5 worth 50~points,
except\dots
\item Three 1s worth 1000~points, with each additional
1 worth 1000~points
\item Three $x$s worth $100x$~points ($x\ne 1$)
with each additional $x$ worth $100x$~points
\end{itemize}
\end{frame}

\begin{frame}{Scoring subsequent rolls}
\begin{itemize}
\item Subsequent rolls scored scored in same way as first,
except\dots
\item Dice rolled previously and retained cannot
contribute to combination formed by subsequent roll
\item So straight and three-pair not possible in subsequent
rolls, since require six dice
\begin{example}
\begin{itemize}
\item Suppose first roll results in $2,2,3,4,4,6$
\item Worth zero points, so turn over
\end{itemize}
\end{example}
\begin{example}
\begin{itemize}
\item Suppose first roll results in $1,1,3,4,5,6$
\item Worth 250~points, so player must reroll \alert{some}
dice
\item Could retain $1,1,5$ or could
retain $1,1$ or \dots
\end{itemize}
\end{example}
\item The fewer dice rerolled, the greater
the probability of farkling
\end{itemize}
\end{frame}

\begin{frame}
\begin{example}
\begin{itemize}
\item Suppose first roll yields $1,2,2,2,6,6$
\item Worth 300~points
\item Could end turn, banking 300~points
\item Could also reroll $6,6$, hoping for more 1s and 5s
\item \alert{Caution:}
probability of farkling with two dice somewhat high!
(calculated later)
\item Could also reroll $1,6,6$
\item Probability of farkling lower with three dice
\item \dots but retaining three 2s 
(worth only 200~points)
takes them out of circulation
\item Better to reroll $2,2,2,6,6$
\end{itemize}
\end{example}
\end{frame}

\begin{frame}
\begin{example}
\begin{itemize}
\item Suppose first roll yields $1,1,1,1,5,6$
\item Worth 2050~points!
\item Player should retain $1,1,1,1$
\item Risk-adverse players should also retain $5$ and end turn
\item Adventurous players might reroll~$5,6$ (hoping
for more 1s and 5s)
\item If both dice show~$1$ or $5$ player gets another
turn
\item If neither die shows~$1$ or $5$ player loses everything
\end{itemize}
\end{example}

\end{frame}

\begin{frame}{Pop quiz}
\[\begin{array}{r|llll|l}
&\text{same}&\text{same}&\text{different}&&\\
\text{education}&\text{county}
&\text{state}&\text{state}&\text{abroad}&\text{total}\\\hline
\text{no degree}&2247&470&254&98&3069\\
\text{high school}&3842&1074&712&145&5773\\
\text{some college}&3319&1020&707&112&5158\\
\text{bachelor's}&2072&760&667&182&3681\\
\text{graduate}&913&383&461&118&1875\\\hline
\text{total}&12,393&3707&2801&655&19,556
\end{array}\]
Calculate probability that
randomly selected individual
\begin{enumerate}
\item moved to different state given that she
has only high school diploma
\item does \alert{not} have graduate degree given that he moved
within same state
\item moved abroad \alert{and} has only bachelor's
\end{enumerate}
\end{frame}

\end{document}
